%!TEX TS-program = xelatex
%!TEX encoding = UTF-8 Unicode

\documentclass[14pt,a4paper]{extarticle}

%%%%%%% START COMMON PREAMBULA %%%%%%%%%

\usepackage{indentfirst} % Indent first paragraph after section header
\usepackage{mathtools} % Mathematical tools to use with amsmath

\usepackage{mathspec} % Specify arbitrary fonts for mathematics in XeTeX
\defaultfontfeatures{Mapping=tex-text}
\usepackage{xunicode} % Generate Unicode characters from accented glyphs
\usepackage{xltxtra} % “Extras” for LaTeX users of XeTeX
\usepackage{breqn} % Automatic line breaking of displayed equations

\setmainfont{Times New Roman}
\newfontfamily\cyrillicfont{Times New Roman}
\setmathsfont(Digits,Latin,Greek){Times New Roman}

\usepackage{polyglossia} % Multilingual support for XeLaTeX
\setdefaultlanguage{russian}

\usepackage[
  left=2cm,right=2cm,top=2cm,bottom=2cm
]{geometry} % Flexible and complete interface to document dimensions

%\usepackage{misccorr} % точка в номерах заголовков
\usepackage{tikz} % Create PostScript and PDF graphics in TeX
\usepackage{makeidx} % Standard LaTeX package for creating indexes
\usepackage[
  colorlinks=true, allcolors=blue,
]{hyperref} % Extensive support for hypertext in LaTeX
\usepackage{pgfplots} % Create normal/logarithmic plots
\pgfplotsset{width=15cm, compat=newest}
\usepackage{multirow} % Create tabular cells spanning multiple rows
\usepackage{hypcap} % Adjusting the anchors of captions
\usepackage{array} % Extending the array and tabular environments
\usepackage{amsthm} % Typesetting theorems (AMS style)
\usepackage{amssymb}

\usepackage{titlesec} % Select alternative section titles

\titleformat{\section}
  {\normalfont\Large\bfseries\centering}{\thesection. }{0pt}{}
\titleformat{\subsection}
  {\normalfont\large\bfseries\centering}{\thesubsection. }{0pt}{}
\titleformat{\subsubsection}
  {\normalfont\large\bfseries\centering}{\thesubsubsection. }{0pt}{}

\pagestyle{myheadings}

\newcolumntype{P}[1]{>{\centering\arraybackslash$} p{#1} <{$}}

\def\arraystretch{1.5}

\usepackage[nottoc]{tocbibind}
\usepackage{thmtools}
%\addto\captionsenglish{%
  %\renewcommand{\listtheoremname}{List of Definitions}%
%}

\newtheorem{theorem}{Теорема}[section]
\newtheorem{lemma}{Лемма}[section]
\newtheorem{corollary}{Следствие}[theorem]
\theoremstyle{definition}
\newtheorem{definition}{Определение}[section]
\theoremstyle{remark}
\newtheorem{example}{Пример}[section]

\renewcommand{\[}{\begin{dmath*}[compact]}
\renewcommand{\]}{\end{dmath*}}
\newcommand{\be}{\begin{enumerate}}
\newcommand{\ee}{\end{enumerate}}
\newcommand{\ds}{\displaystyle}
\newcommand{\sep}{ , \ \allowbreak }
\newcommand{\ivr}{\rule[-2.25ex]{0pt}{6ex}}
\newcommand\f[2]{\dfrac{#1}{#2}}
%%%%%%% END COMMON PREAMBULA %%%%%%%%%

\usepackage{tabularx} % Tabulars with adjustable-width columns
\usepackage{multicol} % Intermix single and multiple columns

%\usepackage{cmap} % Make PDF files searchable and copyable
%\usepackage{graphicx} % Enhanced support for graphics
%\usepackage{psfrag} % Replace strings in encapsulated PostScript figures
%\usepackage{caption2} % Customising captions in floating environments
%\usepackage{soul} % Hyphenation for letterspacing, underlining, and more
%\usepackage{soulutf8} % Permit use of UTF-8 characters in soul
%\usepackage{fancyhdr} % Extensive control of page headers and footers
%\usepackage{ltxtable} % Longtable and tabularx merge
%\usepackage{paralist} % Enumerate and itemize within paragraphs
%\usepackage{longtable} % Allow tables to flow over page boundaries
%\usepackage{accsupp} % Better accessibility support for PDF files

\makeatletter
\newcommand{\sigmalgebra}{\text{\textcircled{$\sigma$}}}
\newcommand\eeq[1][]{\stackrel{\mathclap{\normalfont\mbox{#1}}}{=}}
\makeatother

\title{\textbf{Функциональный анализ}}
\author{Годес Ю.Я.}
\date{}

%\usepackage{geometry}
%\geometry{a4paper, landscape, total={277mm, 190mm}, left=10mm, top=10mm}
%\pgfplotsset{
  %width=52mm,
  %compat=1.13,
  %axis lines=middle,
  %xtick=\empty,
  %ytick=\empty,
%}
%\usepackage{titlesec}
%\titleformat{\section}{\normalfont\scriptsize\bfseries\raggedright}
  %{\makebox[30pt][l]{\thesection}}{0pt}{}
%\titleformat{\subsection}{\normalfont\scriptsize\bfseries\raggedright}
  %{\makebox[30pt][l]{\thesubsection}}{0pt}{}
%\titleformat{\subsubsection}{\normalfont\scriptsize\bfseries\raggedright}
  %{\makebox[30pt][l]{\thesubsubsection}}{0pt}{}
%\usepackage{enumitem}
%\renewcommand{\be}{\begin{enumerate}[leftmargin=.2cm]}
%\setlength\itemindent{0cm}
%\setlength{\parindent}{.2cm}

\begin{document}
%\begin{multicols}{6}
%\scriptsize
\sloppy
\maketitle

\tableofcontents

\section{Метрические пространства.} \label{sec:mp}

\textbf{Лекция №1.}

Под пространством в современной математике понимается совокупность объектов
произвольной природы (функции, набор чисел, набор функций, объекты
геометрической природы) между которыми установлены соотношения аналогичные
пространственным соотношениям в трехмерном пространстве.

К понятию метрического пространства приходят путем определения расстояния между
элементами произвольного множества (путем определения метрики).

\subsection{Определение метрического пространства.
Примеры метрических пространств.}

\begin{definition}[метрическое пространство]
  Множество $X$ называется метрическим пространством,
  если для любых двух его элементов $x, y \in X$
  определена однозначная действительная функция
  $\rho(x, y)$ - расстояние между элементами.
  $x$ и $y$ (иначе метрика), которая удовлетворяет следующим условиям:

  \be
    \item Аксиома неотицательности:
    \[ {\rho(x, y) \geq 0} \sep {\forall x, y \in X} \sep
    {\rho(x, y) = 0} \Leftrightarrow {x = y} \]
    \item Аксиома симметрии:
    \[{\rho(x, y) = \rho(y, x)} \sep {\forall x, y \in X}\]
    \item Аксиома треугольника:
    \[\rho(x, y) \leq {\rho(x, z) + \rho(z, y)} \sep
    {\forall x, y, z \in X}\]
  \ee
\end{definition}

Рассмотрим некоторые примеры метрического пространства.

При доказательстве справедливости аксиомы треугольника для этих пространств
необходимо использовать неравенство Гельдера, Минковского.

Неравенство Гельдера:
\[
  \left|\sum_{n = 1}^\infty a_n b_n\right| \leq\allowbreak
  {\left(\sum_{n = 1}^\infty |a_n|^p\right)^{\f{1}{p}}}\cdot
  {\left(\sum_{n = 1}^\infty |b_n|^q\right)^{\f{1}{q}}}\sep
  {\f{1}{p} + \f{1}{q} = 1} \sep {p \geq 1}
\]
\[
  {\left(\int_a^b |x(t) y(t)|^p dt \right)^{\f{1}{p}}} \leq\allowbreak
  {\left(\int_a^b |x(t)|^p dt \right)^{\f{1}{p}}} \cdot
  {\left(\int_a^b |y(t)|^p dt \right)^{\f{1}{p}}}
\]

Неравенство Минковского:
\[
  {\left(\sum_{n=1}^\infty|a_n+b_n|^p\right)^{\f{1}{p}}} \leq\allowbreak
  {\left(\sum_{n=1}^\infty|a_n|^p\right)^{\f{1}{p}}} +
  {\left(\sum_{n=1}^\infty|b_n|^p\right)^{\f{1}{p}}}\sep {p\geq 1}
\]
\[
  {\left(\int_a^b|x(t)+y(t)|^p dt \right)^{\f{1}{p}}} \leq\allowbreak
  {\left(\int_a^b|x(t)|^p dt \right)^{\f{1}{p}}} +
  {\left(\int_a^b|y(t)|^p dt \right)^{\f{1}{p}}}
\]

Эти неравенства справедливы при условии сходимости всех входящих в них рядов и
интегралов.

Неравенства справедливы также и для конечных сумм.
Интегралы в этих неравенствах понимаются в смысле Лебега.

\subsubsection{Примеры метрических пространств.}

\be
  \item $R_p^n$ - $n$-мерное арифметическое пространство элементов, которое
  является упорядоченным набором действительных чисел.
  \[x = (x_1, \dots, x_n) \in R_p^n\]
  Метрика в этом пространстве определяется равенством:
  \[\rho(x, y) = {\left(\sum_{k = 1}^n |y_k - x_k|^p\right)^{\f{1}{p}}} \sep
  {p \geq 1}\]

  При $p = 2$ получается $n$-мерное евклидово пространство.
  Сразу понятно, что на одном и том же множестве метрику можно ввести
  разными способами.

  \item $R_\infty^n$ - $n$-мерное арифметическое пространство с метрикой:
  \[\rho(x, y) = \max_{k = 1, 2, \dots, n} |y_k - x_k|\]

  Можно доказать, что:
  \[\lim_{p \to \infty} \rho_p(x, y) = \rho_\infty(x, y)\sep\]
  где $\rho_p$ - метрика в $R_p^n$, а $\rho_\infty$ - метрика в $R_\infty^n$.

  \item $l_p$ - элементами этого пространства является бесконечные числовые
  последовательности $x=(x_1,x_2,\dots,x_n,\dots)\in l_p$ такие,
  что сходится ряд:
  \[\sum_{n=1}^\infty|x_n|^p<\infty.\]
  Метрика в этом пространстве определяется равенством:
  \[\rho(x, y) = \left(\sum_{n=1}^\infty|y_n-x_n|^p\right)^{\f{1}{p}} \sep
  {p \geq 1}\]
  $l_2$ - бесконечное евклидово пространство.

  \item $l_\infty$ - пространство ограниченных последовательностей,
  элементами этого пространства являются бесконечные числовые последовательности
  такие, что:
  \[\sup_{n=1,2,\dots}|x_n|<\infty \sep \rho(x,y)=\sup_{n=1,2,\dots}|y_n-x_n|\]

  \item Пространство непрерывных функций $C[a, b]$, элементами этого
  пространства являются все возможные функции $x(t)$ непрерывные на отрезке
  $[a, b]$.
  Метрика вводится следующим образом:
  \[ \rho (x, y) = \max_{a \leq t \leq b} |y(t) - x(t)| \]

  Такая метрика называется Чебышевской.
  Можно также рассматривать пространство непрерывных функций $C_2[a, b]$
  \[ \rho(x,y) = \left(\int_a^b |y(t)-x(t)|^2 dt\right)^{\f{1}{2}} < \infty \]

  \item $L_p[a,b]\sep p \geq 1$ - пространство функций интегрируемых по Лебегу в
  $p$-той степени на $[a,b]$. То есть таких функций, что существует:
  \[ \int_a^b|x(t)|^p dt \]
  С метрикой:
  \[ \rho(x,y) = \left(\int_a^b|y(t)-x(t)|^pdt\right)^{\f{1}{p}} \]

  \item $L_\infty[a, b]$ - пространство ограниченных функций, элементами этого
  пространства являются функции $x(t)$ такие, что существует:
  \[ \sup_{a \leq t \leq b} |x(t)| \]
  С метрикой:
  \[ \rho(x,y) = \sup_{a \leq t \leq b} |y(t)-x(t)| \]
\ee

\subsection{Предельные точки множества. Сходимость в метрическом пространстве.}

Важнейшим понятием анализа является понятие предела.
Многие факты анализа основаны исключительно на существовании расстояния между
любыми двумя точками числовой прямой и не затрагивает алгебраическую
природу действительных чисел.
Поэтому можно обобщить понятия предела для произвольного метрического
пространства.

\textbf{Лекция №2.}

\begin{definition}[открытый шар как множество]
  Открытым шаром радиуса $r$ с центром $x_0$ называют множество точек
  метрического пространства $X$, удовлетворяющее условию $\rho(x, x_0) < r$.
\end{definition}

\begin{definition}[открытый шар как окрестность]
  Открытым шаром радиуса $r$ с центром $x_0$ также называют
  $r$-окрестностью точки $x_0$ и обозначают $O_r(x_0)$.
  \[ O_r(x_0) = {\{x \in X: \rho(x, x_0) < r\}} \]
\end{definition}

\begin{definition}[замкнутый шар]
  Замкнутым шаром радиуса $r$ с центром $x_0$ называют множество точек
  метрического пространства $X$,
  удовлетворяющее условию: $\rho(x, x_0) \leq r$
\end{definition}

Рассмотрим некоторое множество $M \subset X$.

\begin{definition}[изолированная точка]
  Точка $x \in M$ называется изолированной точкой множества $M$,
  если существует окрестность $O_\epsilon(x)$
  этой точки не содержащая других точек множества $M$.
\end{definition}

\begin{definition}[предельная точка]
  Точка $x \in X$ называется предельной точкой множества $M \subset R$,
  если любая окрестность этой точки содержит бесконечное число
  элементов множества $M$.
\end{definition}

Предельная точка множества $M$ может принадлежать этому множеству, а может
и не принадлежать ему.

\begin{definition}[замыкание множества]
  Замыканием множества $M$ называется множество,
  которое получается в результате добавления к нему предельных точек,
  и обозначается $[M]$.
\end{definition}

Замыкание $[M]$ состоит из точек трех типов: изолированные точки, предельные
точки входящие в $M$, предельные точки не входящие в $M$.

\subsubsection{Сходимость в метрическом пространстве.}

Рассмотрим последовательность точек $\{x_n\}_{n=1,2,\dots}\subset X$.

\begin{definition}[предел последовательности]
  Говорят, что $\ds\lim_{n \to \infty} x_n = x$, если
  $\forall \epsilon > 0 \sep \exists N = N(\epsilon): \rho(x_n, x)<\epsilon \sep
  \forall n \geq N$, иначе записывают $x_n \underset{x}{\to} x$
\end{definition}

Это означает, что для любого $\epsilon > 0$, $O_\epsilon(x)$ содержит все
члены последовательности начиная с некоторого номера. Непосредственно из
определения предела вытекает:

\be
  \item Единственность предела: если последовательность имеет предел,
  то он единственный.

  \item Если последовательность сходится к некоторому пределу,
  то любая ее подпоследовательность сходится к этому же пределу.
\ee

Доказательство идентично доказательству для числовых пределов.

\begin{theorem}
  Для того чтобы точка $x \in X$ была предельной точкой множества M
  необходимо и достаточно, чтобы существовала последовательность элементов
  $\{x_n\}\subset M$ такая, что $x_n\neq x \sep x_n\to x$
\end{theorem}

\subsection{Открытые и замкнутые множества в метрическом пространстве}

\begin{definition}[замкнутое множество]
  Множество $M$ метрического пространства $X$ называют замкнутым,
  если оно содержит все свои предельные точки $[M]=M$.
  Замкнутые множества совпадают со своими предельными точками.
\end{definition}

\begin{definition}[внутренняя точка]
  Точка $x \in M$ называется внутренней точкой этого множества,
  если существует окрестность этой точки $O_\epsilon(x)\subset M$.
\end{definition}

\begin{definition}[открытое множество]
  Множество $M$ метрического пространства $X$ называют открытым,
  если все его точки является внутренними.
\end{definition}

\begin{theorem}
  Для того, чтобы множество $M \subset X$ было открытым (замкнутым)
  необходимо и достаточно, чтобы его дополнение для всего пространства $X$,
  $X \backslash M$, было замкнутым (открытым).
\end{theorem}

\begin{proof}
  Пусть множество $M$ открытое и $x \in M$ произвольная точка.
  Так как множество $M$ открытое, то точка $x$ - внутренняя, то есть существует
  $O_\epsilon(x)\subset M$. Следовательно, эта окрестность не входит в
  $X \backslash M$, следовательно $X \backslash M$ - замкнутое.
  Пусть $M$ - замкнутое, возьмем произвольную точку $x \notin M$, отсюда следует
  что $x\in X\backslash M$, следовательно существует $O_{\epsilon}(X)$, которая
  не содержит ни одного элемента из $M$.
  $O_{\epsilon'}(x)\subset X\backslash M$.
  То есть любая точка $x \notin M$ является внутренней точкой для
  $X \backslash M$.
\end{proof}

\subsubsection{Свойства открытых и замкнутых множеств.}

\be
  \item Пересечение любого числа замкнутых множеств является
  замкнутым множеством.
  \[ \bigcap _\alpha \underset {\text{замк.}} {M_\alpha}\text{ - замкнутое.} \]

  \item Объединение конечного числа замкнутых множеств является
  замкнутым множеством.
  \[ \bigcup _{k = 1}^n \underset {\text{замк.}} {M_k}\text{ - замкнутое.} \]

  \item Объединение любого числа открытых множеств является
  открытым множеством.
  \[ \bigcup _{\alpha} \underset {\text{откр.}} {M_\alpha}\text{ - открытое.} \]

  \item Пересечение конечного числа открытых множеств является
  открытым множеством.
  \[ \bigcap _{k = 1}^n \underset {\text{откр.}} {M_k}\text{ - открытое.} \]

  \item Замыкание $[M]$ является наименьшим замкнутым множеством содержащем $M$.

  \item Отрытый шар в метрическом пространстве является открытым множеством.
  Замкнутый шар в метрическом пространстве является замкнутым множеством.
\ee

\begin{proof}[Доказательство свойства 1]
  Пусть все $M_\alpha$ - замкнутые, $x$ - произвольная
  предельная точка $\bigcap _\alpha M_\alpha$ пересечения.
  Тогда любая ее окрестность содержит бесконечно много точек из каждого
  $M_\alpha$, то есть $x$ является предельной точкой для каждого $M_\alpha$.
  Так как все $M_\alpha$ - замкнутые, то $x\in M_\alpha$, а тогда и
  $x \in \bigcap _\alpha M_\alpha$, то есть $\bigcap _\alpha M_\alpha$ замкнуто.
\end{proof}

\begin{proof}[Доказательство свойства 2]
  Пусть все $M_k \sep k=\overline{1,n}$ замкнутые множества,
  точка $x \notin \bigcup _{k = 1}^n M_k$, тогда $x \notin M_k$,
  поэтому $x$ не является предельной точкой ни для одного замкнутого $M_k$.
  Следовательно $\forall M_k \sep \exists \epsilon_k > 0: O_{\epsilon_k}(x)$
  содержит не более чем конечное число элементов $M_k$.
  Возьмем $\epsilon = \min_{k=1,2,\dots,n} \epsilon_k$.
  Тогда $O_\epsilon(x)$ содержит не более чем конечное число элементов
  $\bigcup _{k = 1}^n M_k$.
  Следовательно она не является предельной точкой множества,
  таким образом $\bigcup _{k = 1}^n M_k$ замкнутое.
\end{proof}

\begin{proof}[Доказательство свойства 3]
  \[ \bigcup _\alpha \underset{\text{откр.}}{M_\alpha} =
  \bigcup _\alpha\left(X\backslash\underset{\text{замк.}}{F_\alpha} \right) =
  X\backslash \underbrace{\bigcap _\alpha F_\alpha}_{\text{замк.}}
  \text{ - открытое} \]
\end{proof}

\subsubsection{Примеры отрытых и замкнутых множеств.}
\be
  \item $R^1$ - прямая, $(a, b)$ - открытое, $[a, b]$ - замкнутое.

  \item $C[a, b]\sep |x(t)| < K \sep \forall t \in [a, b]$

  $x(t) \in C[a, b]$ - множество функций с центром 0 и радиусом K.

  $y(t) \in C[a, b]$ - непрерывные фиксированные функции

  $x(t) < y(t) \sep \forall t \in [a, b]$ - открытое множество

  $x(t) \leq y(t) \sep \forall t \in [a, b]$ - замкнутое множество
\ee

В произвольном метрическом пространстве структура открытых и замкнутых
множеств весьма сложна.
Исчерпывающие описание структуры открытого и замкнутого множества можно дать
только на одной числовой прямой в $R^1$.

\begin{theorem}
  Всякое открытое множество на действительной оси представляет собой
  объединение конечного или счетного числа попарно не пересекающихся интервалов.
\end{theorem}

\subsection{Плотные подмножества. Сепарабельные пространства.}

\textbf{Лекция №3.}

Рассмотрим два множества $A, B\subset X$.

\begin{definition}[плотное множество]
  Множество $A$ называется плотным в $B$, если $B \subset [A]$.
\end{definition}

\begin{definition}[всюду плотное множество]
  Множество $A$ называется всюду плотным в метрическом
  пространстве $X$, если его замыкание совпадает с метрическим пространством
  пространством $X$, то есть $[A]=X$.
\end{definition}

Если множество $A$ всюду плотно в пространстве $X$, то:

\be
  \item Любая окрестность любой точки $x \in A$ пространства содержит хотя
  бы одну точку множества $A$, отличной от $x$.

  \item Для любой точки пространства $x \in X$  существует последовательность
  элементов ${x_n} \subset A, x_n \to x$.
\ee

Например, на действительной оси всюду плотно множество рациональных чисел.

\begin{definition}[сепарабельное метрическое пространство]
  Метрическое пространство называется сепарабельным,
  если в нем существует счетное всюду плотное множество.
\end{definition}

Множество называется счетным если между этим множеством и множеством
натуральных чисел можно установить взаимно однозначное соответствие.

Укажем счетные, всюду плотные множества в некоторых метрических пространствах:

\begin{enumerate}
  \item На действительной оси всюду плотно множество рациональных чисел $R_n^1$.

  \item В арифметических пространствах $R_n^p, R_\infty^n$ всюду плотно
  множество векторов с рациональными координатами.

  \item В пространстве непрерывных функций С[a,b] всюду плотно множество
  многочленов с рациональными коэффициентами.

  \item В пространстве лебеговых функций $L_p[a,b]$ счетным, всюду плотным
  множеством является множество многочленов с рациональными коэффициентами (в
  последствии мы покажем что множество непрерывных функций всюду плотно в
  $L_p[a,b]$) %???

  \item В пространстве последовательностей $l_p$ счетным, всюду плотным
  множеством является множество последовательностей с рациональными членами в
  которых только конечное число членов отличное от нуля. %???
\end{enumerate}

Все перечисленные пространства являются сепарабельными.

Ограниченное пространство $l_{\inf}$ не является сепарабельным.

\subsection{Непрерывные отображения метрических пространств.
Гомеоморфизм и изометрия.}

Рассмотрим два метрических пространства X, Y
\[\rho_X(x_1,x_2)\sep x_1,x_2\in X \]
\[\rho_Y(y_1, y_2)\sep y_1,y_2\in Y \]
\[ {f: X \to Y}\]

Вместо термина ``отображение'' используется термины ``функция'' и
``оператор'', которые являются экивалентными.
В разных разделах функционального анализа сложились свои традиции
использования этого термина.

\begin{definition}[функционал]
  Функционалом называется числовая функция определенная на
  некотором метрическом пространстве.
  Иначе функционал это отображение некоторого метрического
  пространства на действительную ось (или на комплексную плоскость).
\end{definition}

\begin{definition}[непрерывное отображение]
  Отображение $f: X \to Y$, называется непрерывным в точке $x_o \in X$,
  если $ \forall \epsilon > 0 \sep \exists \delta = \delta (\epsilon) > 0:
  \rho_y(f(x), f(x_0)) < \epsilon$ при $\rho_x(x,x_0)<\delta \sep
  \forall x \in X$
\end{definition}

Эквивалентное определение непрерывности на языке последовательностей:
\begin{definition}[непрерывное отображение на языке последовательностей]
  $f: X \to Y$ называется непрерывным в точке $x_0 \in X$, если
  $\forall \{x_n\}\subset X: f(x_n) \xrightarrow[Y]{} f(x_0)$
  при $x_n\xrightarrow[X]{} x_0$
\end{definition}

Доказательство этих определений непрерывности дословно воспроизводят
соответствующие доказательства для функций соответствующих переменных.

Пусть f взаимно однозначное отображение пространства $X$ на $Y$,
$f: X \to Y$, тогда существует обратное отображение
$f^{-1}: Y\to X$.

\begin{definition}[гомеоморфизм или гомеоморфное отображение]
  Отображение $f$ называется гомеоморфизмом или гомеоморфным отображением,
  если оно взаимно однозначно и взаимно непрерывно.
\end{definition}

\begin{definition}[изометрия или изометрическое отображение]
  Отображение $f: X \to Y$ называется изометрией или
  изометрическим отображением,
  если оно взаимно однозначное и оно сохраняет расстояние между точками.
  \[ \rho_Y(f(x_1,x_2))=f_X(x_1,x_2) \sep {\forall x_1,x_2\in X}\]
\end{definition}

Изометрия является частным случаем гомеоморфизма, два пространства между
которыми можно установить гомеоморфизм (изометрию) называются гомеоморфными
(изометрическими). Если два пространства изометрические, то метрические
соотношения между их элементами идентичны, различаться может только природа
этих элементов, что с точки зрения теории метрических пространств несущественно.

\subsection{Полные метрические пространства}

В анализе существенную роль играет свойство полноты числовой прямой, то есть
что любая фундаментальная последовательность действительных чисел является
сходящейся.

\begin{definition}[фундаментальная последовательность]
  Последовательность называется фундаментальной,
  если она удовлетворяет условию Коши,
  то есть расстояние между двумя элементами последовательности становится сколь
  угодно малым с некоторого номера.
\end{definition}

Рассмотрим последовательность элементов $\{x_n\} \in X$,
последовательность называется фундаментальной если она удовлетворяет
условию Коши, то есть $ \forall \epsilon > 0 \sep \exists N = N(\epsilon):
\rho(x_n, x_m) < \epsilon \sep \forall n, m \geq N $.

Любая сходящаяся последовательность является фундаментальной.

\begin{proof}
  Пусть $x_n \to x$, тогда
  $\forall \epsilon > 0 \sep \exists N: \rho(x_n, x) < \epsilon \sep
  \forall n \geq N$,
  тогда согласно аксиоме треугольника
  $\rho(x_n, x_m) \leq \rho(x_n, x)+\rho(x, x_m) < 2\epsilon \sep
  \forall n, m \geq N$,
  что и доказывает фундаментальность последовательности.
\end{proof}

Обратное утверждение в произвольном метрическом пространстве неверно.
Например \[ {X = \{x \in R^1: a < x < b\}} \sep {x_n = a + \f{1}{n}}\] %???

\begin{definition}[полное метрическое пространство]
  Метрическое пространство называется полным если любая
  фундаментальная последовательность его элементов является сходящейся.
\end{definition}

Полными являются следующие метрические пространства:

\begin{enumerate}
  \item $R^1$
  \item $R_p^n, R_\infty^n$, так как сходимость в этих пространствах означает
  по координатную сходимость, а пространство $R^1$ является полным
  \item $l_p, l_\infty$ - пространство последовательностей %???
  \item $L_p[a,b]$ %???
  \item $L_\infty$ %???
  \item $C[a,b]$
\end{enumerate}

\begin{theorem}
  Пространство непрерывных функций $C[a,b]$ является полным,
  а сходимость в этом пространстве означает равномерную сходимость.
\end{theorem}

\begin{proof}
  Рассмотрим произвольную фундаментальную последовательность
  непрерывных функций: $\{x_n(t)\} \subset C[a, b]$, тогда:
  \[ \forall \epsilon > 0 \sep \exists N: \rho(x_n, x_m) =
  \max_{a \leq t \leq b} |x_n(t) - x_m(t)| < \epsilon \sep
  {\forall n, m \geq N} \]
  легко можно узнать известный критерий Коши равномерной сходимости
  функциональной последовательности, в соответствии с этим критерием
  \[ x_n(t) \overset{[a,b]}{\rightrightarrows} x(t)\]
  причем функция $x(t)$ непрерывная $x(t)\in C[a,b]$
\end{proof}

\textbf{Лекция №4}

Докажем теперь что $x_n(t)$ сходится к $x(t)$, $x_n(t) \to x(t)$,
так как последовательность сходиться равномерно, то
$ \forall \epsilon > 0 \sep \exists N = N(\epsilon): |x_n(t) - x(t)| <
\epsilon \sep \forall t \in [a,b] \sep \forall n \geq N$
отсюда сразу вытекает:
\[ \rho(x_n, x) = \max_{a \leq t \leq b} |x_n(t) - x(t)| < \epsilon \sep
{\forall n \geq N} \sep {\lim_{n \to \infty} \rho(x_n,x) = 0}\]

Пополнение метрического пространства: оказывается что любое неполное
метрическое пространство можно включить в полное.

\begin{definition}[пополнение метрического пространства]
  Пространство $\bar X$ называется пополнением X, если $X \subset \bar{X}$ и $X$
  всюду плотно в $\bar X$: $[X] = \bar X$
\end{definition}

Например: действительная ось это пополнение пространства действительных чисел.
Пополнением пространства $X = \{x \in R^1: a < x < b\}$ является
$\bar{X} = \{x \in R^1: a \leq x \leq b\}$

\begin{theorem}[о пополнении]
\label{th:о пополнении}
  Любое неполное метрическое пространство $X$ можно
  включить в полное в качестве всюду плотного подмножества,
  причем такое пополнение единственное с точностью до изометрии,
  оставляющей неподвижными точки неполного пространства $X$.
\end{theorem}

На действительной оси имеет место лемма о вложенных отрезках. Пусть дана
последовательность вложенных отрезков $[a_n, b_n]: a_1 \leq a_2 \leq \dots \leq
a_n \leq \dots \leq b_n \leq b_{n-1} \leq \dots \leq b_1 $, тогда пересечение
этих отрезков является не пустым множеством и пересечение состоит из одной
точки, при условии что длина отрезка стремится к нулю:
$\ds \lim_{n \to \infty}(b_n - a_n) = 0$.

Аналогом этого свойства для произвольного полного метрического пространства
является теорема о вложенных шарах.

\begin{theorem}
  Для того чтобы метрическое пространство $X$ было полным,
  необходимо и достаточно, чтобы пересечение любой последовательности вложенных
  замкнутых шаров состояло из единственной точки.
\end{theorem}

\begin{proof}[Доказательство необходимости.]
  Пусть метрическое пространство является полным, то есть любая фундаментальная
  последовательность его элементов сходится. Рассмотрим последовательность
  вложенных замкнутых шаров
  $B_{1} \supset B_{2} \supset \dots \supset B_{n} \supset \dots$
  c центрами $x_{n}$ и радиусами $r_{n}$, последовательность $\{x_n\}$ является
  фундаментальным потому что $\rho(x_n, x_m) < r_n\sep m > n$,
  а $r_n \to 0$ при $n \to \infty$

  Так как пространство полное, то $x_{n}$ сходится к некоторой точке $x$,
  причем точка x принадлежит пересечению каждого из шаров
  $x \in \bigcap _{n = 1}^{\infty} B_{n}$ (так как точка $x$
  предельная точка для каждого шара, а все шары замкнуты).

  Для доказательства единственности предположим что существуют две точки $x, y$
  принадлежащие пересечению шаров
  $x, y \in \bigcap _{n = 1}^{\infty} B_{n} \sep \rho(x, y) = \epsilon > 0$,
  так как $r_n \underset{n \to \infty}{\to} 0$ то для указанного
  $\epsilon \sep \exists N: r_N < \f\epsilon2$
  то есть точки x, у не могут одновременно принадлежать шару $B_N$, таким образом
  получили противоречие и доказали единственность.
\end{proof}

\begin{proof}[Доказательство достаточности.]
  Предположим что пресечение любой последовательности вложенных
  замкнутых шаров состоит из единственной точки,
  докажем что в этом случае пространство является полным.
  Для этого рассмотрим произвольную фундаментальную последовательность
  $\{x_{n}\} \subset X$, так как последовательность фундаментальная, то
  $\forall \epsilon > 0 \sep \exists N: \rho(x_n, x_N) < \epsilon \sep
  \forall n > N$, отсюда следует что можно указать последовательность номеров
  $n_1 < n_2 <\dots < n_k < \dots $, обладающую следующим свойствами:

  \begin{dgroup*}
  \[\rho(x_n, x_{n_1}) < \f{1}{2}\sep {\forall n > n_{1}} \]
  \[\rho(x_n, x_{n_2}) < \f{1}{2^{2}}\sep {\forall n > n_{2}} \]
  \[\dots\]
  \[\rho(x_n, x_{n_k}) < \f{1}{2^{k}}\sep {\forall n > n_{k}}\]
  \end{dgroup*}

  Рассмотрим последовательности замкнутых шаров $B_k$ с центром в точке
  $x_{n_k}$ и радиусом $r_n = \f{1}{2^{k}}$, по условию теоремы эта
  последовательность шаров имеет не пустое пересечение,
  состоящее из одной точки $x$, таким образом
  последовательность $x_n$ содержит сходящуюся подпоследовательность
  $x_{n_k} \to x$, а если фундаментальная последовательность содержит
  сходящуюся подпоследовательность, то и сама последовательность сходится, что и
  доказывает полноту пространства.
\end{proof}

\subsection{Принцип сжимающих отображений}

Рассмотрим отображение A метрического пространства $X$ в себя:
$A: X \to X \sep \forall x \to Ax$.

\begin{definition}[отображение сжатия или сжимающее отображение]
  Отображение A называется отображением сжатия или сжимающим отображением,
  если $\exists \alpha \in [0, 1]:
  \rho(Ax, Ay) \leq \alpha \rho(x,y) \sep \forall x, y \in X$.
\end{definition}

Натуральные степени отображения $A$ определяются следующим образом
$A^nx = A^{n-1}Ax \sep {n=2, 3, \dots}$, легко видеть
$ \rho(A^nx, A^ny) \leq \alpha^n \rho(x, y)\sep {\forall x, y \in X}$.%???

\begin{definition}[неподвижная точка отображения]
  Точка $x$ пространства называется неподвижной точкой отображения,
  если $Ax = x$.
\end{definition}

\begin{theorem}[принцип сжимающих отображений или теорема о неподвижной точке]
  Любое отображение сжатия в полном метрическом пространстве имеет единственную
  неподвижную точку.
\end{theorem}

Иными словами, если $Ax$ отображение сжатия, то уравнение
$Ax = x$ имеет единственное решение в метрическом пространстве $X$.

\begin{proof}
  Доказательство конструктивное, то есть дает алгоритм построения этой
  неподвижной точки.

  Возьмем произвольную точку $x_0 \in X$, построим последовательность
  $x_1 = Ax_0 \sep x_2 = Ax_1 = A^{2}x_{0}$ и т.д.; вообще
  $x_n = Ax_{n-1} = A^{n}x_{0}$

  Докажем что эта последовательность является фундаментальной, причем будем
  считать что $m > n$.
  \[\rho(x_n, x_m)=\rho(A^nx_o, A^n\underbrace{A^{m-n}x_0}_{x_{m-n}})
    \leq \alpha^n \rho(x_0, x_{m-n})
    \leq \alpha^n (\rho(x_0, x_1) + \rho(x_1, x_2) + \dots +
    \rho(x_{m-n-1}, x_{m-n})) =
  \alpha^n (\rho(x_0, x_1) + \rho(Ax_0, Ax_1) + \dots +
  \rho(A^{m-n-1}x_0, A^{m-n-1}x_1)) \leq
  \alpha^n (1 + \alpha + \dots + \alpha^{m-n-1})\rho(x_0,x_1) \leq
  \alpha^n (1 + \alpha + \dots) \rho(x_0, x_1) =
    \f{\alpha^n}{1 - \alpha} \rho(x_0, x_1) \to 0\]
  Что и доказывает фундаментальность последовательности $x_{n}$.

Так как пространство является полным то последовательность
$x_{n} \to x \in X$, то есть $\forall \epsilon > 0 \sep
\exists N: \rho(x_{n}, x) < \epsilon\sep \forall n \geq N$.

\textbf{Лекция №5}

\begin{dgroup*}
\[{\rho(Ax_{n},Ax)\leq\alpha\rho(x_{n},x_{n})<\alpha\epsilon}\sep
{\forall n \geq N} \]
\[{\lim_{n\to \infty}\underbrace{Ax_{n}}_{x_{n+1}}=Ax=x} \]
\[{\lim_{n\to \infty}{x_{n+1}}=x}\]
\end{dgroup*}
где $Ax$ - неподвижная точка

Единственность неподвижной точки докажем от противного, пусть у нас две точки
$x, y:\quad Ax=x\sep Ay=y$, тогда:
\[ \underbrace{\rho (A_{x}, A_{y})}_{\rho(x, y)}\leq\alpha\rho(x, y)\]

Поэтому $(1-\alpha)\rho(x,y) \leq 0\sep \rho(x, y) \leq 0\sep x=y$.
\end{proof}

\begin{theorem}[обобщение принципа сжимающих отображений]
  Пусть $A$ непрерывное отображение пространства $X$ в себя $A: X\to X$, такое
  что некоторая натуральная степень этого отображения является отображением
  сжатия, тогда отображение $A$ имеет единственную неподвижную точку.
\end{theorem}

\begin{proof}
  Так как $B$ отображение сжатия, то согласно принципу отображения сжатия:
  $\ds {\lim_{n\to \infty}B^{n}x_{0}=x=Bx}\sep{\forall x_{0}\in X}$

  Тогда используя непрерывность отображения $A$ и тот факт,
  что $B=A^{m}$ получаем:
  \[ x = \lim_{n\to \infty}B^{n}Ax =
  \lim_{n\to \infty}AB^{n}x =
  A\lim_{n\to \infty}B^{n}x = Ax\]

  Неподвижная точка отображения $A$ является неподвижной и для отображения $B$,
  а для неподвижной точки отображения $B$ единственность доказана.
\end{proof}

\subsection{Простейшие применения принципа сжимающих отображений}

\subsubsection{Решения уравнения \texorpdfstring{$x=f(x)$}{Lg}}

Рассмотрим функцию $f(x)$, удовлетворяющую на отрезке $[a, b]$ условию Липшица
с константой K < 1:
\[ |f(x_1)-f(x_2)|\leq K|x_1-x_2|\sep{\forall x_1, x_2 \in [a, b]} \sep
{0<K<1}\]

Эта функция задает отображение метрического пространства $[a, b]$ в себя:
$f:[a,b] \to[a, b]$.
Выясним при каких условиях это отображение является отображением сжатия.
Согласно принципу сжимающих отображений уравнение $x=f(x)$ имеет
единственное решение и это решение может быть получено
методом последовательных приближений $x_{n+1}=f(x_n), x_{n+1} \in [a, b]$,
а в качестве начального приближения возьмем $x_0 \in [a, b]$.

\begin{tikzpicture}[>=stealth]
\begin{axis}[
  restrict y to domain=-1:5,
  samples=20,
  ticks=none,
  xmin = -0.5, xmax = 4,
  ymin = -0.5, ymax = 4,
  unbounded coords=jump,
  axis x line=middle,
  axis y line=middle,
  xlabel={$x$},
  ylabel={$y$},
  declare function={f(\x)=(\x-1)^2/2+0.5;},
  ]
  \addplot[thick, blue, domain=0.5:3.5] {f(x)};
  \addplot[thick, blue, domain=0:4] {x};

  \draw[dotted] (axis cs:0.75,0.75) -- (axis cs:0.75,0) node[below] {$a$};
  \draw[dotted] (axis cs:3,3) -- (axis cs:3,0) node[below] {$b$};

  \draw[dotted] (axis cs:2.75, {f(2.75)})
             -- (axis cs:2.75,0) node[below] {$x_0$};
  \draw[dotted] (axis cs:2.75, {f(2.75)})
             -| (axis cs:{f(2.75)}, 0) node[below] {$x_1$};
  \draw[dotted] (axis cs:{f(2.75)}, {f(f(2.75))})
             -| (axis cs:{f(f(2.75))}, 0) node[below] {$x_2$};
\end{axis}
\end{tikzpicture}

Решение уравнения $F(x)=0$ на отрезке $[a,b]$, причем будем считать
$F(a)<0,\; F(b)>0$.
Функция $F(x)$ имеет непрерывную производную на отрезке $[a,b]$,
это легко сходится к предыдущему случаю, рассмотрим функцию
\begin{dgroup*}
\[ f(x)=x-\lambda F(x) \sep \lambda \neq 0 \sep {x=f(x)}\]
\[ 1 - \lambda K_2 \leq f'(x) = 1 - \lambda F'(x) \leq 1 - \lambda K_1\]
\end{dgroup*}

При достаточно малых $\lambda$, f является отображением сжатия,
так как непрерывно дифференцируемая функция заведомо удовлетворяет
условию Липшеца, поэтому при достаточно малых $\lambda$ при решении
уравнения $F(x)=0$ можно использовать метод последовательных приближений.

\subsubsection{Линейное отображение \texorpdfstring{$R^n \to R^n$}{Lg}}

Точка $y=Ax\sep y=(y_1,\dots,y_n) \in R^n\sep x=(x_1,\dots,x_n) \in R^n$
\[ y_i=\sum_{j=1}^n a_{i,j}x_j+b_i\]

Выясним при каких условиях отображение $A$ является сжимающим.
Эти условия зависит от выбора метрики в n-мерном арифметическом пространстве.
Установить это условие для пространств $R_1^n,R_2^n,R_\infty^n$.
Нарисовать $\epsilon$ окрестность точки для пространств
$R_1^2,R_2^2,R_\infty^2$.

\subsection{Применение принципа сжимающих отображений
к решению интегральных уравнений.}

\begin{definition}[интегральное уравнением первого рода]
  Интегральное уравнение называется
  \textit{интегральным уравнением первого рода},
  если неизвестная функция входит в него только под знаком интеграла.
\end{definition}

\begin{definition}[интегральное уравнением второго рода]
  Интегральное уравнение называется интегральным уравнением второго рода,
  если неизвестная функция входит в него как под знаком интеграла так и вне его.
\end{definition}

Интегральное уравнение Фредгольма второго рода:
\[ f(x)-\lambda\int_a^bK(x,y)f(y)dy=\phi(x)\]

Здесь $f(x)$ - неизвестная функция, $K(x,y)$ - ядро интегрального уравнения,
это является известной функцией, которая предполагает ограничения:
$|K(x,y)|\leq M\sep\forall x,y\in[a,b]$
$\phi(x)$ - заданная функция, непрерывная на отрезке $[a,b]$,
$\lambda$ - произвольный параметр.

\begin{definition}[однородное уравнение]
  Если $\phi(x) = 0$, то уравнение называется однородным,
  в противном случае неоднородным.
\end{definition}

Рассмотрим отображение $Af(x)$, которое действует на функцию $f$:
\[ Af(x)=\phi(x)+\lambda\int_a^bK(x,y)f(y)dy\]

\[ A: C[a, b]\to C[a,b]\sep f,\phi \in C[a,b]\]

Выясним при каком условии отображение $A$ будет отображением сжатия
 $f_1, f_2 \in C[a,b]$

\[ {\rho(Af_1,Af_2)} = {\max_{a \geq x \geq b} | Af_1(x)-Af_2(x) |} =
\max_{a \geq x \geq b} \left| \lambda\int_a^bK(x, y) \cdot \allowbreak
(f_1(y)-f_2(y))dy \right| \leq
|\lambda|M(b-a) \cdot \allowbreak
\max_{a \geq x \geq b} | f_1(x)-f_2(x) | = \allowbreak
\underbrace{|\lambda|M(b-a)}_{<1}\rho(f_1,f_2) \]

Интегральное уравнение Фредгольма имеет единственное решение,
которое может быть получено методом последовательных приближений.

\[ f_{n+1}(x)=\phi(x)+\lambda\int_a^bK(x,y)f_n(y)dy \sep
{ f_0^{(x)} \in C[a,b]}\]

\textbf{Лекция №6}

\subsubsection{Нелинейное интегральное уравнение второго рода.}

\[f(x)-\lambda\int_a^bK(x,y,f(x))dy=\phi(x)\]

Ядро интегрального уравнения $K(x,y,z)$ удовлетворяет условию Липшеца по своему
``функциональному'' аргументу.

\[|K(x,y,z_1)-K(x,y,z_2)| \leq M|z_1-z_2| \sep
{\forall z_1,z_2\sep M>0}\]


$f(x), \phi(x)$ непрерывные функции на отрезке $[a,b]$.
Рассмотрим оператор $A:C[a,b]\to C[a,b]$

\[ Af(x) = {\phi(x)+\lambda\int_a^bK(x,y,f(y))dy} \]

Выясним в каком случае отображение $A$ будет сжимающим.

\[ {\rho(Af_1, Af_2)} = {\max_{a\leq x\leq b}|Af_1(x)-Af_2(x)|}=
\max_{a\leq x\leq b}\left|\lambda\int_a^b(K(x,y,f_1(y))-\allowbreak
K(x,y,f_2(y)))dy\right|\leq
\max_{a\leq x\leq b}(|\lambda|(b-a)M|f_1(x)-f_2(x)|)
= \underbrace{|\lambda|(b-a)M}_{<1}\rho(f_1,f_2) \]
При достаточно малых $\lambda$ это отображение сжимающие.

\subsubsection{Интегральное уравнение Вольтеры второго рода}

Интегральное уравнение Вольтеры второго рода отличается от интегрального
уравнения Фредгольма переменным верхним пределом интегрирования.

\[f(x)-\lambda\int_a^xK(x,y)f(y)dy=\phi(x)\]

Неизвестная функция $f(x)$ предполагается непрерывной на $[a,b]$,
заданная правая часть непрерывна на $[a,b]$,
ядро считается ограниченной функцией в квадрате:
$|K(x,y)|\leq M\sep\forall x,y\in[a,b]\sep A: C[a,b] \to C[a,b]$.

Рассмотрим отображение:

\[Af(x)=\phi(x)+\lambda\int_a^xK(x,y)f(y)dy\]

Покажем что некоторая степень отображения $A$ ($A^n$),
является отображением сжатия:
\[\rho(Af_1,Af_2)=
\max_{a\leq x\leq b}\left|\lambda\int_a^xK(a,b)(f_1(y)-f_2(y))dy\right|
\leq |\lambda|M(x-a)\rho(f_1,f_2)\]

Предположим, что $\rho(A^mf_1,A^mf_2)\leq
|\lambda|^m\f{M^m}{m!}(x-a)\rho(f_1,f_2)$

Используя это предположение, оценим
\[{\rho(A^{m+1}f_1,A^{m+1}f_2)} = \max_{a\leq x\leq b}\left|\lambda
\int_a^xK(x,y)\cdot(A^mf_1(y)-A^mf_2(y))dy\right|
\leq|\lambda|^{m+1}\f{M^{m+1}}{(m+1)!}\cdot(x-a)^{m+1}\rho(f_1,f_2)\]

В силу метода математической индукции справедлива оценка
для любого натурального $n$,
таким образом для произвольного значения $\lambda$ можно указать такое $m$,
что $|\lambda|^m\f{M^m}{m!}(b-a)^m<1$.
Таким образом для отображения $A$ существует при любом $\lambda$
такое натуральное значения $m$, при котором $A^m$ является отображением сжатия,
тогда можно применить обобщение принципа сжимающего отображения,
которое указывает на существование единственной неподвижной точки оператора $A$,
эта неподвижная точка и будет решением интегрального
уравнения Вольтера второго рода.

Для нахождения решения можно применить инерционный процесс.

\textbf{Замечание!} Интегральное уравнение Вольтеры второго рода можно
рассматривать как частный случай интегрального уравнения Фредгольма,
если доопределить ядро следующим образом $K(x,y)=0\sep y>x$.
Но отображение $A$ является сжимающим только при достаточно малых $\lambda$,
а для уравнения Вольтеры мы показали, что $A^m$ является сжимающим отображением
при произвольных $\lambda$.

\subsection{Компактность в метрических пространствах}

Из анализа известно теорема Больцано-Вейерштрасса:
из любой ограниченной числовой последовательности можно извлечь
сходящуюся подпоследовательность.
Обобщение этого факта привело к понятию компактности
в метрических пространствах.

Рассмотрим некоторое множество $M \subset X$.

\begin{definition}[предкомпактное множество]
  Множество называется предкомпактным,
  если из любой последовательности его элементов ${x_n} \subset M$
  можно извлечь сходящуюся подпоследовательность $x_{n_k} \to x \in X$.
\end{definition}

Точка $x$ может принадлежать множеству $M$, а может и не принадлежать ему.

\begin{definition}[компактное множество]
  Множество $M$ называется компактным если пределы всех указных
  подпоследовательностей принадлежат этому множеству.
\end{definition}

Множество является компактным тогда и только тогда когда оно
предкомпактно и замкнуто.
Метрическое пространство $X$ называется компактным если из любой
последовательности его элементов можно извлечь сходящуюся подпоследовательность.

Компактное метрическое пространство всегда является полным.

Числовая ось ($R^1$) не является компактным пространством. %???

%Метрическое пространство ($[a,b]$) является компактным пространством. %???

%??????
Оказывается что в $n$-мерном арифметическом ($R_p^n$) пространстве
предкомпактность эквивалентно его ограниченности,
так как сходимость в этом пространстве является по координатной сходимостью.

\begin{theorem}
  Любое ограниченное множество $M \subset R_p^n$ предкомпактно,
  а ограниченное замкнутое множество компактно.
\end{theorem}

\begin{proof}
  Пусть $M$ ограниченное множество.
  Рассмотрим последовательность его элементов: $x^k=(x_1^k,\dots,x_n^k)\in M$.
  Числовая последовательность $x_1^k$ ограниченная,
  поэтому из нее можно извлечь сходящуюся подпоследовательность $x_1^k$,
  которая сходится к некоторой точке $x_1^0$.
  Рассмотрим $x_2^k$ из нее можно извлечь сходящуюся подпоследовательность
  $x_2^k$, которая сходится к некоторой точке $x_2^0$,
  а из последовательности $x_1^{k_2}$ можно извлечь $x_1^{k_2}\to x_1^0$.
  Продолжая этот процесс получим последовательности координат
  $x_i^{k_n}\to x_i^0$.

  Рассмотренный метод доказательства называется диагональным процессом.
  \begin{dgroup*}
  \[ x_1^{k_1} \to x_1^0\]
  \[\begin{matrix}
  x_1^{k_2} \to x_1^0 & \dots & x_1^{k_n} \to x_1^0 \\
  x_2^{k_2} \to x_2^0 & \dots & x_2^{k_n} \to x_2^0 \\
  \dots\\
  & & x_n^{k_n} \to x_n^0 \\
  \end{matrix}\]
  \[x^{k_n}=(x_1^{k_n},\dots,x_n^{k_n})\sep x^{k_n}\to x^0 \in R_p^n\]
  \end{dgroup*}

  Так как сходимость в $R_n^p$ означает по координатную сходимость.
\end{proof}

\subsubsection{Эквивалентная терминология для описания компактности множества}

Вариант I: предкомпактное, компактное.

Вариант II: компактное в пространстве, компактное в себе.

Вариант III: компактное, бикомпактное.

\textbf{Лекция №7}

В пространстве непрерывной на $C[a,b]$ существуют замкнутые
ограниченные множества.
Таким множеством является, например, единичный шар.
Для доказательства некомпактности единичного шара можно рассмотреть
последовательность $x_n(t)=sin2^{n-1}\pi t$.
Для простоты рассмотрим пространство на отрезке $C[0,1]$.

Докажем что расстояние между любыми двумя элементами этой последовательности
больше либо равно 1, пусть $m>n$:
\[\rho(x_n,x_m) = {\max_{0\leq t \leq 1} |x_n(t)-x_m(t)|} \allowbreak=
\max_{0\leq t \leq 1} |sin 2^{n-1}\pi t-\allowbreak
sin 2^{m-1}\pi t|\geq 1 \sep x_n|_{t=2^{-n}}=1, x_m|_{t=2^{-n}}=0 \]

Эта последовательность принадлежит единичному шару,
но из нее нельзя вывести сходящуюся подпоследовательности.

\subsection{Критерий компактности множеств в метрических пространствах}

Рассмотрим некоторое множество $M\subset X$.

\begin{definition}[$\epsilon$-сеть, эпсилон-сеть]
  Множество $N_{\epsilon} \subset X$ называется $\epsilon$-сетью для
  множества $M$, если $\forall x \in M \sep \exists x_{\epsilon}
  \in N_{\epsilon}: \rho(x,x_{\epsilon})<\epsilon$.
\end{definition}
\begin{definition}[конечная $\epsilon$-сеть, конечная эпсилон-сеть]
  $\epsilon$-сеть называется конечной если она состоит из
  конечного числа элементов.
\end{definition}

\begin{theorem}[Хауздорфа, критерий предкомпактности множества]
  Для того чтобы множество $M \subset X$ было предкомпактным необходимо,
  а в случае полного пространства и достаточно,
  чтобы для этого множества при произвольном положительном $\epsilon$
  существовала конечная $\epsilon$-сеть.
\end{theorem}

\begin{proof}[Доказательство необходимости]
  Пусть множество $M$ является предкомпактным.
  Возьмем произвольный элемент $x_1 \in M\sep \epsilon > 0$.
  Построим $O_{\epsilon}(x_1)$.
  Если $M \subset O_{\epsilon}(x_1)$,
  то множество состоящие из одного элемента $\{x_1\}$ и есть
  конечная $\epsilon$-сеть, в противном случае найдется элемент
  $x_2 \in M\sep x_2 \not\in O_\epsilon(x_1)$,
  если $M\subset O_\epsilon(x_1)\cup O_\epsilon(x_2)$,
  то $\{x_1,x_2\}$ и есть конечная $\epsilon$-сеть.
  Продолжая этот процесс получим такой набор точек $x_1, x_2,\dots,x_n$
  при котором $\rho(x_i,x_j)\geq \epsilon\sep \forall i,j=1,2,\dots,n \sep
  i \neq j$.

  Этот процесс является конечным, в противном случае
  (в случае бесконечного процесса) была бы получена последовательность
  ${x_n}\subset M\sep n=1,2,\dots$, расстояние между любыми двумя точками
  было бы больше $\epsilon$.
  Из такой последовательности невозможно выделить сходящуюся
  подпоследовательность,
  а это противоречит предположению о предкомпактности множества.
\end{proof}

\begin{proof}[Доказательство достаточности]
  Пусть для любого положительного $\epsilon$ существует конечная
  $\epsilon$-сеть множества $M$.
  Рассмотрим последовательность $\epsilon_n \to 0$.
  Для каждого $\epsilon_n$ существует конечная $\epsilon_n$ сеть.

  Рассмотрим произвольную последовательность
   $A=\{a_{1},\dots,a_n,\dots\}\subset M$.
  Возьмем $\epsilon_1$ и построим шары радиусом $\epsilon_1$
  с центром в точке $\epsilon$-сети, число таких шаров конечное.
  Один из этих шаров $B_1$ будет содержать бесконечное число членов
  последовательности выделим из последовательности $A$ подпоследовательность
  $A_1\subset B_1$.

  Берем $\epsilon_2$ и строим шары радиуса $\epsilon_2$ с центрами в точках
  $\epsilon_2$-сети.
  Один из этих шаров будет содержать бесконечное число членов
  последовательностей $A_1$. Обозначим $A_2$ подпоследовательность
  последовательности $A_1$, содержащуюся в шаре $B_2$:
  $A_2\subset A_1\sep A_2\subset B_2$.

  Продолжая этот процесс получим вложенные последовательности
  $A\supset A_1\supset A_2\supset \dots\supset A_n\supset \dots$.
  Причем $A_n$ внутри шара $B_n$ радиуса $\epsilon_n$.
  Построим последовательность различных $\{x_n\}$ следующим образом:
  $x_1 \in A_1, x_2 \in A_1, x_2 \not\in A_1, x_2 \neq x_1$ и рассмотрим точки
  $x_n, x_{x+k} \in B_n$ тогда $\rho(x_n,x_{x+k})\leq 2 \epsilon_n \to 0$.
  Последовательность $\{x_n\}$ является фундаментальной, тогда в силу полноты
  оно сходится в этом пространстве.
  Таким образом $\{x_n\}$ это сходящаяся подпоследовательность
  последовательности $A$.
  Следовательно множество $M$ компактно и теорема доказана.
\end{proof}

\begin{corollary}
  Для того чтобы множество $M$ полного метрического пространства $X$
  было предкомпактным достаточно чтобы для этого множества
  существовала предкомпактная $\epsilon$-сеть.
\end{corollary}

\begin{proof}
  Пусть для произвольного $\epsilon$ существует предкомпактная $\epsilon$-сеть
  множества $M$.
  Тогда по теореме Хауздорфа для множества $N_\epsilon \subset M$
  существует конечная $\epsilon$-сеть $N_0 \subset X$, тогда для
  $\forall x \in M \exists x_\epsilon \in N_\epsilon:
  \rho(x,x_\epsilon) < \epsilon$, а для указанного
  $x_\epsilon \exists x_0\in N_0: \rho(x_\epsilon,x_0)<\epsilon$.
  Согласно неравенству треугольника
  $\rho(x,x_0)\leq \rho(x,x_\epsilon) + \rho(x_\epsilon,x_0) \leq 2\epsilon$.
  Значит $N_0$ конечная $\epsilon$-сеть для множества $M$.
  Тогда по теореме Хауздорфа пространство предкомпактно.
\end{proof}

Следствиями теоремы являются также два важных свойства компактных множеств:

\be
  \item Компактное метрическое пространство сепарабельное.
  \item Предкомпактное множество любого метрического пространства ограничено.
\ee

\begin{proof}[Доказательство свойства 1]
  Пусть метрическое пространство $X$ является компактным,
  тогда для него существует при произвольном $\epsilon$
  конечная $\epsilon$-сеть.
  Рассмотрим $\epsilon_n \to 0$. Для каждого $\epsilon$ существует конечная
  $\epsilon$-сеть $N_n$ метрического пространства $M$.
  $N=\bigcup _{n_1}^{\infty} N_n$ образует счетное всюду плотное множество,
  что и доказывает его сепарабельность.
\end{proof}

\begin{proof}[Доказательство свойства 1]
  Рассмотрим предкомпактное множество $M$
  метрического пространства $X$, согласно теореме Хауздорфа для этого множества
  существует конечная 1-сеть $N_1$ тогда
  $\forall x \in M\quad \exists x_n \in N_1: \rho(x, x_n) \leq 1$.
  Обозначим $d=\max_{n=1,2,\dots,m}\rho(x_i,x)$, где $m$ число элементов сети.
  Тогда $\rho(x, x_i) \leq \rho(x, x_n)+\rho(x_n, x_i) \leq 1+d$.
  Что и доказывает неограниченность множества.
\end{proof}

\subsubsection{Конечное покрытие компактного множества}

Замкнутый отрезок числовой оси является компактным множеством.
Известно из любого покрытия замкнутого отрезка интервалов можно выбрать
конечное подпокрытие (лемма Гейне-Бореля).
Этот результат обобщается на компактном множестве в произвольных
метрических пространств.

Система отрытых множеств $\{G_\alpha\}$ образует покрытие множества $M$,
если любой элемент множества $M$ принадлежит хотя бы одному из $G_\alpha$.

\begin{theorem}
  Для того чтобы замкнутое множество $M$ метрического пространства $X$ было
  компактным необходимо и достаточно чтобы из любого его покрытия можно было
  выделить конечное подпокрытие.
\end{theorem}

\subsection{Критерий компактности множеств в пространстве непрерывных функций}

Непосредственно использование теоремы Хауздорфа в произвольных
метрических пространствах затруднительно, поэтому большую роль играют
компактности множеств в конкретных метрических пространствах.
Теорема Арцела представляет собой критерий предкомпактности множества
в пространстве непрерывных функций $C[a,b]$.

\textbf{Лекция №8}

Рассмотрим множество $M\subset C[a,b]$.

\begin{definition}[равномерно ограниченное множество]
  Множество $M$ называется равномерно ограниченным,
  если существует постоянная $K>0$, такая что для
  $\forall x(t)\in M \quad |x(t)|\leq K \forall t \in [a,b]$.
\end{definition}

\begin{definition}[равностепенно ограниченное множество]
  Множество $M$ называется равностепенно непрерывным, если для
  $\forall \epsilon > 0 \ \exists \delta = \delta(\epsilon)>0:
  \forall x(t)\in M\quad \forall t_1,t_2\in [a,b]$
  удовлетворяет неравенству $|t_1-t_2|<\delta \quad |x(t_1)-x(t_2)|<\epsilon$.
\end{definition}

\begin{theorem}[Арцела]
  Для того ч тобы множество $M\in C[a,b]$ было предкомпактным
  необходимо и достаточно, чтобы оно было равномерно ограниченно
  и равностепенно непрерывно.
\end{theorem}

Пример. Рассмотрим множество функций удовлетворяющую на отрезке $[a,b]$
условию Гельдера.
\[ M=\{x(t)\in C[a,b]: |x(t_1)-x(t_2)|\leq K|t_1-t_2|^\alpha \sep
0 < \alpha \leq 1 \sep \forall t_1,t_2 \in [a;b]\}\]

Рассмотрим $M_0=\{x(t)\in M:x(a)=0\}$.
Легко видеть, что множество $M_0$ равномерно ограничена и равностепенно.
Согласно теореме Арцела множество $M_0$ предкомпактно.

\subsection{Свойства непрерывных функционалов на компактных множествах}

Доказательство Теоремы Вейерштрасса о свойствах непрерывных функций
на отрезке числовой прямой существенно опирается на компактность этого отрезка.
Эти теоремы обобщаются на функциональном определении на компактном
множестве абстрактного метрического пространства.

\begin{theorem}
  Пусть $M\subset X$, а функционал $f(x)$ определен на некотором множестве
  $x \in M$, тогда на этом множестве:
  \be
    \item Функционал является ограниченным
    \item Функционал достигает на этом множестве своих
    точной верхней и точно нижней граней.
  \ee
\end{theorem}

\section{Мера и интеграл Лебега}

\subsection{Мера Лебега множеств в арифметическом пространстве
\texorpdfstring{$R$}{Lg}}

Мера Лебега является обобщением понятий длины отрезка, площади плоской фигуры,
объема трех или $n$-мерного тела.

Мера Лебега множеств в $R^n$ определяется в 3 этапа:

На \textbf{первом этапе}: Меры Лебега определяются для параллелепипедов в $R^n$,
то есть для следующих множеств.

\[\Pi = \{x=(x_1,\dots,x_n)\in R^n: x_i\in [a_i,b_i], i=1,2,\dots,n\sep
a_i<b_i\}\]

Рассмотрим параллелепипеды с произвольным включением граней ($x_i<x_i<b_i$).

Мера Лебега параллелепипеда считается равным его объему:
\[m\Pi = \Pi_{i=1}^n(b_i-a_i)\]
Введенная таким образом мера обладает свойством неотрицательности.

На \textbf{втором этапе} мера Лебега определяется для элементарных множеств.
Элементарным множеством, называется объединение конечного числа конечного
числа параллелепипедов, то есть
\[A=\bigcup _{i=1}^K \Pi_i\sep\Pi_i \cap \Pi_j = \emptyset \sep i\neq j\]
причем эти параллелепипеды попарно не пересекаются.

Мера Лебега элемента множества $m'A=\sum_{i=1}^Km\Pi_i$.

Мера элемента множества обладает свойствами счетной аддитивности
($\sigma$-аддитивности).

Мера объединения счетного числа попарно непересекаемых элементарных множеств
равна сумме мер этих множеств.

Совокупность элементарных множеств в $R^n$ замкнуто относительно операций
пересечения, объединения и разности.

Мера m представляет собой распространение меры $m$ на элементарные множества.

\textbf{Третий этап} определение меры

Рассмотрим всевозможные множества $A$, которые содержатся внутри
произвольного параллелепипеда $A\subset F \subset R^n$

\begin{definition}[величина {$\mu^*A$}]
  Внешние меры Лебега множества $A$ называют величиной $\mu^*A$,
  это точная нижняя грань:
  $\mu^*A=inf\sum_{i=1}^\infty m\Pi_i: A\subset\bigcup _{i=1}^\infty\Pi_i\sep
  \Pi_i\cup\Pi_j=\emptyset\sep i\neq j$.
\end{definition}

Внутренние меры Лебега $\mu_*A=\mu E-\mu^*(E\backslash A)$.
Можно доказать что
$\mu_*A=sup\sum_{i=1}^\infty m\Pi_i: \bigcup _{i=1}^\infty\Pi_i \subset A \sep
\Pi_i\cup\Pi_j=\emptyset\sep i\neq j$

\begin{definition}[множество, измеримое по Лебегу]
  Множество $A$ называют измеримым по Лебегу, если совпадают его внешние и
  внутренние меры Лебега, то есть $\mu^*A=\mu_*A=\mu A$.
\end{definition}

Число $\mu A$ - это мера Лебега множества $A$.
Меры Лебега обладают свойствами пополнения и счетной аддитивности и являются
распространением меры $m'$ элементарных множеств на
произвольные множества из $R^n$.

Совокупность измеримых множеств описываемыми следующими утверждениями:
\be
  \item Объединение и пересечение счетного числа измеримых множеств измеримы.
  Разность двух измеримых множеств измеримо.
  \item Все замкнутые и открытые множества в $R^n$ измеримы.
  \item Множество состоящие из конечного или счетного числа элементов
  измеримо и его мера равно нулю.
  \item Полнота меры: любое подмножество множества нулевой меры измеримо
  и его мера равно нулю.
\ee

Свойства меры Лебега:
\be
  \item Счетная аддитивность

  Если $A=\bigcup _{i=0}^\infty A_i\sep A_i\cap A_j=\emptyset\sep i\neq j$,
  то $\mu A=\sum_{i=1}^\infty \mu A_i$

  \item Непрерывность меры

  Если рассматривается вложенная последовательность множеств
  $A_1 \supset A_0 \supset \dots \supset A_i \supset \dots$,
  то $\mu\bigcap _{i=1}^\infty A_i=\lim_{i \to \infty} \mu A_i $.

  Для возрастающей последовательности
  $A \supset A _ \epsilon \supset \dots \supset A_i \supset \dots \sep
  \mu \bigcup _ {i = 1}  ^ \infty A_i = \lim_{i \to \infty} \mu A_i$.

  \item Инвариантность относительно изометрического отображения

  Если два множества $A$ и $B$ конгруэнтны и одно из них измеримо,
  то измеримо и второе и их меры совпадают.
\ee

\begin{definition}[конгруэнтные множества]
  Два множества называют конгруэнтными, если одно из них является
  образом другого при некотором изометрическом отображении $R^n \to R^n$
\end{definition}

\textbf{Лекция №9}

\subsubsection{О множествах бесконечной меры}

Разобьем все пространство $R^n$ на параллелепипеды
$E_{k_1,k_2,\dots,k_n}=\{(x_1,\dots,x_n) \in R^n:k_i<x_i \leq k_i+1 \} \sep
i=\overline{1,n}, k_i=0, \pm 1, \pm 2, \dots$.

\begin{definition}[множество, измеримое по Лебегу]
  Множество $A$ из $R^n$ называется измеримым по Лебегу,
  если измеримы все множества $A \cap E_{k_1,k_2,\dots,k_n}$
  и мерой этого множества называют суммой ряда.
  \[\mu A: \sum_{k_1} \dots \sum_{R_n} \mu (A \cap E_{k_1,k_2,\dots,k_n})\]
\end{definition}

Если ряд сходится, то мера множества $A$ - конечное число,
если же ряд расходится, то мерой есть $\mu A = + \infty$.
Все свойства меры переносятся и а этот случай. отличается только одно,
мера объединения счетного числа попарно непересекаемых множеств может быть
как конечным числом, так и бесконечностью.

\subsubsection{Пример поясняющий своеобразне меры Лебега.}

Рассмотрим множества $A$ b $B$ рациональных и иррациональных числ
некоторого интервала $(a,b)$.
Оба эти множества различаются по мере Лебега.
$\mu A = 0$, так как это множество счетное, а $\mu B = |b-a|$.
А мера Жордана этих множеств не существует, так как не совпадает
нижняя и верхняя меры Жордана, это связано с тем, что при построении
меры Жордана рассматриваются конечные системы параллелепипедов,
а при построении меры Лебега счетные. Мера Лебга - обобщение меры Жордана.

\subsubsection{Мера Лебега на абстрактных множествах.}

\begin{definition}[мера Лебега на некотором абстрактном множестве]
  Мерой Лебега на некотором абстрактном множестве $E$ называют
  неотрицательную функцию подмножеств этого множества обладающих
  свойством счетной аддитивности.
\end{definition}

Оказывается, что совокупность подмножеств абстрактного множества для которых
определена мера Лебега является $\sigma$-алгебра.

\begin{definition}[$\sigma$-алгебра, сигма-алгебра]
  $\sigma$-алгеброй называют совокупность подмножеств некоторого
  множества замкнутое относительно разности пересечения и объединения
  счетного числа подмножеств, а также содержит множество $E$.
\end{definition}

Обозначим ее $ \sigmalgebra \sep A \subset E \sep A_i \in \sigmalgebra \sep
{A_i \backslash A_j \in \sigmalgebra} \sep
\bigcup _ {i=1}^\infty \in \sigmalgebra \sep
\bigcap _ {i=1}^\infty A_i \in \sigmalgebra \sep {E \in \sigmalgebra}$.
$\sigmalgebra$ - это сигма-кольцо с единицей.

\subsubsection{Термин ``почти всюду''.}

Будем говорить, что некоторое свойство выполняется на множестве $A$ почти всюду,
если мера множества точек $x \in A$ на котором это свойство не выполняется $=0$.
Иными словами свойство выполняется почти всюду если оно выполняется
на множестве во всех точках за исключением множества точек нулевой меры.

Если две функции $f(x)$ b $g(x)$ на неполном множестве $A$ совпадают
почти всюду ($f(x) \eeq[\text{п. в.}] g(x) \sep x \in A$),
то эти функции называют эквивалентными. Для непрерывной функции
называют эквивалентными.
Для непрерывной функции эквивалентность равносильно тождественности.

Для примера рассмотрим функцию Дирихле заданную на интервале [0,1],
\[f(x) = \begin{cases}
  1, & x \text{ - рациональный} \\
  0, & x \text{ - иррациональный}
\end{cases}\]
на $[0,1]$ она эквивалентна тождественному 0.

\[f(x) \eeq[\text{п. в.}] g(x) \sep x \in A \sep
\mu \{ x \in A \cdot f(x) \neq g(x) \} = 0 \sep f(x) \sim g(x)\]

\subsection{Измеримые функции}

Будем рассматривать функции определенные на некотором множестве $E \subset R^n$.

\begin{definition}[измеримая функция]
  Функция $f(x)$ называется измеримой на $E$, если для любого
  действительного числа $c \sep \forall \{x: f(x) < c \}$ измеримо.
\end{definition}

\begin{definition}[множества Лебега функции]
  Множествами Лебега функции $f(x)$ называют множества четырех типов:
  \be
    \item $\{x: f(x) < c\}$
    \item $\{x: f(x) \leq c\}$
    \item $\{x: f(x) > c\}$
    \item $\{x: f(x) \geq c\}$
  \ee
\end{definition}

Можно доказать основываясь на свойстве меры Лебега,
что из измеримости множеств одного типа вытекает
измеримость всех остальных типов.

Например, если первое множество измеримо,
то по свойствам меры Лебега измеримо и второе множество.
\[\left\{x: f(x) \leq c\}= \bigcup_{k=1}^\infty\{x:f(x) < c - \f{1}{k}\right\}\]

Если функция $f(x)$ измерима, то измеримы: само множество $E$ и множество
$\{x:f(x)\leq x\}$.

Свойства измеримых функций:

\be
  \item $f(x)=const \sep x \in E$ - измеримо.

  \item Всякая функция определена на множестве меры 0 измерима.

  \item Функция определенная на измеримом подмножестве $E'$ множества
  $E$ измерима.

  \item Если функция $f(x)$ измерима на каждом $E_R$, то измерима на их
  объединении $\bigcup _ {k=1}^\infty E_k$.

  \item Замкнутость множества измеримых функций относительно арифметических
  операций если $f(x)$ и $g(x)$ измеримы, то измеримы $af(x)+ bg(x)$, где $a,b$
  действительные числа и $f(x)g(x)$, $\f{f(x)}{g(x)}$ если $g(x) \neq 0$.

  \item Замкнутость множества измеримой функции относительно предельного
  перехода если функции $f_k(x)$ измеримы и $f_k(x)$ сходится к $f(x)$
  почти всюду $f_k(x) \xrightarrow{п.в.} f(x)$ на $E$, то $f(x)$ измеримо.

  \item Всякая функция непрерывная на множестве $E$ измерима на нем.

  \item Если две функции эквивалентны на множестве $E$ и одна из них измерима,
  то измерима и вторая.
\ee

\subsubsection{Простые функции}

Функция $f(x)$ называется простой, если она измерима и принимает
не более чем счетное число значений.

\begin{theorem}[о структуре простых функций]
  Функция $f(x)$ принимающая счетное число значений $y_k\sep k=1,2,\dots$
  является простой и тогда и только тогда,
  когда измеримы все множества $\{x: f(x) = y_k \}$.
\end{theorem}

\begin{theorem}[критерий измеримости функций]
  Для того чтобы функция $f(x)$ заданная на $E$ была измерима
  необходимо и достаточно чтобы существовала последовательность
  простых функций $f_k (x) \xrightarrow{\text{п.в.}} f(x), x \in E$.
\end{theorem}

Теорема Егорова Д.Ф. устанавливает связь между сходимостью почти всюду и
равномерной сходимости.
А теорема Лузина Н.Н. устанавливает связь между измеримыми функциями и
непрерывными функциями.

\begin{theorem}[Егорова, $\sim$ 1910]
  Если последовательность
  $\ds f_k(x) \xrightarrow{\text{п.в.}} f(x) \sep x \subset E$,
  то $\forall \delta > 0 \ \exists$ подмножество
  $ E_\delta \subset E : \mu E_\delta > \mu E - \delta $ и на множестве
  $E_\delta \sep f_k(x) \rightrightarrows f(x)$ равномерно, $x \in E_\delta$.
\end{theorem}\

\begin{theorem}[Лузина, $\sim$ 1919]
  Для того чтобы функция $f(x)$ была измеримой на $[a,b]$
  необходимо и достаточно чтобы существовала функция $\phi(x)$ -
  непрерывная на этом отрезке отличающаяся от $f(x)$ на множестве меры 0,
  то есть совпадающая с $f(x)$ почти всюду на этом отрезке.
  \[\mu\{x:f(x) \neq \phi (x)\} = 0\]
\end{theorem}

Эту теорему можно обобщить на функции определяющие на множествах $R^n$.

\textbf{Лекция №10}

\begin{theorem}[Лузина, $\sim$ 1919]
  Для того чтобы функция $f(x)$ была измеримой на $[a,b]$ необходимо и
  достаточно чтобы $\forall \epsilon > 0$ существовала функция $\phi(x)$ -
  непрерывная на этом отрезке отличающаяся от $f(x)$ на множестве
  сколь угодно малой меры, то есть совпадающая с $f(x)$
  почти всюду на этом отрезке.
  \[\mu\{x:f(x) \neq \phi (x)\} < \epsilon\]
\end{theorem}

\subsection{Интеграл Лебега по множеству конечных мер}

При построении интегральной суммы Римана точки области интегрирования
группировались по признаку их близости.
При таком подходе к конструкции интеграла,
интегрируемыми оказываются все непрерывные функции,
а также функции имеющие небольшое число точек разрыва.

При построении интеграла Лебега,
точки области интегрирования группируются по признаку близости
значений функции в этих точках.
Такая конструкция интеграла позволяет расширить класс интегрируемых функций.
Будем рассматривать функции определенные на множестве конечной меры
($\mu E < +\infty \sep E \subset R^2$),
все рассматриваемые функции предполагаются измеримыми,
все рассматриваемые множества также предполагаются измеримыми.

\subsubsection{Интеграл Лебега от простой функции.}

Рассмотрим простую функцию $f(x)$, которая принимает
счетное число значений
$y_1, y_2, \dots, y_n, \dots \sep y_i \neq y_j, i \neq j$.

\begin{definition}[интеграл Лебега от простой функции]
  Интегралом Лебега от простой функции $f(x)$ по множеству $A$
  называется число $\ds\int_A f(x) d\mu = \sum_{k=1}^\infty y_k\mu A_k$,
  где $A_k=\{x\in A:f(x)=y_k\}$.
\end{definition}

\[f(x) = \begin{cases}
  1, & x\text{ - рациональный} \\
  0, & x\text{ - иррациональный}
\end{cases}\]

\subsubsection{Интеграл Лебега от измеримой функции.}

Рассмотрим $f(x)$ измеримую на множестве $A$.

\begin{definition}[интегрируемая по Лебегу функция]
  Функция $f(x)$ называется интегрируемой по Лебегу (суммируемой),
  если существует последовательность простых интегрируемых по Лебегу функций
  сходящийся к $f(x)$ равномерно на множестве $A$.
\end{definition}

\begin{definition}[интеграл Лебега от измеримой функции]
  Интегралом Лебега от $f(x)$ называется число:
  \[\int_A f(x) d\mu = \lim_{k \to \infty} \int_A f_k(x) d\mu\].
\end{definition}

Доказывается, что введенное определение является корректным,
то есть предел в определении интеграла существует и не зависит
от выбора последовательности $f_k(x)$.

Доказано, что приведенная конструкция эквивалентна конструкции
при помощи интегральным сум.

\subsubsection{Определение интеграла Лебега при помощи интегральных сум}

Рассмотрим $f(x)$ измеримую на множестве $A$ и принимающую на этом
множестве значения из интервала $[a,b]$.
Разобьем интервал $[a,b]$ на частичные промежутки при помощи точек
$a=y_0 < y_1 < \dots < y_m = b$.
Рассмотрим множество $A_i = \{x \in A: y_{i-1} \leq f(x) < y_i\}$.
Все множества $A_i$ измеримы как пересечение
двух множеств Лебега функции $f(x)$.

\begin{tikzpicture}[>=stealth]
\begin{axis}[
  restrict y to domain=-1:5,
  samples=100, % you don't need 1000, it only slows things down
  ticks=none,
  xmin = -1, xmax = 5,
  ymin = -1, ymax = 6,
  unbounded coords=jump,
  axis x line=middle,
  axis y line=middle,
  xlabel={$x$},
  ylabel={$y$},
  declare function={f(\x)=(\x^3)/16+1;},
  ]
  \addplot[thick, blue, domain=1:4] {f(x)};

  \draw[dashed] (axis cs:0,{f(4)}) node[left] {$b$} -|
                (axis cs:4,0);

  \draw[dashed] (axis cs:0,0.5) node[left] {$a$} -- (axis cs:{f(1)},0.5);
  \draw[dashed] (axis cs:{f(1)},{f(1)}) -- (axis cs:{f(1)},0);

  \draw[dashed] (axis cs:0,{f(4)}) node[left] {$b$} -|
                (axis cs:4,0);

  \draw[dashed] (axis cs:0,{f(3.5)}) node[left] {$y_{m-1}$} -|
                (axis cs:3.5,0);

  \draw[dashed] (axis cs:0,{f(2.75)}) node[left] {$y_i$} -|
                (axis cs:2.75,0);

  \draw[dashed] (axis cs:0,{f(2.25)}) node[left] {$y_{i-1}$} -|
                (axis cs:2.25,0);

  \draw[dashed] (axis cs:0,{f(1.5)}) node[left] {$y_1$} -|
                (axis cs:1.5,0);

  \draw[] (axis cs:{(1+1.5)/2},0) node[below] {$A_1$};

  \draw[] (axis cs:{(2.25+2.75)/2},0) node[below] {$A_i$};

  \draw[] (axis cs:{(3.5+4)/2},0) node[below] {$A_m$};

\end{axis}
\end{tikzpicture}

\[A_i \cap A_j = \emptyset, i \neq j, A=\bigcup_{i=1}^m A_i\]

На каждом промежутке $[y_{i-1}, y_i]$ выбираем произвольную точку $\xi_i$
и строим сумму $\ds\sigma = \sum _{i=1}^m \xi_i \mu A_i$.

\begin{definition}[интеграл Лебега при помощи интегральных сум]
  Интегралом Лебега называется предел интегральной суммы
  при стремлении диаметра разбиении $\lambda$ к нулю.
  \[\int_A f(x) d\mu=\lim_{\alpha \to 0} \sigma \sep
  \alpha=\max_{i=1,\dots,m}(y_i-y_{i-1})\]
\end{definition}

По аналогии с суммами Дарбу для интеграла Римана вводятся
нижние и верхние суммы Лебега.
\[\underline{\sigma} = \sum _{i=1}^m y_{i-1}\mu A_i \sep
\overline{\sigma} = \sum _{i=1}^m y_{i}\mu A_i\]

Если пределы нижней и верхней сум Лебега при $\alpha \to 0$ совпадают,
то функция интегрируемая по Лебегу и этот предел является значением интеграла.
\[\int_A f(x) d\mu=\lim_{\alpha \to 0} \underline{\sigma}
= \lim_{\alpha \to 0} \overline{\sigma}\]

\subsubsection{Выводы}
\be
  \item Всякая ограниченная измеримая функция интегрируемая
  по Лебегу на множестве конечной меры.

  \item Изменения значения функции на множестве меры $0$
  не влияет на значение интеграла Лебега.
\ee

\subsection{Свойства интеграла Лебега}

\be
  \item Линейное свойство
  \begin{dgroup*}
    \[\int_A (f(x)+g(x))d\mu=\int_A f(x)d\mu \int_A g(x)d\mu \]
    \[\int_A \lambda f(x) d\mu = \lambda \int_A f(x) d \mu\]
  \end{dgroup*}
  Причем из существования интеграла в правых частях вытекает существование
  интегралов в левой части.

  \item Монотонность.
  Если $f(x), g(x)$ интегрируемая на множестве $A$ и $f(x) \leq g(x), x \in A$.
  \[\int_A f(x) d\mu \leq \int_A g(x) d\mu\]
  В частности если  $m\leq f(x) \leq M, x \in A$.
  \[ m\mu A\leq \int_A f(x) d \mu \leq M\mu A \]

  \item Свойство о равенстве интегралов.
  Если $f(x) \sim g(x)$ на множестве $A$, то интегралы равны.
  \[\int_A f(x)d\mu = \int_A g(x) d\mu\]

  \item Если $|f(x)|\leq \phi(x)$ и функция $\phi(x)$ интегрируемая,
  то и $f(x)$ интегрируемая.

  \item Интегралы $\ds\int_A f(x) d\mu, \int_A |f(x)| d\mu$ существуют
  или не существуют одновременно.
  Для сравнения если $f(x)$ интегрируемая, то и $|f(x)|$ интегрируемая по $у$,
  а обратное вообще говоря неверное.
  Рассмотрим \[f(x) = \begin{cases}
    1, & x \text{ - рациональный} \\
    -1, & x\text{ - иррациональный}
  \end{cases}\]

  \item Если $\ds\int_A |f(x)| d\mu = 0$, то $f(x)\sim 0$ на множестве $A$,
  то есть $f(x) = 0, x \in A$.

  \item Счетная аддитивность ($\sigma$ аддитивность).
  Если функция $f(x)$ интегрируемая на множестве $A$ и множество $A$
  представлено как объединение счетного числа попарно непересекающихся
  измеримых множеств
  $\ds A = \bigcup_{k=1}^\infty A_k, A_i \cap A_j=\emptyset \sep i \neg j$,
  то $\ds\int_A f(x)d\mu = \sum_{k=1}^\infty \int_{A_k} f(x) d \mu$,
  причем ряд в правой части сходится абсолютно.
  В частности функция $f(x)$ интегрируемая на любом подмножестве
  нулевой меры множестве $A$.

  Справедливо следующие обращение этого свойства:
  функция $f(x)$ интегрируема на множестве $A$, если сходится ряд
  \[\int_A f(x) d \mu =\sum_{k=1}^\infty \int_{A_k} f(x) d \mu\]
\ee

\textbf{Лекция №11}

\subsection{Интеграл Лебега по множеству бесконечной меры}

Пусть некоторое множество $X$ не имеет конечной меры такой что
$\mu X = + \infty$, однако может быть представлена в виде объединения
счетного числа множеств конечной меры, образующих возрастающую
последовательность
$\ds X = \bigcup_{k=1}^\infty X_k \sep X_k \subset X_{k+1} \sep
\mu X_k <+ \infty$

Интеграл Лебега по множеству $X$ бесконечной меры называют число:
\[\int f(x) d\mu = \lim_{k \to \infty} \int_{X_k}f(x) d \mu\]

Если этот предел существует и не зависит от выбора последовательности множества
$X_n$ для интеграла по множеству бесконечной меры остается справедливы
все свойства предыдущего пункта с той разницей, что интеграл по объединению
попарно непересекаемых множеств конечной меры может быть
бесконечным (свойство 7).

\subsection{Сравнение интервалов Лебега и Римана}

Сформируем соответствующую теорему для случая интеграла Лебега
по конечному отрезку числовой прямой.

\begin{theorem}
  Если функция $f(x)$ интегрируема по Риману на отрезке $[a,b]$ и
  $\ds\int_a^b f(x) dx = I$, то $f(x)$ интегрируема по Лебегу на этом отрезке
  $\ds\int_{[a,b]} f(x) d \mu = I$.
\end{theorem}

В смысле Римана на числовой прямой интеграл от неограниченной функции
или интеграл по неограниченному промежутку могут существовать
только несобственный.

Если отрезок неограниченный:
\[\int_a^\infty f(x)dx = \lim_{A \to \infty} \int_a^A f(x) dx\]

Если функция $f(x)$ неограниченная в точке $a$:
\[\int_a^b f(X)dx = \lim_{\epsilon \to +0} \int_{a+\epsilon}^b f(x) dx\]

Несобственные интегралы называют абсолютно сходящимися,
если существует аналогичный интеграл от $|f(x)|$.
В противном случае эти интегралы называют условию сходящейся.
Оказывается, если несобственный интеграл в смысле Римана сходится абсолютно,
то интеграл Лебега от функции $f(x)$ существует и равен несобственному
интегралу в смысле Римана от этой функции.

Если же несобственный интеграл по Риману сходится условно,
то интеграл по Лебегу не существует.
Это связано с тем, что в смысле Лебега $f(x)$ и $|f(x)|$ интегрируемые
или не интегрируемые одновременно.

Теория меры Лебега позволяет дать ответ на вопрос в каком случае
ограниченная функция является интегрируемой по Риману.

\begin{theorem}[критерий интегрирования ограниченной функции по Риману]
  Для того, чтобы ограниченная функция $f(x)$ определенная на некотором
  параллелепипеде $n$-мерного арифметического пространства $R^n$ была
  интегрируемая по Риману необходимо и достаточно,
  чтобы мера Лебега множества точки разрыва этой функции равна нулю.
\end{theorem}

\subsection{Пространство \texorpdfstring{$L_p[a,b]$}{Lg} интегрируемых
по Лебегу функций и их полнота.}

Рассмотрим множество функции $L_p[a,b]$, элементами которого являются
функции $f(x)$ интегрируемый по Лебегу в $p$-ой степени по Лебегу на $[a,b]$.
\[\int_a^b |f(x)|^p dx, p \geq 1\]

В дальнейшем используем для обозначения интеграла Лебега обозначим
интеграл Римана.

Определим расстояние между двумя функциями:
\[\rho(f,g)= \left( \int_a^b |f(x) - g(x)|^p dx \right) ^{\f{1}{p}}\]

Для выполнения аксиом метрического пространства необходимо считать
эквивалентные функции на $[a,b]$ одним $\epsilon$-пространством.

Если $s>p\leq 1$, то $L_s[a,b] \subset L_p[a,b]$.

\begin{definition}[сходимость в среднем по {$L_1[a,b]$}]
  Сходимость по метрике пространства $L_1[a,b]$ по отрезку $[a,b]$
  называется сходимостью в среднем.
\end{definition}
\begin{definition}[сходимость в среднем квадрата по {$L_2[a,b]$}]
  Сходимость по метрике пространства $L_2[a,b]$ называют
  сходимостью в среднем квадрата.
\end{definition}

\begin{theorem}
  Пространство $L_p[a,b] \sep p \leq 1$ является конечным сепарабельным
  метрическим пространством.
\end{theorem}

Множество непрерывных функций всюду плотным в $L_p[a,b]$.
Согласно теореме Вейерштрасса множество многочленов с рациональным
коэффициентом всюду плотно в множестве непрерывных функций,
а следовательно всюду плотно в $L_p[a,b]$.
Множество элементов с рациональными коэффициентами является
счетным всюду плотным.

Замечание о сходимости в $L_p[a,b]$:
\be
  \item Из сходимости в $L_p[a,b]$ вытекает сходимость в $L_1[a,b]$.

  \item Из последовательности сходящихся в $L_p[a,b]$ можно извлечь
  подпоследовательность сходящуюся почти всюду.

  \item Из сходимости почти всюду не вытекает сходимость в $L_1[a,b]$.

  \textbf{Например}
  \[f_n(x)=\begin{cases}
    n, & 0 \leq x \leq \f{1}{n} \\
    0, & \f{1}{n} \leq x \leq 1
  \end{cases}\]
  \begin{tikzpicture}[>=stealth]
  \begin{axis}[
    restrict y to domain=-1:5,
    samples=100, % you don't need 1000, it only slows things down
    ticks=none,
    xmin = -1, xmax = 5,
    ymin = -1, ymax = 5,
    unbounded coords=jump,
    axis x line=middle,
    axis y line=middle,
    xlabel={},
    ylabel={},
    ]
    \draw[blue,line width=2pt,] (axis cs:0,3) -- (axis cs:2,3);

    \draw[blue,line width=2pt,] (axis cs:2,0) node[below] {$\f{1}{n}$} --
                       (axis cs:4,0) node[below] {$1$};

    \draw[dashed] (axis cs:2,3) -- (axis cs:2,0);
  \end{axis}
  \end{tikzpicture}

  $f_n(x) \to 0 \sep x \neq 0$, то есть на отрезке
  $[0,1] \sep f_n(x) \xrightarrow[]{\text{п. в.}} 0$,
  однако $\ds\int_0^1 f_n(x) dx = 1$, значит в $L[0,1]$ эта
  последовательность расходится.

  \item Из равномерной сходимости вытекает сходимость в $L_p[a,b]$.
\ee

\section{Нормированные пространства}

\subsection{Понятия линейного пространства. Линейная зависимость.}

\begin{definition}[линейное пространство]
  Некоторое множество $L$ называют линейным пространством если для
  любого его элемента определены операции сложения и умножения на число.
\end{definition}

Различают действительные и комплексные пространства, в зависимости от того,
какой набор чисел используется.

\subsubsection{Аксиомы линейных пространств}

Операция сложения определяется для любых элементов $x,y \in L$
удовлетворяющих четырем аксиомам.

\be
  \item Аксиома коммутативности: $x+y = y+x$.

  \item Аксиома ассоциативности: $(x+y)+z = x+(y+z)$.

  \item Аксиома существования нулевого элемента:
  $\exists 0 \in L: x+0=x \sep \forall x \in L$.

  \item Аксиома существования противоположного элемента:
  $\forall x \in L \sep \exists -x \in L: x+(-x)=0$.
\ee

Операция умножения на число удовлетворяющих четырем аксиомам.
\be
  \item Ассоциативность:$(\alpha \beta) x = \alpha (\beta x) \sep
  \forall \alpha, \beta \sep \forall x$.

  \item Аксиома умножения на 1: $1\cdot x = x \sep \forall x \in L$.

  \item Дистрибутивность: $\alpha (x+y)=\alpha x + \alpha y \sep
  (\alpha + \beta) x = \alpha x + \beta x$
\ee

\begin{definition}[линейно независимая система элементов]
  Систему элементов пространства $L_1 \sep x_1, \dots, x_n$ называют
  линейно независимой, если $\ds\sum_{k=1}^n \alpha_k x_k = 0$
  при всех $\alpha_k = 0$.
\end{definition}
\begin{definition}[линейно зависимая система элементов]
  В противном случае система называется зависимой если один элемент
  является линейной комбинацией остальных элементов.
\end{definition}

\begin{definition}[бесконечная система элементов]
  Бесконечной системой элементов линейного пространства называют
  линейно независимой, если линейно независимая любая ее конечная подсистема.
\end{definition}

Говорят, что линейное пространство имеет размерность $n$, $dim\ L=n$,
если в нем существует система из $n$ линейно независимых элементов,
а любая система из $(n+1)$ элементов является линейно зависимой.

Говорят, что линейное пространство является бесконечномерным,
если в нем существует линейно независимая система из любого числа элементов.

\begin{definition}[линейное многообразие]
  Линейное многообразие: множество $L'\subset L$ называется
  линейным многообразием если оно замкнуто относительно множества
  операций введенных на $L$, то есть
  \be
    \item $x+y \in L' \sep \forall x,y \in L'$
    \item $\alpha x \in L' \sep \forall x \in L' \sep \forall \alpha$ - число.
  \ee
\end{definition}

Линейное многообразие само является линейным пространством по отношению к
операциям введенным на $L$.

\textbf{Лекция №12}

\subsection{Нормированные пространства.
Сходимость в нормированных пространствах.
Банаховы пространства. Ряды в нормированных пространствах.}

\begin{definition}[нормированное пространство и норма]
  Линейное пространство $L$ называется нормированным пространством если
  $\forall x \in L$ определена числовая функция (функционал, норма)
  удовлетворяющее следующим аксиомам (обозначается $ \Vert x \Vert $):
  \be
    \item Аксиома неотицательности:
    \[{\Vert x \Vert > 0} \sep {\forall x \neq 0} \sep {\Vert x \Vert = 0}
    \Leftrightarrow {x = 0} \]

    \item Аксиома положительной однородности:
    \[ {\Vert \alpha x \Vert} = {|\alpha| \cdot  \Vert x \Vert} \sep
    {\forall x \in L} \sep {\forall \alpha} \]

    \item Аксиома треугольника:
    \[ {\Vert x+y \Vert} \leq {\Vert x \Vert + \Vert y \Vert} \sep
    {\forall x,y \in L} \]
  \ee
\end{definition}

Из аксиом нормы вытекает неравенство:
$| \ \Vert x \Vert  -  \Vert y \Vert \ | \leq \Vert x \pm y \Vert $.

Действительно
\begin{dgroup*}
  \[ {\Vert x \Vert} = {\Vert x + y - y \Vert} \allowbreak
  \leq {\Vert x + y \Vert + \Vert - y \Vert} \allowbreak
  = {\Vert x + y \Vert + \Vert y \Vert} \]
  \[ {\Vert y \Vert} \leq {\Vert x + y \Vert + |x|} \]
\end{dgroup*}

Любое нормированное пространство является метрическим пространством если ввести
на нем метрику согласованную с нормой $ \rho(x,y) = \Vert x - y \Vert$.

\[\ds\begin{array}{|*3{>{\ds}c|}} \hline
  \text{МП} & \text{Норма } \Vert x \Vert & \text{Метрика } \rho(x,y) \\\hline
  R_p^n
    & \left(\sum_{k=1}^n |x_k|^p\right)^{\f{1}{p}}
    & \left(\sum_{k=1}^n|x_k-y_k|^p\right)^{\f{1}{p}} \\
  & & x=(x_1,\dots,x_n) \\
  R_\infty^n
    & \max_{k=1,\dots,n}|x_k|
    & \max_{k=1,\dots,n}|x_k-y_k| \\
  l_p
    & \left(\sum_{n=1}^\infty |x_n|^p\right)^{\f{1}{p}}
    & \left(\sum_{n=1}^\infty|x_n-y_n|^p\right)^{\f{1}{p}} \\
  l_\infty
    & \sup_{n=1,2,\dots} |x_n|
    & \sup_{n=1,2,\dots} |x_n-y_n| \\
  C[a,b]
    & \max_{a \leq t \leq b}|x(t)|
    & \max_{a \leq t \leq b}|x(t)-y(t)| \\
  L_\infty[a,b]
    & \sup_{a \leq t \leq b}|x(t)|
    & \sup_{a \leq t \leq b}|x(t)-y(t)| \\
  L_p[a,b]
    & \left(\int_a^b |x(t)|^p dt\right)^{\f{1}{p}}
    & \left(\int_a^b |x(t)-y(t)|^p dt\right)^{\f{1}{p}} \\\hline
\end{array}\]

Все факты относительно сходимости в метрических пространствах распространяются
на нормированные пространства.

Пусть $x$, $y$ - элементы метрического пространства $L$,
а $\alpha_n$ - последовательность чисел на числовой оси.

\subsubsection{Непрерывность алгебраических операций}

Если
$ \ds x_n \xrightarrow[L]{} x \sep y_n \xrightarrow[L]{} y \sep
\alpha_n \xrightarrow[R^1]{} \alpha $
выполняется, то
$ \ds x_n+y_n \xrightarrow[L]{} x+y \sep \alpha_n \cdot x_n
\xrightarrow[L]{} \alpha x $.
Действительно
\begin{dgroup*}
  \[ {\Vert x_n+y_n-(x+y) \Vert} \leq {\Vert x_n-x \Vert  +  \Vert y_n-y \Vert}
  \xrightarrow[R^1]{} 0\]
  \[ {\Vert \alpha_n x_n - \alpha x \Vert} \allowbreak =
  {\Vert \alpha_n (x_n - x) + (\alpha_n - \alpha) x \Vert} \allowbreak \leq
  {|\alpha_n| \cdot \Vert x_n-x\Vert
  \allowbreak+|\alpha_n-\alpha| \cdot \Vert x \Vert}
  \allowbreak \xrightarrow[R^1]{} 0 \].
\end{dgroup*}

\subsubsection{Непрерывность нормы}

Если $x_n \xrightarrow[L]{} x$, то $ \Vert x_n \Vert
\xrightarrow[R^1]{} \Vert x \Vert $.
Действительно ${|\ \Vert x_n \Vert - \Vert x \Vert \ |} \allowbreak
\leq {\Vert x_n - x \Vert} \to 0$.

\subsubsection{Ряды в нормированных пространствах}

Говорят что ряд $\ds\sum_{n=1}^\infty x_n = x$ сходится и сумма равняется $x$,
если последовательность частичных сумм ряда сходится к элементу $x$.
\[{\lim_{n \to \infty} \sum_{k=1}^n x_k = x} \sep
{\left \Vert \sum_{k=1}^n x_k - k \right \Vert \to 0} \sep
{\sum_{k=1}^n x_k \to x}\]

\subsubsection{Банаховы пространства}

\begin{definition}[банахово пространство]
Полное нормированное пространство называется банаховым.
\end{definition}

В теме \ref{sec:mp} была сформирована теорема \ref{th:о пополнении}
о пополнении метрического пространства,
эта теорема обобщается на случай нормированных пространств.

\begin{theorem}[о пополнении нормированного пространства]
\label{th:о пополнении нормированного пространства}
  Всякое нормированное пространство $L$ может быть включено в банахово
  пространство $L^+$ в качестве всюду плотного линейного многообразия
  $L \subset L^+ \sep [L]=L^+$.
\end{theorem}

\begin{definition}[замкнутое линейное многообразие]
  Подпространством нормированного пространства называется
  замкнутое линейное многообразие.
\end{definition}

\subsubsection{Абсолютная сходимость рядов в банаховых пространствах.}

Рассмотрим ряд составленных из элементов банахового пространства
$\ds\sum_{n=1}^\infty x_n$.
Предположим что сходится ряд составленных из нормированных элементов
$\ds\sum_{n=1}^\infty \Vert x_n \Vert < \infty$.
В этом случае говорят что ряд составленных из элементов пространства
сходится абсолютно.

Докажем что из сходимости ряда $\ds\sum_{n=1}^\infty \Vert x_n \Vert < \infty$
вытекает сходимость ряда $\ds\sum_{n=1}^\infty x_n$ и имеет место обобщение
неравенства треугольника.
\[\left \Vert \sum_{n=1}^\infty x_n \right \Vert \leq
\sum_{n=1}^\infty \Vert x_n \Vert\]

Докажем что последовательность частичных сумм $\ds\sum_{n=1}^\infty x_n$
является фундаментальной.
Для определенности $m>n$.
\[{\left \Vert \sum_{k=1}^m x_k - \sum_{k=1}^n x_k \right \Vert} \allowbreak
= {\left \Vert \sum_{k=n+1}^m x_k \right \Vert} \allowbreak
\leq {\sum_{k=n+1}^m \Vert x_k \Vert}\]

Так как этот ряд $\ds\sum_{n=1}^\infty \Vert x_n \Vert < \infty$ сходится,
то последовательность его частичных сумм фундаментальная, то есть:
\[ {\forall \epsilon > 0} \sep \exists N:
{\left| \sum_{k=1}^m \Vert x_k \Vert - \sum_{k=1}^n \Vert x_k \Vert \right|}
\allowbreak = {\sum_{k=n+1}^m \Vert x_k \Vert < \epsilon}
\sep {\forall n,m \geq N} \]

Следовательно последовательность частичных сумм $\sum_{n=1}^\infty x_n$
фундаментальная и в силу полноты пространства она сходится,
то есть ряд действительно сходится.

Для доказательства обобщения неравенства треугольника достаточно перейти к
пределу при $n \to \infty$ в неравенстве:
\[ \left \Vert \sum_{k=1}^n x_k \right \Vert \leq \sum_{k=1} \Vert x_k \Vert \]
и воспользоваться непрерывностью нормы.

\subsection{Евклидовы пространства.
Характеристическое свойство евклидовых пространств.}

\begin{definition}[евклидово действительное нормированное пространство]
  Действительное нормированное пространство $f$ называется евклидовым
  если для любых двух элементов $x$, $y$ этого пространства определено
  действительное число $(x,y)$ которое называется скалярным произведением
  этих элементов и скалярное произведение удовлетворяет следующим аксиомам.

  \be
    \item Аксиома симметрии:
    \[(x,y)=(y,x) \sep {\forall x,y \in L}\]

    \item Аксиома аддитивности:
    \[ (x + y, z) = (x,z) + (y,z) \sep {\forall x,y,z \in L} \]

    \item Аксиома однородности:
    \[\alpha (x,y) = (\alpha x,y) \sep {\forall x,y \in L} \sep
    {\forall \alpha \text{ - действительный}} \]

    \item Аксиома неотицательности
    \[ (x,x) > 0 \sep\forall x\neq 0 \sep {(x,x)=0 \Leftrightarrow x=0} \]
  \ee
\end{definition}

В евклидовых пространствах норма предполагается согласованной
со скалярным произведением $ \Vert x \Vert =\sqrt{(x,x)}$.
При таком определении нормы выполняется все аксиомы нормы.

\textbf{Лекция №13}

\subsubsection{Неравенство Коши-Буняковского}
\[ (x,y)^2 \leq (x,x) \cdot (y,y) \text{ или }
|(x,y)| \leq \Vert x \Vert \cdot \Vert y \Vert\]

Для доказательства неравенства Коши-Буняковского рассмотрим
скалярное произведение
\[(x + \alpha y, x+ \alpha y) = (x,x) + 2 \alpha (x,y) + \alpha^2 (y,y) \leq 0
\sep {\forall \alpha}\].
Таким образом $(x,y)^2 - (x,x)(y,y) \leq 0$,
отсюда и вытекает неравенство Коши-Буняковского.

При помощи неравенство Коши-Буняковского доказывается неравенство
треугольника для нормы согласованной со скалярным произведением.
\begin{dgroup*}
  \[\sqrt{|(x+y,x+y)|} \allowbreak = \sqrt{(x,x) + 2(x,y) + (y,y)} \allowbreak
  \leq \sqrt{ (\sqrt{(x,x)})^2 + 2 \sqrt{{x,x}}\sqrt{{y,y}} + (\sqrt{(y,y)})^2 }
  \allowbreak = \sqrt{(x,x)} + \sqrt{(y,y}\]
  \[\Vert x+y \Vert \leq \Vert x \Vert + \Vert y \Vert\]
\end{dgroup*}

\begin{definition}[унитарное комплексное линейное пространство]
  Комплексное линейное пространство со скалярным произведением
  называется унитарным.
\end{definition}

В унитарным пространстве по сравнению с евклидовым модифицируется только
аксиома симметрии для скалярного произведения $(x,y)=\overline{(y,x)}$.
Тогда в унитарных пространствах
\begin{dgroup*}
  \[\alpha (x,y) = (\alpha x, y) = \overline{(y,\alpha x)}\]
  \[\overline{\alpha} \overline{(x,y)}= (y,\alpha x)\]
\end{dgroup*}

Возникает вопрос, какое место занимает евклидово пространство среди
всех нормированных пространств.
Ответ на этот вопрос дает теорема.

\begin{theorem}[характеристическое свойство евклидовых пространств]
Для того чтобы действительное нормированное пространство $L$ было евклидово
необходимо и достаточно чтобы для любых двух элементов этого пространства
выполнялось равенство параллелограмма:
\[\Vert x + y \Vert^2 +\Vert x - y \Vert^2 \allowbreak
= 2 (\Vert x \Vert^2 + \Vert  y \Vert^2) \sep {\forall x,y \in L} \]
\end{theorem}

Необходимость сразу вытекает из аксиомы скалярного произведения.
Для доказательства достаточности необходимо определить
скалярное произведение равенством
\[(x,y)=\f{1}{4}(\Vert x+y \Vert^2 - \Vert x-y \Vert^2 )\]
и доказать справедливость аксиом скалярного произведения.

\[\ds\begin{array}{|*3{>{\ds}c|}} \hline
\begin{tabular}{@{}c@{}} Евклидово \\ пространство\end{tabular}
    & \begin{tabular}{@{}c@{}} Скалярное \\ произведение \\$(x,y) $\end{tabular}
    & \begin{tabular}{@{}c@{}} Норма \\ $\Vert x \Vert$ \end{tabular} \\\hline
  R_2^n
    & \sum_{k=1}^n x_k y_k
    & \sqrt{\sum_{k=1}^n x_k^2} \\
  l_2
    & \sum_{n=1}^\infty x_k y_k
    & \sqrt{\sum_{n=1}^\infty x_k^2} \\
  L_2[a,b]
    & \int_a^b x(t) y(t) dt
    & \sqrt{\int_a^b x^2(t) dt} \\\hline
\end{array}\]

При $p \neq 2$ в пространствах $R_p^n \sep l_p  \sep L_p [a,b]$
невозможно ввести норму со скалярным произведением,
так как не выполняется равенство параллелограмма.

\subsubsection{Непрерывность скалярного произведения.}

Если $x_n \to x \sep y_n \to y$, то $(x_n, y_n) \to (x,y)$. Действительно
\[|(x_n,y_n) - (x,y) \allowbreak
= |(x_n- x,y_n) + (x, y_n - y)| \allowbreak
\leq |(x_n - x, y_n)| + |(x, y_n - y)| \allowbreak
\leq \Vert x_n - x \Vert \cdot \Vert y_n \Vert \allowbreak
  + \Vert x \Vert \cdot \Vert y_n - y \Vert \to 0 \].

\subsection{Ортогональные системы элементов в евклидовом пространстве.
Теорема о ортогонализации.}

\subsubsection{Полная система элементов нормированного пространства}

\begin{definition}[полная система элементов]
  Система элементов нормированного пространства $L$ называется
  полной если замыкание множества всех ее линейных комбинаций
  (кратко говорят линейное замыкание) совпадает со всем пространством.
\end{definition}

Иными словами, любой элемент нормированного пространства может быть представлен
в виде ряда по полной системе.
\[ x=\sum_{n=1}^\infty \alpha_n x_n \allowbreak
= \lim_{n \to \infty} \sum_{k=1}^n \alpha_k x_k\]

\begin{definition}[базис нормированного пространства]
  Полная линейно независимая система элементов нормированного пространства
  называется базисом.
\end{definition}

\subsubsection{Ортогональные системы элементов. Евклидово пространство}

\begin{definition}[ортогональная система элементов евклидово пространства]
  Система ненулевых элементов евклидово пространства $\{x_\alpha\}$
  называется ортогональной,
  если $(x_\alpha, x_\beta) = 0 \sep \alpha \neq \beta$.
\end{definition}
\begin{definition}[ортонормированная система элементов евклидово пространства]
  Если кроме того норма равняется единице $\Vert x_\alpha \Vert = 1$,
  то система называется ортонормированной.
\end{definition}

Если $\{x_\alpha\}$ ортогональна,
то $\left\{\f{x_\alpha}{\Vert x_\alpha \Vert}\right\}$ ортонормированная.

Ортогональные системы элементов является линейно независимыми.
Для доказательства линейной независимости рассмотрим произвольную
конечную подсистему $x_k \sep k=\overline{1,n}$ и предположим что
\[\sum_{k=1}^n \lambda_k x_k = 0\]
где $\lambda_k$ - любое число, тогда
\begin{dgroup*}
  \[\left(\sum_{k=1}^n \lambda_k x_k, x_i\right) \allowbreak
  = \sum_{k=1}^n \lambda_k (x_k, x_i) = \lambda_i \cdot \Vert x_i \Vert ^ 2 \]
  \[ \lambda_i \cdot \Vert x_i \Vert ^ 2 = 0 \allowbreak
  \Leftrightarrow {\lambda_i = 0} \sep
  {i = 1,2,\dots,n}\]
\end{dgroup*}

\begin{definition}[ортогональный базис]
  Полное ортогональная система элементов евклидово пространства
  называется ортогональным базисом.
\end{definition}

\begin{definition}[ортонормированный базис]
  Полное ортонормированная система элементов евклидово пространства
  называется ортонормированным базисом.
\end{definition}

В пространстве $R_2^n$ ортонормированный базис образует векторы:
\begin{align*}
  e_1&=(1,0,\dots,0) \\
  &\dots \\
  e_k&=(\underbrace{0,0,\dots,1}_k,0,\dots,0) \\
  &\dots \\
  e_n&=(1,0,\dots,0,1)
\end{align*}

Любой элемент $x=(x_1,\dots,x_n)$ может быть представлен в виде
\[\sum_{k=1}^n x_k e_k \sep {\forall x \in R_2^n}\]

В пространстве $l_2$ ортонормированный базис образует векторы:
\begin{align*}
  e_1&=(1,0,\dots,0,\dots) \\
  &\dots \\
  e_n&=(\underbrace{0,0,\dots,1}_n,0,\dots,0,\dots) \\
  &\dots \\
\end{align*}

Выберем любой элемент
\[x=(x_1,\dots,x_n,\dots) \sep \sum_{n=1}^\infty x_n^2 < \infty\]

Для доказательства полноты нужно доказать что ряд
$x = \sum_{n=1}^\infty$ сходится.
\[\left\Vert x - \sum_{k=1}^n x_k e_k \right\Vert \allowbreak
= \left\Vert \sum_{k=n+1}^\infty x_k e_k \right\Vert \allowbreak
= \Vert (0,\dots,0,x_{n+1}, x_{n+2},\dots) \Vert \allowbreak
= \sum_{k=n+1}^\infty x_k^2 \to 0 \sep {n \to \infty} \]

В пространстве $L_2[a,b]$ ортогональную систему образует
тригонометрическая система элементов
$\left\{ \f{1}{2}, cos \f{2 \pi n t}{b-a} ,
sin \f{2 \pi nt}{b-a} \right\}_{n=1,2,\dots} $,
ортогональность можно проверить непосредственно.

Полнота тригонометрической системы в $L_2[a,b]$ вытекает из теоремы Вейерштрасса
о равномерном приближении непрерывных функций тригонометрическими многочленами и
плотности множества непрерывных функций в $L_2[a,b]$.

Из рассмотренных примеров видно что ортогональный базис в этих пространствах
состоит их конечного или счетного числа элементов. Оказывается что этот факт
является общим для сепарабельных евклидовых пространств.

\begin{lemma}
  Всякая ортонормированная система сепарабельного евклидова пространства
  не более чем счетно.
\end{lemma}

\textbf{Лекция №14}

\begin{proof}
  Путь $\{x_\alpha\}$ ортонормированная система в
  сепарабельном евклидовом пространстве.
  \[(x_\alpha,x_\beta)=0 \sep \alpha = \beta \sep
  (x_\alpha,x_\beta) \allowbreak = \Vert x_\alpha \Vert ^2\]
  Найдем расстояния между двумя элементами ортонормированной системы:
  \[\Vert x_\alpha - x_\beta \Vert ^ 2 \allowbreak
  = (x_\alpha - x_\beta, x_\alpha - x_\beta) \allowbreak
  = (x_\alpha, x_\alpha) - 2(x_\alpha, x_\beta) + (x_\beta,x_\beta) \allowbreak
  = 2 \sep {\alpha \neq \beta}\]

  Построим шары $B_\alpha$ с центрами в элементах $x_\alpha$ и радиусом
  $\frac{1}{2}$.
  Эти шары попарно не пересекаются.
  Так как пространство сепарабельное,
  то в нем существует счетное всюду плотное множество $\{a_n\}$.
  Каждый из шаров $B_\alpha$ содержит покрайней мере один элемент их
  этого множества.
  Следовательно множество этих шаром не более чем счетно.
  Лемма доказана.
\end{proof}

\begin{theorem}[о ортогонализации]
  Пусть $\{f_n\}$ линейно независимая система элементов
  евклидова пространства $L$,
  тогда существует ортонормированная система $\{\phi_n\}$
  обладающая следующими свойствами:
  \begin{dgroup*}
    \[\phi _n = a_{n1}f_1+\dots + a_{nn}, a_{nn} \neq 0\]
    \[f_n = b_{n1}\phi_1 + \dots+b_{nn}\phi_n, b_{nn} \neq 0\]
  \end{dgroup*}
\end{theorem}

\begin{theorem}
  Ортонормированная система $\{\phi_n\}$ определена с точностью
  до множителя $\pm 1$.
\end{theorem}

\begin{proof}
  Доказательство конструктивное, так как оно дает алгоритм построения
  ортонормированной системе по заданной линейно независимой системы.
  Доказательство проведем методом математической индукции.

  Определим $\phi = \f{f_1}{\Vert f_1 \Vert}$.
  Предположим что построим $n-1$ элементом ортонормированной системы
  $\phi_1, \dots, \phi_{n-1}$. Определим $\phi^*$
  \[\phi^* = f_n - \sum_{k=1}^n (f_n,\phi_k) \phi_k \sep
  {(\phi^* , \phi_i) = 0} \sep {i=1,2,\dots, n-1} \sep
  (\phi_k, \phi_i) = \delta_{ki} \allowbreak
  = \begin{cases} 1, & k=i \\ 0, & k\neq i \end{cases}\]

  Следовательно $\phi_n^*$ ортогонален всем элементам $\phi_i$.
  $\phi_n = \f{\phi_n^*}{\Vert \phi_n^* \Vert}$
  станет элементом ортонормированной системы $\phi_n^* \neq 0$
  (в силу линейной независимости системы $\{f_n\}$).
  Тогда согласно методу математической индукции теорема доказана.
\end{proof}

\begin{corollary}
В сепарабельном евклидовом пространстве существует ортонормированный базис.
\end{corollary}

\begin{proof}
В сепарабельном евклидовом пространстве существует счетное всюду
плотное множество $\{\psi_n\}$.
Удалим из этого множества элементы $\psi_k$ которые являются
линейными комбинациями $\psi_n \sep i<k$. В результате получим
линейно независимую систему.
Эта система является полной.
Применяя к этой системе процесс ортогонализации получаем
ортонормированный базис.
\end{proof}

\subsubsection{Многочлен Лежандра}

Система многочленов $\{1,t,t^2,\dots, t^n,\dots\}$ является полной
в $L_2[a,b]$,
так как множество непрерывных функций всюду плотно в $L_2[a,b]$ и имеет
место теорема Вейерштрасса о равномерном приближении непрерывной
функции к многочленам.

Применяя к этой системе многочленов в пространстве $L_2[-1,1]$ процесс
ортогонализации, описанный в теореме, получаем систему многочленов Лежандра,
которая обладает всеми свойствами которые описаны в теореме.

\[\int_{-1}^1 P_n(t)P_k(t)dt = 0 \sep n \neq k\]

\subsection{Ряды Фурье по ортонормированной системе.
Замкнутые ортонормированные системы и их полнота.}

Рассмотрим произвольный элемент $f \in L$, пусть $\{\phi_n\} \subset L$
ортонормированная система элементов.
Пространство считается сепарабельным.

\begin{definition}[коэффициенты Фурье]
  Скалярное произведение $(f,\phi_n)$ называется коэффициентами Фурье
  элемента $f$ по ортонормированной системе $\{\phi_n\}$.
\end{definition}

\begin{definition}[ряд Фурье]
  Ряд $\sum_{n=1}^\infty (f_n, \phi_n) \phi_n$ называется рядом Фурье
  по ортонормированной системе $\{\phi_n\}$.
  Частным случаем является тригонометрический ряд.
\end{definition}

Исследование сходимости ряда Фурье начнем с вычислений отклонения элемента
$f$ от произвольной линейной комбинации первых $n$ элементов
ортонормированной системы по норме пространства $L$.
\[ \left \Vert f- \sum_{k=1}^n \alpha_k \phi_k \right \Vert ^2
= (f - \sum_{k=1}^n \alpha_k \phi_k, f - \sum_{i=1}^n \alpha_i \phi_i)
= (f,f) - 2 \sum_{k=1}^n \alpha_k (f_n,\phi_k)
  + \sum_{i=1}^n \sum_{k=1}^n \alpha_i \alpha_k (\phi_i, \phi_k)
= \sum_{k=1}^n \underbrace{(\alpha_k^2 - 2 \alpha_k (f,\phi_k) +(f,\phi_k)^2)}_
    {(\alpha_k-(f,\phi_k))^2}
  + \Vert f \Vert ^2 - \sum_{k=1}^n (f,\phi_k)^2\]
Таким образом
\[\left \Vert f- \sum_{k=1}^n \alpha_k \phi_k \right \Vert ^2
= \Vert f \Vert^2 - \sum_{k=1}^n (f, \phi_k)^2
  + \sum_{k=1}^n (\alpha_k - (f,\phi_k))^2\]

\subsubsection{Минимальное свойство частичных сумм ряда Фурье.}

Среди всевозможных линейных комбинаций первых $n$ элементов
ортонормированной системы минимальное отклонение от заданного
элемента $f$ по норме пространства имеет $n$-ая частичная сумма
ряда Фурье этого элемента.
\[ \left\Vert f - \sum_{k=1}^n (\alpha_k \phi_k) \phi_k \right \Vert
\leq \left \Vert f - \sum_{k=1}^n \alpha_k \phi_k \right \Vert \sep
{\forall \{\alpha_k\}}\]

Отклонение $n$-ой частичной суммы ряда Фурье от элемента $f$ задается
тождеством Бесселя.

\[\left \Vert f - \sum_{k=1}^n (f,\phi_k) \phi_k \right \Vert ^ 2
= \Vert f \Vert ^ 2 - \sum_{k=1}^n(f,\phi_k)^2 \]

Из тождества Бесселя следует ограниченность частичными
суммами ряда с положительными членами $\sum_{n=1}^\infty (f,\phi_k)^2$.
\[\sum_{k=1}^n (f,\phi_k)^2 \leq \Vert f \Vert ^2 \]
То есть частичная сумма ограниченна.

Итак этот числовой ряд сходится и имеет место неравенство Бесселя
\[\sum_{n=1}^\infty (f,\phi_k)^2 \leq \Vert f \Vert ^2\]

\begin{definition}[замкнутая ортонормированная система]
  Ортонормированная система элементов $\{\phi_n\}$ называется замкнутой если
  $\forall f \in L$ ряд Фурье по этой ортонормированной системе сходится
  к указанному элементу по норме пространства.
  \[\left \Vert f - \sum_{k=1}^n (f,\phi_k) \phi_k \right \Vert
  \xrightarrow[n \to \infty]{} 0 \sep \forall f\in L\]
\end{definition}

Ряд Фурье сходится и ортонормированная система является замкнутой.

Для замкнутой ортонормированной системы превращается в равенство Пар\-се\-валя.

\[\sum_{n=1}^\infty (f, \phi_n)^n = \Vert f \Vert ^2\]

\begin{theorem}[о полноте замкнутой системе элементов]
  Ортонормированная система элементов сепарабельного евклидово пространства
  является замкнутой тогда и только тогда когда она является полной.
\end{theorem}

\begin{proof}
Пусть ортонормированная система $\{\phi_n\}$ является замкнутой,
тогда любой элемент $f \in L$ стремится к нулю.
\[f = \lim_{n \to \infty} \sum_{k=1}^n (f, \phi_k) \phi_k
= \sum_{n=1}^\infty (f,\phi_n) \phi_n\]

Если система замкнутая, то она является полной,
что и указывает на полноту системы $\{\phi_n\}$.

Путь ортонормированная система $\{\phi_k\}$ является полной,
выберем произвольный элемент $f \in L$ в силу полноты системы
существует $\alpha_n$, такие что
\[f = \sum_{n=1}^\infty \alpha_n \phi_n \sep
\left \Vert f - \sum_{k=1}^n \alpha_k \phi_k \right \Vert \to 0 \]

Воспользуется минимальной свойством частичных сумм ряда Фурье.
\[ \left\Vert f - \sum_{k=1}^n (f,\phi_k) \phi_k \right\Vert
\leq \left\Vert f - \sum_{k=1}^n \alpha_k \phi_k \right\Vert \to 0 \]
То
\[\left\Vert f - \sum_{k=1}^n (f,\phi_k) \phi_k \right\Vert \to 0 \sep
{n \to \infty}\]
Что и указывает на замкнутость системы $\{\phi_k\}$
Теорема доказана.
\end{proof}

\textbf{Лекция №15}

\textbf{Замечание.} Равенство Парсеваля в бесконечно мерной пространстве
является аналогом теоремы Пифагора.
\begin{dgroup*}
\[(\overrightarrow{f} \cdot \overrightarrow{i})^2
  + (\overrightarrow{f} \cdot \overrightarrow{j})^2
=|\overrightarrow{f}|^2\]
\[\sum_{k=1}^\infty (f,\phi_k)^2 = \Vert f \Vert^2\]
\end{dgroup*}

\subsection[Полные сепарабельные евклидовы пространства]
{Полные сепарабельные евклидовы пространства. Теорема Фишера-Рисса.}

В предыдущем пункте рассматривалось сепарабельное евклидово пространство,
которое не обязательно является полным.
Было установлено, что сходимость ряда $\sum_{n=1}^\infty c_n^2$
является необходимым условием для того
чтобы числа $c_n$ представляли коэффициенты Фурье некоторого элемента $f$ по
ортонормированной системе $\{\phi_n\}$ (не обязательно полное).

Оказывается что в полных сепарабельных евклидовых пространствах указанные
условия не только необходимым, но и достаточным.

\begin{theorem}[Фишера-Рисса]
  Пусть $L$ полное сепарабельное евклидово пространство,
  $\{\phi_n\}$ - ортонормированная система элементов,
  а числа $c_n$ таковы что сходится ряд $\sum_{n=1}^\infty c_n^2 < \infty$,
  тогда существует элемент $f \in L$ этого пространства,
  такой что числа $c_n$ является коэффициентами Фурье этого элемента по
  ортонормированной системе $\{\phi_n\}\sep \exists f \in L: c_n=(f, \phi_n)$,
  а сумма ряда
  $\sum_{n=1}^\infty c_n^2 = \Vert f \Vert ^ 2 \sep
  (c_1, \dots, c_n, \dots) \in l_2$.
\end{theorem}

\begin{proof}
  Рассмотрим элементы $f_n$ которые являются $f_n = \sum_{k=1}^nc_k \phi_k$.
  Легко видеть, что $(f_k,\phi_k)= c_k \sep k=1,2,\dots,n$,
  а $\Vert f_n \Vert ^ 2 = \sum_{k=1}^n c_k^2$.
  Докажем что последовательность $f_n$ является фундаментальной
  в пространстве $L$.
  Для этого для произвольного натурального $p_n$:
  \[\Vert f_{n+p}- f_n\Vert^2 = \sum_{k=n+1}^{n+p} \Vert c_k \phi_k \Vert^2
  = \left(\sum_{k=n+1}^{n+p}c_k \phi_k,\sum_{k=n+1}^{n+p} c_k \phi_k \right)
  = \sum_{k=n+1}^{n+p} c_k^2\]

  Так как ряд сходится, то последовательность его частичных сумм является
  фундаментальной, тогда
  $ \forall \epsilon > 0 \sep \exists N : \sum_{k=n+1}^{n+p} c_k^2 < \epsilon
  \sep {n \geq N}$, тогда
  \[\Vert f_{n+p} - f_n \Vert ^2 < \epsilon \sep {n \geq N}\]
  $\{f_n\}$ - фундаментальная в силу полноты сходимости.
  То есть $f_n \to f \in L \sep f = \lim_{n \to \infty} f_n$
  В равенстве $(f_k,\phi_k)= c_k \sep k=1,2,\dots,n$, зафиксируем $k$,
  перейдем к пределу при $n \to \infty$ и воспользоваться непрерывностью
  скалярного произведения и получим $(f,\phi_k)=c_k$.
  Переходя к пределу в $\Vert f_n \Vert^2 = \sum_{k=1}^n c_k^2$ при
  $n \to \infty$ и учитывая непрерывность нормы получаем
  $\sum_{n=1}^\infty c_n^2 = \Vert f \Vert ^2$
  И теорема фишера-Рисса доказана.
\end{proof}

\begin{theorem}
  Для того чтобы ортонормированная система элементов $\{\phi_n\}$
  полного сепарабельного евклидово пространства была полной необходимо и
  достаточно чтобы не существовал элемент пространства, кроме нулевого,
  ортогональной ко всем элементам ортонормированной системы.
\end{theorem}

\begin{proof}
  Пусть ортонормированная система $\{\phi_n\}$ является полной и элементы $f$
  ортогональны ко всем элементам ортонормированной системы
  $(f,\phi_n) = 0 \sep n=1,2,\dots$.
  Тогда согласно неравенству Парсеваля
  \[\Vert f \Vert ^2 = \sum_{n=1}^\infty (f,\phi_n)^2 = 0 \sep f = 0\]
  Пусть теперь ортонормированная система $\{\phi_n\}$ не является полной,
  тогда найдется покрайней мере один элемент пространства $g \in L$
  для которого справедливо строгое неравенство
  \[\sum_{n=1}^\infty  c_n^2 < \Vert g \Vert^2 \sep {c_n=(g,\phi_n)}\]
  Согласно теореме Фишера-Рисса найдется элемент $f \in L$,
  такой что эти же числа $c_n$ будут его коэффициентами Фурье
  $x_n=(f,\phi_n) \sep \sum_{n=1}^\infty c_n^2 = \Vert f \Vert^2$.
  Отсюда видно, что норма
  $(f-g, \phi_n) = 0 \sep f -g \neq 0 \sep \Vert f \Vert \neq \Vert g \Vert$.
  Теорема доказана.
\end{proof}

В частности из этой теоремы вытекает что два различных элемента пространства
не могут иметь одинаковые ряды Фурье по полной ортонормированной системе.

\subsection{Гильбертово пространство. Теорема о изоморфизме}
\label{subsec:gp_toi}

\begin{definition}[гильбертово пространство]
  Гильбертовом пространством называется полное евклидово пространство
  бесконечной размерности.
\end{definition}

Обычно рассматривают сепарабельные гильбертовы пространства:
\[l_2 \sep L_2[a,b]\]

Пространство $R_2^n$ полное сепарабельное евклидово пространство конечной
последовательности.

\subsubsection{Понятие изоморфизма}

\begin{definition}[изоморфные нормированные пространства]
  Два нормированных пространства $X$ и $Y$ называют изоморфными,
  если между их элементами можно установить взаимно однозначные соответствия
  согласованные с линейными операциями введенными на этих пространствах
  и с нормой, то есть.

  \[X\ni x \leftrightarrow y \in Y \sep
  {X'\ni x' \leftrightarrow y' \in Y'} \sep
  {x + x' \leftrightarrow y + y''} \sep
  {\alpha x \leftrightarrow \alpha y} \sep
  {\Vert x \Vert \leftrightarrow_X \Vert y \Vert_Y}\]
\end{definition}

Говорят что два евклидовых пространства $X$, $Y$ изоморфны если между их
элементами можно установить взаимооднозначные соответствия согласованные
с линейными операциями и со скалярным произведением
\[{x+x' \leftrightarrow y + y'} \sep {\alpha x \leftrightarrow \alpha y}
\sep {(x,x')=(y,y')}\]

Любое конечномерное евклидово изоморфно арифметическому пространству $R_2^n$.

\begin{theorem}[об изоморфизме]
\label{th:об изоморфизме}
  Любые два сепарабельных гильбертовых пространства изоморфны между собой,
  а каждая из них изоморфна в пространстве последовательностей $l_2$.
\end{theorem}

\begin{proof}
  Рассмотрим сепарабельное гильбертово пространство $H$ выберем в этом
  пространстве ортонормированную систему элементов $\{\phi_n\} \subset H$.
  Каждому элементу $f \in H$ поставим в соответствие последовательности его
  коэффициентов Фурье по ортонормированной системе $\{\phi_n\}$.

  Согласно теореме Фишера-Рисса это соответствие будет взаимооднозначное.
  Для того чтобы доказать что $H$ изоморфно $l_2$ остается установить
  согласованность операций и скалярное произведение
  $H \ni g \leftrightarrow b = (b_1, \dots, b_n, \dots) \in l_2 \sep
  b_n= (g,\phi_n)$.
  Отсюда видно, что соответсвие согласованно с линейными операциями
  $f+g \leftrightarrow c+b \sep \alpha f \leftrightarrow \alpha c$

  Остается доказать, что оно согласовано со скалярным произведением.
  Для этого заметим что
  \[\Vert f+g \Vert ^ 2 = (f+g, f+g)
  = \underbrace{(f,f)}_{\Vert f \Vert^2} + 2(f,g)
    + \underbrace{(g,g)}_{\Vert f \Vert^2} \sep (f,g)
  = \f{1}{2}(\Vert f+g \Vert ^2 - \Vert f \Vert^2 - \Vert g \Vert ^2)\]
  Так как $\{\phi_n\}$ полная ортонормированная система, имеет место
  равенство Парсеваля:
  \[\Vert f \Vert^2 = \sum_{n=1}^\infty c_n^2 \sep
  \Vert g \Vert^2 = \sum_{n=1}^\infty b_n^2 \sep
  \Vert f+g \Vert^2 = \sum_{n=1}^\infty(c_n+b_n)^2\]
  Тогда $(f,g) = \sum_{n=1}^\infty c_n b_n = (c,b)$
\end{proof}

Теорема об изоморфизме позволяет утверждать, что фактически существует только
одно гильбертово пространство. Пространство последовательностей представляет
собой ``координатную'' реализацию гильбертова пространства,
а любое пространство $L_2[a,b]$ представляет собой ``функциональную''
реализацию гильбертова пространства.

\subsection{Подпространство гильбертово пространства. Ортогональное дополнение.
Ортогональное проектирование.}

Известно что подпространство нормированного пространства называется
замкнутое линейное многообразие.
Подпространство гильбертова пространства либо конечномерно,
либо само является гильбертовым пространством.
Поэтому в любом подпространстве гильбертово пространства
существует ортонормированный базис.

\textbf{Лекция №16}

Рассмотрим некоторое подпространство $M \subset H$.

\begin{definition}[ортогональное дополнение]
  Ортогональным дополнением называется множество
  $M^\bot = \{f \in H: (f,g) = 0 \sep \forall g \in M\}$.
  Это множество элементов гильбертово пространства ортогонально всем
  элементам подпространства $M$.
\end{definition}

В силу определения подпространства $M^\bot$ само является подпространством
гильбертово пространства $H$.
Очевидно, что ортогональное дополнение к ортогональному дополнению и будет $M$,
поэтому $M$ и $M^\bot$ образуют два взаимодополняющих пространства
гильбертово пространства $H$.

Если $\{\phi_n\}$ ортонормированный базис подпространства $M$, а $\{\phi_n'\}$
ортонормированный базис подпространства $M^\bot$, то в совокупности они образуют
базис в $H$.
Иными словами ортонормированный базис подпространства может быть
расширен до всего пространства.

\begin{theorem}[об ортогональном проектировании]
\label{th:об ортогональном проектировании}
  Пусть $M \subset H$, тогда любой элемент $f$ из $H$ может быть
  единственным образом представлен в виде
  $f = g+h \sep g \in M \sep h \in M^\bot$,
  при этом норма элемента $\Vert f \Vert^2 = \Vert g \Vert^2 + \Vert h \Vert^2$.
  Элемент $g$ называется проекцией элемента $f$ на подпространство $M$.
  Элемент $h$ - перпендикуляр проведенный из $f$ к подпространству $M$.
  $\Vert h \Vert$ - расстояние от элемента $f$ до подпространства $M$.
\end{theorem}

\begin{proof}
  Пусть $\{\phi_n\}$ ортонормированный базис подпространства $M$.
  Возьмем произвольный элемент $f$ из $H$.
  Обозначим $c_n = (f,\phi_n)$ коэффициентами Фурье элемента $f$
  ортонормированной системы $\{\phi_n\}$.

  Ряд $\sum_{n=1}^\infty c_n^2 < \infty$ сходится из тождества Бесселя.
  Тогда по теореме Фишера-Рисса
  \[\exists g \in M: c_n=(g,\phi_n) \sep g = \sum_{n=1}^\infty c_n \phi_n\]
  Рассмотрим элемент $h=f-g$  и докажем что $h$ принадлежит
  ортогональному дополнению.
  Возьмем произвольный элемент $x \in M$, который может быть представлен
  $x=\sum_{n=1}^\infty (x,\phi_n)\phi_n$.
  Легко видеть что $(h,\phi_n) = (f,\phi_n) - (g,\phi_n) = 0$.
  Тогда $(x,h) = 0 \sep \forall x \in M$, следовательно $h$ принадлежит
  ортогональному дополнению.

  Доказательство единственности разложения проведем от противного,
  то есть предположим
  \begin{dgroup*}
    \[f = g+h \sep g \in M \sep h \in  M^\bot \]
    \[ f = g' + h' \sep g' \in M \sep h' \in  M^\bot\]
  \end{dgroup*}
  отсюда вытекает, что $(g, \phi_n) = (g', \phi_n)$.
  Так как $(h, \phi_n) = 0 \sep (h', \phi_n) = 0$, тогда $(g-g',\phi_n)=0$.
  Откуда согласно теореме \ref{th:об изоморфизме} пункта \ref{subsec:gp_toi}
  следует что $g= g'$, что и доказывает единственность.

  \[\Vert f\Vert ^ 2 = (f,f) = (g+h,g+h) = (g,g) + 2(g,h) + (h,h) =
  \Vert g \Vert ^2 + \Vert h \Vert ^2\]

  Теорема доказана.
\end{proof}

\begin{example}
  Пример. Рассмотрим одномерное подпространство $M \subset H$.
  \[M = \{x \in H: x = \alpha e\}\]
  $e$ - некоторый элемент из $H$.

  Любой элемент $f$ мы можем представить в виде
  $f=\alpha e + h \sep h \in M^\bot \sep (h,e) = 0$,
  тогда $(f,e) = \alpha \Vert e \Vert^2$.
  Отсюда находится неизвестный коэффициент
  $ \alpha = \f{(f,e)}{\Vert e \Vert^2}$.
\end{example}

\subsection{Расстояние от элемента нормированного пространства
до подпространства.
Лемма Рисса о почти перпендикуляре.}

Рассмотрим некоторое нормированное пространство $L$, не обязательно евклидово.
Пусть $M\subset L$. Возьмем произвольный элемент $f\in L$,
расстояние от элемента $f$ до подпространства $M$ называется число
$\rho(f,M) = \inf _{x\in M} \Vert f-x \Vert$,
а точная нижняя грань берется по $x \in M$.
Очевидно что $\rho(f,M)> 0$, если $f \not\in M$ и $\rho(f,M) = 0$,
если $f \in M$.
В гильбертовом пространстве справедлива
теорема~\ref{th:об ортогональном проектировании}
об ортогональном проектировании.
\[f=g+h \sep g \in M \sep h \in M^\bot\]
Тогда для любого элемента $x \in M \sep \Vert f- x \Vert ^2 = (g+h-x,g+h-x)$,
где $g-x \in M$, тогда $(g-x,h) = 0$ и
\begin{dgroup*}
  \[\Vert f- x \Vert ^2 = (g-x,g-x) + (h,h)
    = \Vert g-x \Vert^2 + \Vert h \Vert^2\]
  \[\Vert f-x \Vert \geq \Vert g-x \Vert \sep {\forall x \in M}\]
  \[\Vert f-g\Vert = \Vert h \Vert\]
\end{dgroup*}
Что и доказывает что $\rho(f,M) = \Vert h \Vert$.

В произвольном нормированном пространстве отсутствует понятие
скалярного произведения, поэтому там невозможно ввести понятие перпендикуляр,
однако в произвольных нормированных пространствах справедлива лемма о
почти перпендикуляре.

\begin{lemma}[о почти перпендикуляре]
  Пусть $M$ подпространство нормированного пространства $L$ не совпадающие со
  всем пространством, тогда
  \[{\forall \epsilon \in (0,1)} \sep
  {\exists h_\epsilon: \Vert h_ \epsilon\Vert = 1}\]
  и обладающая свойством $\rho(h_\epsilon,M) > 1 - \epsilon$.
\end{lemma}

\begin{proof}
  Возьмем $f \in L \sep f \not\in M$, такой элемент существует так как
  $M \neq L$
  Обозначим $d = \inf _{x\in M} \Vert f - x \Vert$.
  Согласно определению точной нижней грани
  $\forall \epsilon \in (0,1) \sep \exists x_\epsilon \in M : d \leq
  \Vert f- x_\epsilon \Vert < \f{d}{1-\epsilon}$.
  Элемент $h_\epsilon = \f{f-x_\epsilon}{\Vert f-x_\epsilon\Vert}$.
  Возьмем $x\in M$
  \[
    {\Vert h_\epsilon-x \Vert} =\allowbreak
    {\left\Vert \f{f-x_\epsilon}{f-x_\epsilon}- x\right\Vert} =\allowbreak
    {\left\Vert f-\underbrace{(x_\epsilon-x\Vert f-x_\epsilon \Vert)}_{\in M}
    \cdot \allowbreak \f{1}{\Vert f-x_\epsilon \Vert} \right\Vert}
    >\allowbreak \f{d}{\Vert f-x_\epsilon \Vert} > 1- \epsilon
  \]
  Теорема доказана.
\end{proof}

\textbf{Лекция №17}

\section{Линейные операторы в нормированных пространствах}

\subsection{Определение линейного оператора. Непрерывность и ограниченность.
Теорема Хана-Банаха.}

Пусть оператор $A$ действует из линейного пространства $X$ в
линейное пространство $Y$ ($A: X \to Y$), то есть при помощи этого оператора
каждому элементу некоторого линейного пространства $X$ на котором определен
этот оператор ставится в соответствие единственный элемент пространства $Y$.

\begin{definition}[функционал линейного пространства]
  Если пространство $Y$ это числовая прямая или пространственная плоскость,
  то оператор называется функционалом.
\end{definition}

\begin{definition}[линейный оператор]
  Оператор называется линейным если он определен на некотором
  линейном многообразии $X$ и обладает свойством аддитивности и однозначности.
\end{definition}

\[A(\alpha_1x_1 + \alpha_2x_2) = \alpha_1Ax_1 + \alpha_2Ax_2\]
Для любых $x_1$, $x_2$ в области определения оператора и для любых чисел
$\alpha_1$, $\alpha_2$.

Из определения линейного оператора сразу вытекает что множество его значений
представляет собой линейное многообразие.
\[A \cdot \underbrace{0}_{\text{Число}} =
A (\underbrace{0}_{\text{Число}} \cdot x) =
\underbrace{0}_{\text{Число}} \cdot Ax = \underbrace{0}_{\text{Элем.}}\]

Пусть теперь $X$, $Y$ - нормированное пространство, будем считать, что оператор
$A$ определен на всем пространстве $X$, оператор $A$ называется непрерывным
в точке $x_0$ если
$ \forall \{x_n\} \subset X: x_n \underset{x}{\to} x_0 $,
то $Ax_n \to Ax_0$ или  $\Vert Ax_n - A x_0 \Vert \underset{R^1}{\to}0$

\begin{lemma}
  Если оператор $A: X \to Y$ определен на всем пространстве,
  $X$ непрерывен в одной точке пространства,
  то он непрерывен во всех точках пространства.
\end{lemma}

\begin{proof}
  Пусть оператора непрерывен в точке $x_0 \in X$, рассмотрим произвольную точку
  $X \in X$ и произвольную последовательность
  $x_n \to x \sep \Vert x_n - X \Vert \to 0$, тогда $x_n - x + x_0 \to x_0$.
  Так как оператор непрерывен в точке $x_0$,
  то $A(x_n-x+x_0) \to Ax_0 \sep Ax_n \to Ax$,
  что и доказывает непрерывность оператора в точке $x$.
\end{proof}

\subsubsection{Ограниченность оператора}

Если линейный оператор $A$ определен на всем пространстве $X$,
то множество его значений не может быть ограниченным.
Действительно, если оператор
$\Vert A (\alpha x) \Vert = |\alpha| \cdot \Vert Ax \Vert$,
то норма $\Vert A (\alpha x) \Vert$ может быть бесконечно большой.

Линейный оператор называют ограниченным если любое ограниченное множество
$M \subset X$ он переводит в ограниченное множество $A (X) \subset Y$
\[\Vert x \Vert \leq c \forall x \in M \sep \quad
\Vert y \Vert \leq K \sep \forall y \in A (M)\]
Эквивалентное определение ограниченного оператора

\be
  \item Оператор переводит ограниченное множество в ограниченное множество
  $A(M) \subset Y$

  \item $\exists c > 0: \Vert Ax \Vert \leq c \Vert x \Vert \sep
  \forall x \in X$.

  \item Оператор является ограниченным на евклидовом шаре в пространстве $X$,
  $\Vert x \Vert \leq 1$
\ee

Для линейных операторов понятия непрерывности и ограниченности являются
эквивалентными.

\begin{definition}[нулевой оператор]
  Оператор $A$ называют нулевым если $Ax=0 \sep \forall x \in X$.
\end{definition}

Все наши рассуждения относятся к ненулевым операторам.

\begin{theorem}
  Пусть линейный оператор $A$ определен на всем пространстве $X$.
  Оператор является непрерывным тогда и только тогда, когда он является
  ограниченным.
\end{theorem}

\begin{proof}
  Пусть оператор $A$ непрерывный, предположим, что он не является ограниченным,
  тогда он не ограничен на единичном шаре, то есть
  $\forall n = 1,2,\dots\sep\exists x_n: \Vert x_n \Vert < 1 \sep
  \Vert A x_n \Vert > n$.
  Отсюда $\left\Vert A \f{x_n}n \right\Vert > 1$, с другой стороны
  $\left\Vert \f{x_n}n \right\Vert = \f{\Vert x_n \Vert}n \to 0$.
  Тогда в силу непрерывности оператора
  $\left\Vert A \f{x_n}n \right\Vert \geq 0$.
  Пусть оператор $A$ является ограниченным, тогда
  $\exists c>0:\Vert Ax \Vert \leq c \Vert x \Vert \sep \forall x \in X$,
  пусть $x_n \to 0$, тогда $\Vert Ax_n \Vert \leq x \sep \Vert x_n \Vert \to 0$,
  и тогда оператор непрерывен в точке ноль и согласно лемме непрерывен
  в любой точке пространства.
\end{proof}

\subsubsection{Норма линейного оператора.}

Рассмотрим ограниченный (является непрерывным) линейный оператор $A$
определенном на всем
пространстве $X$, нормой оператора $A$ называют число:
\[\Vert A \Vert = \sup_{\Vert x \Vert \leq 1} \Vert Ax \Vert\]

эквивалентное определение нормы оператора.
\begin{dgroup*}
\[\Vert A \Vert = \sup_{\substack{x \in X \\ x \neq 0}}
\f{\Vert Ax \Vert}{\Vert x \Vert}\]
\[\Vert A \Vert = inf \{c: \Vert Ax \Vert \leq c \Vert x \Vert \sep
\forall x \in X\}\]
\end{dgroup*}

Легко видеть, что
$\Vert Ax \Vert \leq \Vert A \Vert \cdot \Vert x \Vert \sep \forall x \in X$

\subsubsection{Линейные функционалы}
Рассмотрим линейные ограниченный функционал $f(x)$ определенном
на действительном пространстве $X$.

$f: X \to R^1$ - это частный случай линейного оператора на который переносится
введенные выше определения.
\begin{dgroup*}
\[\Vert f \Vert = \sup_{\Vert x \Vert \leq 1} |f(x)|\]
\[\Vert f \Vert = \sup_{\substack{x \in X \\ x \neq 0}}
\f{|f(x)|}{\Vert x \Vert}\]
\[\Vert f \Vert = inf \{c: |f(x)| \leq c \Vert x \Vert \sep
\forall x \in X\}\]
\[ |f(x)| \leq \Vert f \Vert \cdot \Vert x \Vert \sep \forall x \in X\]
\end{dgroup*}

\begin{theorem}[Хана-Банаха]
\label{th:Хана-Банаха}
  Всякий линейный ограниченный функционал определенный на некотором
  линейном многообразии действительного нормированного пространства $X$
  может быть продолжен на все пространство без увлечения нормы.
\end{theorem}

Это одна из фундаментальных теорем функционального анализа.

\begin{corollary}
  Если точка $x_0$ не нулевой элемент пространства $X$, то существует линейный
  ограниченный функционал $f(x)$ определенный на всем пространстве
  $\Vert f \Vert = 1$, обладающий тем же свойством, что и
  $f(x_0) = \Vert x_0 \Vert$.
\end{corollary}

\begin{proof}
  Выберем произвольный ненулевой $x \in X$ и определим функционал на
  линейном многообразии $\{\alpha x_0\} \subset X$,
  где $\alpha$ - произвольное число.
  $f(\alpha x_0) = \alpha \Vert x_0 \Vert$, $f(x_0) = \Vert x_0 \Vert$.
  Найдем норму функционала $|f(\alpha x_0)| = |\alpha| \Vert x_0 \Vert =
  \Vert \alpha x_0 \Vert$, $\Vert f \Vert = 1$. Согласно теореме
  \ref{th:Хана-Банаха} Хана-Банаха этот функционал может быть распространен
  на всем пространстве $X$ без увлечения нормы.
\end{proof}

\textbf{Замечание!} Можно так же рассмотреть линейные ограниченные функционалы,
определенном на комплексном нормированном пространстве $X$, которое отображает
это пространство в комплексную плоскость, определения норму в этом случае
сохраняются, для таких функционалов доказана теорема аналогичная теореме
Хана-Банаха.

\textbf{Лекция №18}

\subsection{Пространства линейных непрерывных операторов и его полнота.
Сопряженное пространство.}

\begin{definition}[множество всех линейных непрерывных операторов]
  $L(X,Y)$ - множество всех линейных непрерывных операторов определенных на всем
  пространстве $X$ со значением в пространстве $Y$ ($x$, $Y$ - нормированные
  пространства)
\end{definition}

Определим сумму операторов и произведение оператора на число.

Если $A, B \in L(X,Y)$ и $(A+B)x = Ax + Bx \sep (\alpha A)x = \alpha Ax \sep
\forall x \in X$

При таких определениях множества $L(X,Y)$ образует линейное пространство
(все аксиомы выполняются). Это пространство становится нормированным, если
определить норму оператора так, как это сделано в предыдущем пункте.
\[\Vert A \Vert = \sup_{\Vert x \Vert \leq 1} \Vert Ax \Vert\]

Легко видеть, что все аксиомы при таком определении нормы выполняются.

\subsubsection{Сходимость по норме в пространстве операторов.}

Рассмотрим последовательность операторов $\{A_n\} \subset L(X,Y)$.

Говорят, что эта последовательность сходится к оператору $A \in L(X,Y)$, если
$\Vert A_n - A \Vert \underset{n \to \infty}{\to} 0$.

Из сходимости по норме пространства операторов вытекает:
\be
  \item Поточечная сходимость
  \[A_nx \to Ax \sep \Vert A_nx - Ax \Vert \underset{n \to \infty}{\to} 0 \sep
  x \in X\]

  \item Равномерная сходимость на любом ограниченном множестве пространства $X$.
\ee

То есть для $\forall \epsilon > 0 \exists N = N(\epsilon): \Vert A_nx-Ax \Vert
< \epsilon \sep \forall n \geq N \sep \forall x \in M$

Действительно если $\Vert A_n - A \Vert \to 0 \sep \forall \epsilon > 0 \sep
\exists N: \Vert A_n - A \Vert < \epsilon \sep \forall n \geq N$,
а если множество $M$ ограниченное, то $\exists c \geq 0: \Vert x \Vert \leq c
\sep \forall x \in M$, тогда $\Vert A_nx - Ax \Vert \leq \Vert A_n - A \Vert
\cdot \Vert x \Vert \to 0$, что и указывает на поточечную сходимость.

$\Vert A_nx - Ax \Vert \leq \Vert A_n-A \Vert \cdot \Vert x \Vert <
\epsilon \cdot c \sep \forall x \in M \sep \forall n \geq N$, что и указывает
на равномерную сходимость.

Отметим что из поточечной сходимости сходимость по норме пространства
операторов не следует. А вот из равномерной сходимости последовательности
оператора хотя бы из единичного оператора следует сходимость пространства
операторов.
\[A_nx \to Ax \sep \Vert x \Vert \leq 1 \sep \forall \epsilon > 0 \sep
\exists N : \Vert A_nx-Ax \Vert < \epsilon \forall n \geq N \sep
\forall x \sep \Vert x \Vert \leq 1\]

Тогда $\Vert A_n - A \Vert = \sup_{\Vert x \Vert \leq 1} \Vert A_cx - Ax \Vert
\leq \epsilon < 2 \epsilon \sep \forall n \geq N$, а это говорит, что
$\Vert An - A \Vert \to 0$

\begin{theorem}
  Если пространство $X$ нормированное, а пространство $Y$ банахово,
  то пространство операторов $L(X,Y)$ является полным (банаховым).
\end{theorem}

\begin{proof}
  Рассмотрим произвольную функциональную последовательность операторов
  пространства $L(X,Y)$.
  \[\forall \epsilon > 0 \sep \exists N : \Vert A_{n+p}-A_n \Vert < \epsilon
  \sep \forall n \geq N \sep \forall p = 1,2,\dots\]
  Для доказательства полноты пространства операторов $\forall A \in L(X,Y)$
  к которому сходится последовательность $A_n \to A$. Возьмем произвольный
  элемент $x \in X$ и положим, что последовательность $\{A_nx\}$ является
  фундаментальной в пространстве $Y$.
  \[\Vert A_{n+p}x-A_nx \Vert \leq \Vert A_{n+p} - A_n \Vert \cdot
  \Vert x \Vert \leq \epsilon \Vert x \Vert \sep \forall n \geq N \sep
  \forall p = 1,2, \dots \]
  Так как она фундаментальная, значит, что она сходится в $Y$, тогда существует
  такой предел $\ds \lim_{n \to \infty} A_nx = Ax$.
  Докажем, что определенный таким образом оператор $A$ является непрерывным.
  Воспользуемся неравенством для нормы.
  \[|\ \Vert A_{n+p} \Vert - \Vert A_n \Vert \ | \leq \Vert A_{n+p} - A_n \Vert
  \leq \epsilon \sep \forall n \geq N \sep \forall p = 1,2,\dots\]
  От сюда вытекает, что $\{\Vert A_n \Vert\}$ - фундаментальная
  последовательность.
  Фундаментальная числовая последовательность является ограниченной, от сюда
  следует ограниченность оператора $A$.
  \begin{dgroup*}
    \[\exists c \geq 0 : \Vert A_n \leq c \sep \forall n = 1,2, \dots\]
    \[\Vert A_nx \Vert \leq \Vert A_n \Vert \cdot \Vert x \Vert \leq
    c \Vert x \Vert\]
    \[\Vert Ax \Vert \leq c \Vert x \Vert\]
  \end{dgroup*}
  Что и указывает на ограниченность (непрерывность) оператора $A$,
  следовательно оператор $A \in L(X,Y)$.
  В неравенстве $\Vert A_{n+p}x - A_nx \Vert \leq \Vert A_{n+p} - A_n \Vert
  \cdot \Vert x \Vert < \epsilon \Vert x \Vert$ перейдем к пределу при
  $p \to \infty$.
  \[\Vert A-A_n \Vert \cdot \Vert x \Vert < \epsilon \Vert x \Vert \sep
  \forall n \geq N\]
  От сюда выходит что $\Vert A - A_n \Vert \leq \epsilon \sep \forall n \geq N$.
  Итак фундаментальная последовательность операторов, а следовательно
  и пространство, полное.
\end{proof}

\subsubsection{Сопряженное пространство}

Рассмотрим множество линейных непрерывных функционалов определенных на
пространстве $X$.

\begin{definition}[сопряженное пространство]
  Пространство $L(X,R^1)$ называют сопряженным к пространству $X$ и
  обозначается $X^*$ ($L(X,R^1)=X^*$)
\end{definition}

Согласно доказательству теоремы сопряженное пространство всегда является полным.

Если пространство $X$ не является полным, то согласно теореме
\ref{th:о пополнении нормированного пространства}
о пополнении нормированного пространства оно может быть включено в полное
пространство в качестве всюду плотного линейного многообразия в замыкание $X$
($[X]$).

Сопряженные пространства $X^*$ и $[X]^*$ оказываются изоморфными, то есть между
их элементами можно установить взаимооздначное соответствие сохраняющие
линейные операции и нормы. Этот вывод Следует из теоремы \ref{th:Хана-Банаха}
Хана-Банаха: любое функциональное определение можно распространить на $[X]$,
с сохранением нормы.
Единственность такого распространения вытекает из непрерывности.

Пространство операторов $L(X,Y)$ - пространство линейных непрерывных
операторов, которые отображают пространство $X$ в себя. В частности этому
пространству принадлежит единичный или тождественный оператор
$Ix=x \sep \forall x \in X$.

\textbf{Лекция №19}

В пространстве $L(X,X)$ можно определить произведение операторов как их
суперпозицию (результат последовательного выполнения).

Пусть операторы $A,B \in L(X,X)$, тогда произведение операторов определяется
$(AB)x = A(Bx) \sep \forall x \in X$.
Несложно доказать, что норма произведения не превышает норму $A$ на норму $B$:
$\Vert AB \Vert \leq \Vert A \Vert \cdot \Vert B \Vert$.

В общем случае произведение операторов некомутативно,
но обладает свойством ассоциативности: $(A+B)C = AC+BC$, $A(B+C) = AB+AC$,
$AI = IA = A$.

\subsubsection{Непрерывность произведения операторов.}

Если $A_n \to A \sep B_n \to B$ по норме $L(X,X)$, то $A_nB_n \to AB$.
Действительно, $\Vert A_n B_n - AB \Vert = \Vert A_n(B_n - B) -
(A-A_n)B \Vert \leq \Vert A_n \Vert \cdot \Vert B_n - B \Vert +
\Vert A - A_n \Vert \cdot \Vert B \Vert \to 0$.

Степень оператора определяется очевидным образом
$A^1 = A \sep A^{n+1} = AA^n \sep n = 1,2,\dots$.

\subsection{Примеры линейных операторов и функционалов.}

\subsubsection{Общий вид линейного оператора в конечномерном пространстве.}

Рассмотрим оператор $A: R^n \to R^n$,
пусть $x = (x_1,\dots,x_n)\in R^n \sep y = (y_1,\dots,y_n)\in R^n$,
$e_i$ - базис в $R^n$, $f_j$ - базис в $R^m$.
Тогда $\ds x = \sum_{i=1}^n x_ie_i$, а его образ
\[y = Ax = \sum_{i=1}^n x_i\underbrace{Ae_i}_
{\parbox{4em}{\centering\small образ\\базисных\\векторов}}
= \sum_{i=1}^n x_i \underbrace{\sum_{j=1}^m a_{ji}f_j}_
{\parbox{5em}{\centering\small разложение\\базис\\векторов\\в $f_j$}}\]

Матрица $a_{ji}$ содержит коэффициенты разложения базисных векторов.
\[y = Ax = \sum_{j=1}^m y_j f_j = \sum_{j=1}^m f_j \sum_{i=1}^n a_{ji} x_i\]

Мы получили общий вид линейного оператора в конечномерном пространстве.
Любой такой оператор задается в пространстве матрицей.
Любой линейный оператор действующий в конечномерном пространстве
автоматически непрерывен, а значит и ограничен.
Вычислим норму этого оператора $A$.
При вычислении нормы оператора поступают следующим образом.
\be
  \item Оценивают норму $\Vert Ax \Vert$ сверху
  $\Vert Ax \Vert \leq c \Vert x \Vert \sep \forall x \in X \sep
  \Vert A \Vert \leq c$.

  \item Подбирается элемент пространства $\xi \in X$, такой что
  $\f {\Vert A\xi \Vert}{\Vert \xi \Vert} \geq c$. Нашли оценку снизу.
\ee
Вычислим норму оператора $A$, как оператора $A: R_\infty^n \to R_\infty^n$,
считаем:
\[\Vert A_x \Vert = \max_{j=\overline{1,m}}|y_j|=
\max_{j=\overline{1,m}}\left|\sum_{i=1}^na_{ij}x_i\right| \leq
\underbrace{\max_{j=\overline{1,m}}|\sum_{i=1}^n|a_{ji}|}_{c}
\underbrace{\max_{j=\overline{1,n}}|x_i|}_{\Vert x \Vert}\]

Докажем что норма оператора $A$ это и есть $c$.
Пусть наибольшее значение суммы достигается при $j=j_0$:
$\ds c=\sum_{j=1}^n |a_{j_0i}|$

Рассмотрим элемент пространства
$\xi = (sign\ a_{j_0i})_{i=1,2,\dots}\in R_\infty^n \sep \Vert \xi \Vert = 1$.
\[\Vert A\xi \Vert = \max_{j=\overline{1,m}}
\left|\sum_{i=1}^n a_{ji} \cdot sign\ a_{j_0i}\right|\]

Так как $c$ достигается при $j_0$, а где то может быть и больше:
$\f{\Vert A\xi \Vert}{\Vert \xi \Vert} \geq c$.
Вывод $\Vert A \Vert = c$.

Вычислим теперь ному оператора
$A: R_p^n \to R_q^n \sep p > 1 \sep \f{1}{p}+\f{1}{q} = 1$.
\[\Vert Ax \Vert^q = \sum_{j=1}^m |y_j|^q=
\sum_{j=1}^m\left|\sum_{i=1}^n A_{ji}x_i\right|^q\]
Воспользуемся неравенством Гельдера.
\begin{dgroup*}
\[\Vert Ax \Vert^q \leq
\sum_{j=1}^m \left(\left(\sum_{i=1}^n |a_{ji}|^q\right)^{\f{1}{q}}
\left(\sum_{i=1}^n |x_{i}|^p\right)^{\f{1}{p}}\right)^q=
\sum_{j=1}^m \sum_{i=1}^n |a_{ji}|^q
\underbrace{\left(\sum_{i=1}^n |x_i|^p\right)^{\f{q}{p}}}_{\Vert x \Vert ^ q}\]
\[\Vert Ax \Vert \leq
\underbrace{\left(\sum_{j=1}^m \sum_{i=1}^n |a_{ji}|^q\right)^{\f{1}{q}}}_{c}
\cdot \Vert x \Vert \sep \Vert A \Vert \leq c\]
\end{dgroup*}

Аналогично предыдущему случаю можно подобрать такой элемент,
что $c$ достигается.

\subsubsection{Линейные функционалы в конечномерных пространствах.}

Рассмотрим линейный функционал $\ds f: R^n \to R^1 \sep
x=(x_1,\dots,x_n) \in R^n \sep f(x)=\sum_{i=1}^n a_ix_i$.

Это и есть общий вид линейного функционала в конечномерном пространстве.
$a=(a_1,\dots,a_n)\in R^n$ - некоторый фиксированный элемент из $R^n$
Вычислим норму функционала, как функционала $f: R_p^n \to R^1 \sep p>1$.
\begin{dgroup*}
\[|f(x)|=\left|\sum_{i=1}^n a_ix_i\right| \leq
\left(\sum_{i=1}^n |a_i|^q \right)^{\f{1}{q}}
\left(\sum_{i=1}^n |x_i|^p\right)^{\f{1}{p}}
\leq \Vert a \Vert_{R_q^n} \cdot \Vert x \Vert_{R_p^q}\]
\[\f{1}p + \f{1}q=1 \sep \Vert f \Vert \leq \Vert a \Vert_{R_q^n}\]
\end{dgroup*}
Возьмем $\xi = (|a_i|^{q-1}sign\ a_i)_{i=1,\dots,n}$
\begin{dgroup*}
\[\Vert \xi \Vert_{R_p^n} =
\left(\sum_{i=1}^n |a_i|^{p(q-1)}\right)^{\f{1}{p}} =
\left\{\f{1}{p}=\f{q-1}{q}\sep p(q-1) = 1\right\}
= \left(\sum_{i=1}^n |a_i|^q\right)^{\f{1}{p}}
= \Vert a \Vert_{R_q^n}^{\frac{q}{p}}\]
\[|f(\xi)| = \left|\sum_{i=1}^n a_i |a_i|^{q-1} \cdot sign \ a_i \right| =
\sum_{i=1}^n |a_i|^{q} = \Vert a \Vert_{R_q^n}^n\]
\[\f{|f(\xi)|}{\Vert\xi\Vert} = \Vert a \Vert_{R_q^n} \sep
\Vert f \Vert = \sup_{\substack{x \in X \\ x \neq 0}}\f{|f(x)|}{\Vert x \Vert}\]
\end{dgroup*}

Норма достигается в $R_q^n$: $\Vert f \Vert = \Vert a \Vert_{R_q^n}$.

Таким образом между множеством линейных функционалов заданных на $R_p^n$ и
пространством $R_q^n$ установлен изоморфизм.
Следовательно пространство сопряженное к $R_p^n$, это
$R_q^n: (R_p^n)^*=R_q^n \sep p>1 \sep \f{1}{p}+\f{1}{q}=1$.
Можно доказать, что $(R_1^n)^* = R_\infty^n$.

Общий вид линейного функционала в пространстве последовательностей $l_p$
\[f(x)=\sum_{n=1}^\infty a_nx_n\]
Где $a=(a_1,\dots,a_n,\dots) \in l_q: \Vert f \Vert= \Vert a \Vert_{l_q}$.

\textbf{Лекция №20}

\subsubsection{Линейный функционал в пространстве непрерывных функций}

Рассмотрим функционал определенный равенством
$\ds f(x)=\int_a^b x(t)x_0(t)dt \sep x(t) \in C[a,b] \sep x_0(t) \in C[a,b]$.

Причем $x_0(t)$ имеет конечное число точек перемены знака.
Линейность этого функционала очевидна.
Докажем его ограниченность и найдем норму.
\begin{dgroup*}
\[|f(x)| \leq \int_a^b |x(t)|\cdot |x_0(t)|dt \leq
\int_a^b |x_0(t)|dt \underbrace{\max_{a \leq t \leq b}
|x(t)|}_{\Vert x \Vert} \]
\[\Vert f \Vert \leq \int_a^b |x_0(t)| dt\]
\end{dgroup*}

Рассмотрим функцию $s(t)=sign\ x_0(t)$, тогда $\ds f(s)=\int_a^b |x_0(t)| dt$.

Однако функция $s(t)$ не является непрерывной,
тем не менее $\forall \epsilon > 0$ можно построить непрерывную функцию
$s_\epsilon (t)$ обладающую тем свойством, что $f(s _ \epsilon)$
будет больше $\ds \int_a^b |x_0(t)|dt - \epsilon$.

Тогда по определению $sup$ получаем,
что $\Vert f \Vert = \ds \int_a^b |x_0(t)| dt$.

\subsubsection{Линейный функционал в евклидовом пространстве.}

Рассмотрим евклидово пространство $X$ и некоторый фиксированный элемент
$x_0 \in X$. Определим функционал $f(x)=(x,x_0)$.
Линейность функционала очевидна в силу аксиом скалярного произведения,
ограниченность вытекает из неравенства Коши-Буняковского.
\begin{dgroup*}
\[|f(x)| = |(x,x_0)| \leq \Vert x_0 \Vert \cdot \Vert x \Vert \sep
{\forall x \in X} \sep \Vert f \Vert \leq \Vert x_0 \Vert\]
\[|f(x_0)| = (x_0, x_0) = \Vert x_0 \Vert \cdot \Vert x_0 \Vert \sep
\f{|f(x_0)|}{\Vert x_0 \Vert} = \Vert x_0 \Vert \sep \Vert f \Vert \geq
\Vert x_0 \Vert\]
\[\Vert f \Vert = \Vert x_0 \Vert\]
\end{dgroup*}

\subsubsection{Дельта-функция Дирака.}

Дельта-функция Дирака это функционал определенный на пространстве непрерывных
функций, который к каждой функции этого пространства ставит в соответствие
значение этой функции в некоторой фиксированной точке промежутка
$[a,b]$: $\delta_{t_0}(x) = x(t_0)$.
\[|\delta_{t_0}(x)| = |x(t_0)| \leq \max_{a \leq x \leq b} | x(t)| =
\Vert x \Vert \sep \Vert\delta_{t_0}\Vert \leq 1\]

Рассмотрим функцию $x(t) = c \sep c=const$,
тогда $|\delta_{t_0}(x)| = |c| = \Vert x \Vert \sep
\Vert \delta_{t_0} \Vert = 1$

\subsubsection{Оператор ортогонального проектирования (ортопроектор).}

Рассмотрим гильбертово пространство $H$ и некоторое подпространство
$M \subset H \sep M^\bot$ - ортогональное дополнение этого пространства.
Согласно теореме~\ref{th:об ортогональном проектировании}
об ортогональном проектировании любой элемент
$x \in M$ единственным образом  может
быть представлен в виде $x=y+h \sep y \in M \sep h \in M^\bot$, где
$y$ - проекция элемента $x$ на подпространство $M$.
$h$ - перпендикуляр, опущенный из $x$ на подпространство.
При этом справедлив аналог теоремы Пифагора
$\Vert x \Vert^2 = \Vert y \Vert^2 + \Vert h \Vert ^2$.

Таким образом в гильбертовом пространстве определен оператор ортогонального
проектирования (ортопроектор) на подпространство: $y=Px$.
Этот оператор каждому элементу $X$ ставит в соответствие его проекцию
на подпространство $M$ ($P:H \to H$):
$x=Px+h$, отсюда $h = (I-P)x \sep x \in M \Leftrightarrow x = Px$.

Линейность этого оператора очевидна, докажем ограниченность и найдем норму:
\begin{dgroup*}
\[\Vert Px \Vert ^2 = \Vert x \Vert ^2 - \Vert h \Vert ^2 \sep
\Vert Px \Vert \leq \Vert x \Vert \sep \Vert P \Vert \leq 1\]
\[\Vert Px \Vert = \Vert x \Vert \sep \forall x \in M \sep \Vert P \Vert = 1\]
\end{dgroup*}

\subsubsection{Линейный оператор на пространстве непрерывных функций}

\[A: C[a,b] \to C[a,b] \sep Ax = x(t)x_0(t)\]
где $x_0(t)$ некоторый фиксированный элемент в $C[a,b]$.
Линейность очевидна.
\begin{dgroup*}
\[\Vert Ax \Vert = \max_{a \leq t \leq b} |x(t)x_0(t)| \leq
\max_{a \leq t \leq b} |x_0(t)|\max_{a \leq t \leq b} |x(t)| =
\Vert x_0 \Vert \cdot \Vert x \Vert\]
\[\Vert A \Vert \leq \Vert x_0 \Vert \sep
\Vert Ax_0 \Vert = \Vert x_0 \Vert \cdot \Vert x_0 \Vert \sep
\Vert A \Vert = \Vert x_0 \Vert\]
\end{dgroup*}

\subsubsection{Оператор дифференцирования.}

Рассмотрим оператор $D$ который каждой непрерывно дифференцируемой функции из
пространства $C[a,b]$ ставит в соответствие ее производную:
\[Dx = \f{dx}{dt}\]

Оператор $D$ определен на линейном многообразии непрерывно дифференцируемой
функции. Он не является непрерывным на пространстве $C[a,b]$. Действительно
$x_n =\f{\sin nt}{n} \to 0$, $Dx_n = \cos nt$,
а это расходящаяся последовательность, которая не является ограниченной,
поэтому норму не считаем.

Оператор дифференцирования оказывается линейно ограниченным оператором,
если рассматривать его как оператор заданный на пространстве
непрерывно дифференцируемых функций $C^1[a,b]$ с нормой
$\ds \Vert x \Vert = \max_{a \leq t \leq b} | x(t)| +
\max_{a \leq t \leq b} \left| \f{dx}{dt} \right|$,
в этом случае оператор будет непрерывным.

\subsubsection{Оператор Фредгольма.}
\[K: L_2[a,b] \to L_2[a,b] \sep Kx = \int_a^b K(s,t) x(t) dt\]

Предполагается, что ядро $K(s,t)$ интегрируемо по Лебегу в квадрате в области
$a \leq s,t \leq b$, то есть $\ds\int_a^b \int_a^b |K(s,t)|^2 dsdt$.
\[\Vert Kx \Vert^2 = \int_a^b \left(\int_a^b K(s,t) x(t)dt \right)^2 ds\]

Воспользуемся неравенством Гельдера:
\[\Vert Kx \Vert^2 \leq \int_a^b \int_a^b |K(s,t)|^2 dt
\underbrace{\int_a^b |x(t)|^2 dt}_{\Vert x \Vert ^ 2} ds
\leq \left(\int_a^b \int_a^b |K(s,t)|^2 dsdt\right)^{\f{1}{2}} \Vert x \Vert\]

В норме оператора справедлива оценка $\Vert Kx \Vert^2$.

\subsubsection{Общий вид линейного функционала в пространстве
$L_p[a,b] \sep p > 1$.}

Рисс в 1910 году доказал, что любой линейный функционал на
пространстве $L_p$ на отрезке $[a,b]$ определяется равенством
$\ds f(x) = \int_a^b x(t) \phi(t) dt \sep x(t) \in L_p[a,b] \sep
\phi(t) \in L_q[a,b] \sep \f{1}p+\f{1}q=1$, где $\phi(t)$ определена однозначно,
с точностью до эквивалентности и норма функционала:
$\Vert f \Vert = \Vert \phi \Vert_{L_q[a,b]}$,
тем самым установлено, что пространству сопряженному к $L_p[a,b]$
является $L_q[a,b]$.

\subsection{Общий вид линейного функционала в гильбертовом пространстве.}

\begin{theorem}[Рисса, об общем виде функционала]
  Пусть $f(x)$ линейный ограниченный функционал определенный на всем
  гильбертовом пространстве $H$, тогда существует единственный элемент
  $x_0 \in H$ такой, что $f(x)=(x,x_0) \sep \forall x \in H$
  и норма функционала $\Vert f \Vert = \Vert x_0 \Vert$.
\end{theorem}

И наоборот каждый элемент $x_0$ определяет функционал $f(x)=(x,x_0)$ с нормой
$\Vert f \Vert = \Vert x_0 \Vert$.

Теорема Рисса устанавливает изоморфизм между сопряженным пространством $H^*$
(пространством функционала) и пространством $H$,
таким образом гильбертово пространство является самосопряженным.

\textbf{Лекция №21}

\begin{proof}
  Рассмотрим произвольный линейный непрерывный функционал $f(x)$
  заданный на всем гильбертовом пространстве $H$.
  Если $f(x)=0 \sep \forall x \in H$, то $x_0=0$.
  Введем в рассмотрение множество нулей $H_0 = \{x \in H : f(x) = 0\}$.
  Очевидно, что $H_0$ является подпространством гильбертова пространства $H$,
  то есть замкнутым линейным многообразием.
  Пусть $H_0^\bot$ ортогональное дополнение пространства $H$:
  \[H_0^\bot = \{x \in H: (x,y)=0 \sep y \in H\].

  Выберем некоторый фиксированный элемент $h \in H_0^\bot$
  и для любого $x \in H$ положим $y=x- \f{f(x)}{f(h)}h$.
  В силу линейности функционала $f(y)=0$.
  Следовательно $y \in H_0$, тогда $(y,h)=0$.
  и тогда $0=(y,h)=(x,h)-\f{f(x)}{f(h)}\underbrace{(h,h)}_{\Vert h \Vert ^2}$.
  Следовательно $\f{f(x)}{f(h)} \Vert h \Vert ^2 = (x,h)$,
  тогда $f(x)= (x,\f{f(h)}{\Vert h \Vert ^2} h)=(x,x_0)$,
  где $x_0=\f{f(x)}{\Vert h \Vert^2}h$.

  Единственность этого элемента вытекает из теоремы об ортогональном
  проектировании, которая утверждает, что
  $x=y+z \sep y \in H_0 \sep z \in H_0^\bot$.
  Теорема доказана.
\end{proof}

\textbf{Замечание!} Размерность подпространства $h_0^\bot$ равняется
$1$: $dim\ H_0^\bot = 1$.


\subsection{Пространство основных и обобщенных функций.
Регулярные и сингулярные обобщенными функциями.}

Рассмотрим множество финитных бесконечно дифференцируемых функций
определенных на числовой прямой, обозначим его $K$. Функция $\phi(t) \in K$,
если она общается в $0$ вне некоторого конечного отрезка (финитного) и
является бесконечно дифференцируемой.
Множество таких функций образуют линейное пространство с обычными
алгебраическими операциями.
На этом пространстве норма не определяется, но вводится понятие сходимости.

\begin{definition}[сходящаяся последовательность элементов множества
финитных бесконечно дифференцируемых функций]
  Последовательность элементов $\phi_n \in K$ называется сходящаяся к элементу
  $\phi$, то есть $\phi_n \underset{K}{\to}\phi$.
  Если существует конечный отрезок вне которого все функции $\phi(t)=0$ и
  последовательность производных
  $\f{d^k\phi_n}{dt^k} \rightrightarrows \f{d^k\phi}{dt^k} \sep k=0,1,\dots$.
\end{definition}

При этом равномерность сходимости по $K$ не предполагается.

\begin{definition}[обобщение функции]
  Общение функции называется линейный непрерывный функционал определенный
  на пространстве основных функций $T(\phi)$ и $T \in K^*$,
  $K^*$ - пространство обобщенных функций.
\end{definition}

Непрерывность этого функционала означает,
что $T(\phi_n) \to T(\phi) \sep \forall \{\phi_n\}\in K: \phi_n \to \phi$.
Любая локально интегрируемая функция $f(t)$
(интегрируемая на любом конечном промежутке) определяет
обобщенную функцию (функционал):
\[T(\phi)=\int_{-\infty}^\infty f(t)\phi(t)dt \sep \forall \phi \in K\]

\begin{definition}[регулярние обобщенные функции]
  Обобщенные функции, которые определяются таким образом называются регулярными.
\end{definition}

Между множеством регулярных обобщенных функций и множеством локально интегрируемых
функций можно установить взаимно однозначное соответствие.
Поэтому обобщенную функцию, которая порождается локально интегрируемой
функцией $f(t)$ обозначают точно также $f(t)$.

Рассмотрим функцию Хевисайда:
\[\theta (t) = \begin{cases}
1, t>0 \\ 0, t<0\end{cases}\].

Эта функция порождает регулярную обобщенную функцию и обозначают $f(t)$.
\[\theta (\phi) = \int_{-\infty}^\infty \theta (t)\phi(t)dt
= \int_0^\infty \phi(t)dt\]

Значит регулярная обобщенная функция $\theta$ ставит
$K \ni \phi \overset{\theta}{\to} \int_0^\infty \phi(t) dt \sep \theta \in K^*$.

\[\theta _{t_0}(\phi)=\int_{-\infty}^\infty \theta (t-t_0)\phi(t)dt
= \int_{t_0}^\infty \phi (t)dt\]

\begin{definition}[сингулярные обобщенные функции]
  Все обобщенные функции, которые не являются регулярными,
  называются сингулярными.
\end{definition}

Примером сингулярной обобщенной функцией является $\delta$-функция Дирака.
\begin{dgroup*}
\[\delta(\phi)=\phi(0)\]
\[\delta_{t_0}(\phi) = \phi(t_0)\]
\end{dgroup*}

Очевидно, что $\delta$-функция не является регулярной обобщенной функцией,
так как не существует локально интегрируемой функции $\delta(t-t_0)$
обладающей свойством:
\[\phi(t_0) = \int_{-\infty}^\infty \delta (t-t_0)\phi(t)dt\]

Тем не менее формально значение обобщенной функции записывают в виде
такого интеграла.

Оказывается, что $\delta$-функция Дирака может быть предствалена как
предел последовательности регулярных обобщенных функций.

Сходимость в пространстве обобщенных функций $K^*$ определяется
как поточечной сходимость.
Говорят, что $\{T_n\} \subset K^* \sep T_n \to T \in K^*$,
если $T_n(\phi) \to T(\phi) \sep \forall \phi \in K$.


\listoftheorems

%\end{multicols}
\end{document}
