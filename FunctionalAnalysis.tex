\documentclass[12pt]{report}
%\documentclass[6pt,fleqn,leqno]{report}

\usepackage[warn]{mathtext}
            % русские буквы в формулах, с предупреждением
\usepackage[T2A]{fontenc}
            % внутренняя кодировка  TeX
\usepackage[utf8]{inputenc}
            % кодовая страница документа
\usepackage[english, russian]{babel}
            % локализация и переносы
\usepackage{indentfirst}
            % русский стиль: отступ первого абзаца раздела
\usepackage{misccorr}
            % точка в номерах заголовков
\usepackage{cmap}
            % русский поиск в pdf
\usepackage{graphicx}
            % Работа с графикой \includegraphics{}
\usepackage{psfrag}
            % Замена тагов на eps картинкаx
\usepackage{caption2}
            % Работа с подписями для фигур, таблиц и пр.
\usepackage{soul}
            % Разряженный текст \so{} и подчеркивание \ul{}
\usepackage{soulutf8}
            % Поддержка UTF8 в soul
\usepackage{fancyhdr}
            % Для работы с колонтитулами
\usepackage{multirow}
            % Аналог multicolumn для строк
%\usepackage{ltxtable}
            % Микс tabularx и longtable
%\usepackage{paralist}
            % Списки с отступом только в первой строчке
\usepackage{mathtools}
            % Mathtools provides a series of packages designed
            % to enhance the appearance of documents containing
            % a lot of mathematics.
\usepackage{tikz}
            % Tikz is probably the most complex and powerful tool to create
            % graphic elements in LaTeX. In this article some of the basics
            % will be explained: lines, dots, curves, circles, rectangles, etc.
%\usepackage{longtable}
%\usepackage{tabularx}
\usepackage{amsmath}
            % The amsmath package is a LATEX package that provides
            % miscellaneous enhancements for improving the information
            % structure and printed output of documentsthat contain
            % mathematical formulas.
\usepackage{geometry}
%\geometry{a4paper, total={170mm,257mm}, left=20mm, top=20mm}
%\geometry{a5paper, total={128mm,180mm}, left=10mm, top=10mm}
            % This package provides a flexible and easy interface
            % to page dimensions.
\usepackage{makeidx}
            % Standard package for creating indexes
\usepackage[pdftex, colorlinks=true, pdfstartview=FitV, linkcolor=blue, citecolor=blue, urlcolor=blue, plainpages=false]{hyperref}
            % The hyperref package is used to handle cross-referencing
            % commands in LaTeX to produce hypertext links in the document.
%\usepackage{mathptmx}
%\usepackage[12pt]{moresize}
            % A package for using font sizes
\usepackage{pgfplots}
            % PGFPlots draws high-quality function plots in normal or
            % logarithmic scaling with a user-friendly interface
            % directly in TeX.
%\usepackage{breqn}
%\usepackage{autobreak}
%\allowdisplaybreaks
\usepackage{multicol}
\usepackage{breqn}
\usepackage{mathrsfs}

%\usepackage{geometry}
%\geometry{a4paper, landscape, total={277mm, 190mm}, left=10mm, top=10mm}
%\pgfplotsset{
  %width=57mm,
  %compat=1.13,
  %axis lines=middle,
  %xtick=\empty,
  %ytick=\empty,
%}
%\usepackage{titlesec}
%\titleformat{\chapter}[display]
  %{\normalfont\small\bfseries}{\chaptertitlename\ \thechapter}{1em}{}
%\titleformat{\section}
  %{\normalfont\small\bfseries}{\thesection}{1em}{}
%\newcommand{\pltwdth}{4cm}

\newcommand{\pltwdth}{5cm}


\makeatletter
\newcommand*{\relrelbarsep}{.386ex}
\newcommand*{\relrelbar}{%
  \mathrel{%
    \mathpalette\@relrelbar\relrelbarsep
  }%
}
\newcommand*{\@relrelbar}[2]{%
  \raise#2\hbox to 0pt{$\m@th#1\relbar$\hss}%
  \lower#2\hbox{$\m@th#1\relbar$}%
}
\providecommand*{\rightrightarrowsfill@}{%
  \arrowfill@\relrelbar\relrelbar\rightrightarrows
}
\providecommand*{\leftleftarrowsfill@}{%
  \arrowfill@\leftleftarrows\relrelbar\relrelbar
}
\providecommand*{\xrightrightarrows}[2][]{%
  \ext@arrow 0359\rightrightarrowsfill@{#1}{#2}%
}
\providecommand*{\xleftleftarrows}[2][]{%
  \ext@arrow 3095\leftleftarrowsfill@{#1}{#2}%
}

\newcommand{\be}{\begin{enumerate}}
\newcommand{\ee}{\end{enumerate}}

\renewcommand{\[}{$\\\displaystyle}
\renewcommand{\]}{\\$}
\renewcommand{\[}{$\\\displaystyle}
\newcommand{\sigmalgebra}{\textcircled{$\sigma$}}
\newcommand{\sep}{,\ }
\newcommand{\tth}[1][]{\textbf{Теорема#1.}}
\newcommand{\tdef}{\textbf{Определение.} }
\newcommand{\btev}[1][]{\textbf{Доказательство#1.}
}
\newcommand{\bb}[1][]{\left#1\vphantom{\int_1^2}\right.}
\newcommand{\etev}{$\Box$}
\newcommand\eeq[1][]{\stackrel{\mathclap{\normalfont\mbox{#1}}}{=}}

\makeatother

\title{\textbf{Функциональный анализ}}
\date{}

\begin{document}
%\begin{multicols}{6}
\sloppy

\maketitle

\tableofcontents

\chapter{Метрические пространства}

\textbf{Лекция 1}

Под пространством в современной математике понимается совокупность объектов
произвольной природы (функции, набор чисел, набор функций, объекты
геометрической природы) между которыми установлены соотношения аналогичные пространственным соотношениям в трехмерном пространстве.

К понятию метрического пространства приходят путем определения расстояния между элементами произвольного множества (путем определения метрики)

\section{Определение метрического пространства. Примеры метрических пространств}

Множество $X$ называется метрическим пространством, если для любых двух его
элементов $x, y \in X$ определена однозначная действительная функция
$\rho(x, y)$ - расстояние между элементами. $x$ и $y$ (иначе метрика), которая
удовлетворяет следующим условиям:

\be
  \item Аксиома неотицательности:
  \[  \rho(x, y) \geq 0 \sep\forall x, y \in X\sep
  \rho(x, y) = 0 \Leftrightarrow x = y \]

  \item Аксиома симметрии:
  \[  \rho(x, y) = \rho(y, x)\sep \forall x, y \in X \]

  \item Аксиома треугольника:
  \[  \rho(x, y) \leq \rho(x, z) + \rho(z, y)\sep \forall x, y, z \in X \]
\ee

Рассмотрим некоторые примеры метрического пространства.

\textbf{Проверить для примеров аксиомы}

При доказательстве справедливости аксиомы треугольника для этих пространств необходимо использовать неравенство Гельдера, Минковского

Неравенство Гельдера:
\[
  \bb[|]\sum_{n = 1}^\infty a_n b_n\bb[|] \leq
  \bb[(]\sum_{n = 1}^\infty |a_n|^p\bb[)]^{\cfrac{1}{p}}\cdot
  \bb[(]\sum_{n = 1}^\infty |b_n|^q\bb[)]^{\cfrac{1}{q}}\sep
  \cfrac{1}{p} + \cfrac{1}{q} = 1 \sep p \geq 1
\]
\[
  \bb[(]\int_a^b |x(t) y(t)|^p dt \bb[)]^{\cfrac{1}{p}} \leq
  \bb[(]\int_a^b |x(t)|^p dt \bb[)]^{\cfrac{1}{p}} \cdot
  \bb[(]\int_a^b |y(t)|^p dt \bb[)]^{\cfrac{1}{p}}
\]

Неравенство Минковского:
\[
  \bb[(]\sum_{n=1}^\infty|a_n+b_n|^p\bb[)]^{\cfrac{1}{p}} \leq
  \bb[(]\sum_{n=1}^\infty|a_n|^p\bb[)]^{\cfrac{1}{p}} +
  \bb[(]\sum_{n=1}^\infty|b_n|^p\bb[)]^{\cfrac{1}{p}}\sep p\geq 1
\]
\[
  \bb[(]\int_a^b|x(t)+y(t)|^p dt \bb[)]^{\cfrac{1}{p}} \leq
  \bb[(]\int_a^b|x(t)|^p dt \bb[)]^{\cfrac{1}{p}} +
  \bb[(]\int_a^b|y(t)|^p dt \bb[)]^{\cfrac{1}{p}}
\]

Эти неравенства справедливы при условии сходимости всех входящих в них рядов и интегралов.

Неравенства справедливы также и для конечных сумм. Интегралы в этих неравенствах понимаются в смысле Лебега.

\subsubsection{Примеры метрических пространств}

\be
  \item $R_p^n$ - n-мерное арифметическое пространство элементов, которое является упорядоченным набором действительных чисел. $x = (x_1, \dots, x_n) \in R_p^n$. Метрика в этом пространстве определяется равенством:

  \[
    \rho(x, y) = \bb[(]\sum_{k = 1}^n |y_k - x_k|^p\bb[)]^{\cfrac{1}{p}}\sep p \geq 1
  \]

  При $p = 2$ получается n-мерное евклидово пространство. Сразу понятно, что на одном и том же множестве метрики можно ввести разными способами.

  \item $R_\infty^n$ - n-мерное арифметическое пространство с метрикой:
  \[
    \rho(x, y) = \max_{k = 1, 2, \dots, n} |y_k - x_k| \sep p \geq 1
  \]

  Можно доказать, что:
  \[
    \lim_{p \rightarrow \infty} \rho_p(x, y) = \rho_\infty(x, y)\sep
  \]
  где $\rho_p$ - метрика в $R_p^n$, а $\rho_\infty$ - метрика в $R_\infty^n$.

  \item $l_p$ - элементами этого пространства является бесконечные числовые последовательности $x=(x_1,x_2,\dots,x_n,\dots)\in l_p$ такие, что сходится ряд:
  \[
    \sum_{n=1}^\infty|x_n|^p<\infty.
  \]
  Метрика в этом пространстве определяется равенством:
  \[
    \rho(x, y) = \bb[|]\sum_{n=1}^\infty|y_n-x_n|^p\bb[|] \sep p \geq 1
  \]
  $l_2$ - бесконечное евклидово пространство

  \item $l_\infty$ - пространство ограниченных последовательностей, элементами этого пространства являются бесконечные числовые последовательности такие, что:
  \[
    \sup_{n=1,2,\dots}|x_n|<\infty\sep\rho(x,y)=\sup_{n=1,2,\dots}|y_n-x_n|
  \]

  \item Пространство непрерывных функций $C[a, b]$, элементами этого пространства являются все возможные функции $x(t)$ непрерывные на отрезке $[a, b]$. Метрика вводится следующим образом:
  \[
    \rho (x, y) = \max_{a \leq t \leq b} |y(t) - x(t)|
  \]

  Такая метрика называется Чебышевской. Можно также рассматривать пространство непрерывных функций $C_2[a, b]$
  \[
    \rho(x,y) = \bb[(]\int_a^b |y(t)-x(t)|^2 dt\bb[)]^{\cfrac{1}{2}}
  \]

  \item $L_p[a,b]\sep p \geq 1$ - пространство функций интегрируемых по Лебегу в $p$-той степени на $[a,b]$. То есть таких функций, что существует:
  \[
    \int_a^b|x(t)|^p dt
  \]
  С метрикой:
  \[
    \rho(x,y) = \bb[(]\int_a^b|y(t)-x(t)|^pdt\bb[)]^{\cfrac{1}{p}}
  \]

  \item $L_\infty[a, b]$ - пространство ограниченных функций, элементами этого пространства являются функции $x(t)$ такие, что существует:
  \[
    \sup_{a \leq t \leq b} |x(t)|
  \]
  С метрикой:
  \[
    \rho(x,y) = \sup_{a \leq t \leq b} |y(t)-x(t)|
  \]
\ee

\section{Предельные точки множества. Сходимость в метрическом пространстве}

Важнейшим понятием анализа является понятие предела. Многие факты анализа основаны исключительно на существовании расстояния между любыми двумя точками числовой прямой и не затрагивает алгебраическую природу действительных чисел. Поэтому можно обобщить понятия предела для произвольного метрического пространства.

\textbf{Лекция 2.}

Открытым шаром с центром $x_0$ и радиусом $r$ называют множество точек пространства, удовлетворяющее условию $\rho(x, x_0) < r$

Открытый шар радиуса $r$ называют $r$-окрестностью точки $x_0: O_r(x_0)$.
\[
  O_r(x_0) = \{x \in X: \rho(x, x_0) < r\}
\]

Замкнутым шаром радиусом $r$ с центром $x_0$ называют множество точек пространства, удовлетворяющее условию: $\rho(x, x_0) \leq r$

Рассмотрим некоторое множество $M \subset X$.

Точка $x \in M$ называется изолированной точкой множства $M$, если существует окрестность этой точки не содержащая других точек множества $M: O_\epsilon(x)$

Точка $x \in X$ называется предельной точкой множества $M$, если любая окрестность этой точки содержит бесконечное число элементов множества $M$. Предельная точка множества $M$ может принадлежать этому множеству, а может и не принадлежать ему.

Замыканием множества $M$ называется множество, которое получается в результате добавления к нему предельных точек $[M]$. Замыкание $[M]$ состоит из точек трех типов: изолированные точки, предельные точки входящие в $M$, предельные точки не входящие в $M$.

\subsubsection{Сходимость в метрическом пространстве}

Рассмотрим последовательность точек $\{x_n\}_{n=1,2,\dots}\subset X$. Говорят, что $\lim_{n \rightarrow \infty} x_n = x$, если $\forall \epsilon > 0\quad \exists N = N(\epsilon): \rho(x_n, x)<\epsilon\sep\forall n \geq N$, иначе записывают $x_n\xrightarrow[x]{}x$

Это означает, что для любого $\epsilon > 0 \quad O_\epsilon(x)$ содержит все члены последовательности начиная с некоторого номера. Непосредственно из определения предела вытекает:

\be
  \item Единственность предела: если последовательность имеет предел, то он единственный

  \item Если последовательность сходится к некоторому пределу, то любая ее подпоследовательность сходится к этому же пределу.
\ee

Доказательство идентично доказательству для числовых пределов.

\tth[] Для того чтобы точка $x \in X$ была предельной точкой множества M необходимо и достаточно, чтобы существовала последовательность элементов $\{x_n\}\subset M$ такая, что $x_n\neq x\quad x_n\rightarrow x$

\section{Открытые и замкнутые множества в метрическом пространстве}

Множество $M$ метрического пространства $X$ называют замкнутым, если оно содержит все свои предельные точки $[M]=M$. Замкнутые множества совпадают со своими предельными точками.

Точка $x \in M$ называется внутренней точкой этого множества, если существует окрестность этой точки $O_\epsilon(x)\subset M$.

Множество называется открытым, если все его точки является внутренними.

\tth[] Для того, чтобы множество $M\subset X$ было открытым (замкнутым) необходимо и достаточно, чтобы его дополнение для всего пространства $X$ ($X\backslash M$) было замкнутым (открытым).

\btev[] Пусть множество $M$ открытое и $x \in M$ произвольная точка. Так как множество $M$ открытое, то точка $x$ - внутренняя, то есть существует $O_\epsilon(x)\subset M$. Следовательно, эта окрестность не входит в $X \backslash M$, следовательно $X \backslash M$ - замкнутое. Пусть $M$ - замкнутое, возьмем произвольную точку $x \notin M$, отсюда следует что $x\in X\backslash M$ $\Rightarrow$ существует $O_{\epsilon}(X)$, которая не содержит ни одного элемента из $M$. $O_{\epsilon'}(x)\subset X\backslash M$. То есть любая точка $x \notin M$ является внутренней точкой для $X \backslash M$.

\subsubsection{Свойства открытых и замкнутых множеств.}

\be
  \item Пересечение любого числа замкнутых множеств является замкнутым множеством.
  \[
    \bigcap _\alpha \underset {\text{замк.}} {M_\alpha}\text{ - замкнутое.}
  \]

  \item Объединение конечного числа замкнутых множеств является замкнутым множеством.
  \[
    \bigcup _{k = 1}^n \underset {\text{замк.}} {M_k}\text{ - замкнутое.}
  \]

  \item Объединение любого числа открытых множеств является открытым множеством.
  \[
    \bigcup _{\alpha} \underset {\text{откр.}} {M_\alpha}\text{ - открытое.}
  \]

  \item Пересечение конечного числа открытых множеств является открытым множеством.
  \[
    \bigcap _{k = 1}^n \underset {\text{откр.}} {M_k}\text{ - открытое.}
  \]

  \item Замыкание $[M]$ является наименьшим замкнутым множеством содержащем $M$.

  \item Отрытый шар в метрическом пространстве является открытым множеством. Замкнутый шар в метрическом пространстве является замкнутым множеством.
\ee

\btev[ свойства 1] Пусть все $M_\alpha$ - замкнутые, $x$ - произвольная предельная точка $\bigcap _\alpha M_\alpha$ пересечения. Тогда любая ее окрестность содержит бесконечно много точек из каждого $M_\alpha$, то есть является предельной точкой для каждого $M_\alpha$. Все $M_\alpha$ - замкнутые $\Rightarrow$ $x\in M_\alpha$, а тогда и $x \in \bigcap _\alpha M_\alpha$ \etev

\btev[ свойства 2] Пусть все $M_k \quad k=\overline{1,n}$ замкнутые множества, точка $x \notin \bigcup _{k = 1}^n M_k$ $\Rightarrow$ $x \notin M_k$, поэтому $x$ не является предельной точкой ни для одного замкнутого $M_k$ $\Rightarrow \forall M_k \exists \epsilon_k > 0: O_{\epsilon_k}(x)$ содержит не более чем конечное число элементов $M_k$. Возьмем $\epsilon = \min_{k=1,2,\dots,n} \epsilon_k$, Тогда $O_\epsilon(x)$ содержит не более чем конечное число элементов $\bigcup _{k = 1}^n M_k \Rightarrow $ она не является предельной точкой множества $ \Rightarrow  \bigcup _{k = 1}^n M_k$ - замкнутое \etev

\btev[ свойства 3]
\[
  \bigcup _\alpha \underset{\text{откр.}}{M_\alpha} = \bigcup _\alpha\bb[(]X\backslash\underset{\text{замк.}}{F_\alpha} \bb[)] = X\backslash \underbrace{\bigcap _\alpha F_\alpha}_{\text{замк.}}\text{ - открытое}
\]
\etev

\subsubsection{Примеры отрытых и замкнутых множеств}
\be
  \item $R^1$ - прямая, $(a, b)$ - открытое, $[a, b]$ - замкнутое.

  \item $C[a, b]\quad |x(t)| < K \quad \forall t \in [a, b]$

  $x(t) \in C[a, b]$ - множество функций с центром 0 и радиусом K.

  $y(t) \in C[a, b]$ - непрерывные фиксированные функции

  $x(t) < y(t) \quad \forall t \in [a, b]$ - открытое множество

  $x(t) \leq y(t) \quad \forall t \in [a, b]$ - замкнутое множество
\ee

В произвольном метрическом пространстве структура открытых и замкнутых множеств весьма сложна. Исчерпывающие описание структуры открытого и замкнутого множества можно дать только на одной числовой прямой в $R^1$.

\tth[] Всякое открытое множество на действительной оси представляет собой объединение конечного или счетного числа попарно не пересекающихся интервалов.

\textbf{Лекция 3.}

\section{Плотные подмножества. Сепарабельные пространства}

Рассмотрим два множества $A, B\subset X$.

Множество $A$ называется плотным в $B$, если $B\subset[A]$.

Множество $A$ называется всюду плотным (в пространстве $X$) если его замыкание
совпадает с пространством $[A]=X$.

Если множество A всюду плотно в пространстве X, то:

\begin{enumerate}
  \item Любая окрестность любой точки пространства содержит хотя бы одну точку
  множества A.

  \item Для любой точки пространства $x \in X$  существует последовательность
  элементов ${x_n}\in A, x_n \rightarrow x$.
\end{enumerate}

Например: на действительной оси всюду плотно множество рациональных чисел.

Метрическое пространство называется сепарабельным если в нем существует счетное
всюду плотное множество.

Множество называется счетным если между этим множеством и множеством
натуральных чисел можно установить взаимно однозначное соответствие.

Укажем счетные, всюду плотные множества в некоторых метрических пространствах:

\begin{enumerate}
  \item На действительной оси всюду плотно множество рациональных чисел $R_n^1$.

  \item В арифметических пространствах $R_n^p, R_\infty^n$ всюду плотно
  множество векторов с рациональными координатами.

  \item В пространстве непрерывных функций С[a,b] всюду плотно множество
  многочленов с рациональными коэффициентами.

  \item В пространстве лебеговых функций $L_p[a,b]$ счетным, всюду плотным
  множеством является множество многочленов с рациональными коэффициентами (в
  последствии мы покажем что множество непрерывных функций всюду плотно в
  $L_p[a,b]$)

  \item В пространстве последовательностей $l_p$ счетным, всюду плотным
  множеством является множество последовательностей с рациональными членами в
  которых только конечное число членов отличное от нуля.
\end{enumerate}

Все перечисленные пространства являются сепарабельнми.

Ограниченное пространство $l_{\inf}$ не является сепарабельным.

\section{Непрерывные отображения метрических пространств.
Гомеоморфизм и изометрия.}

Рассмотрим два метрических пространства X, Y
\[\rho_X(x_1,x_2)\sep x_1,x_2\in X \]
\[\rho_Y(y_1, y_2)\sep y_1,y_2\in Y \]
\[ f: X \rightarrow Y\]

Вместо термина "отображение" используется эквивалентные термины "функция" и
"оператор". В разных разделах функционального анализа сложились свои традиции
использования этого термина.

Функционалом называется числовая функция определенная на некотором метрическом
пространстве. Иначе функционал это отображение некоторого метрического
пространства на действительную ось (или на комплексную плоскость).

Отображение $f: X\rightarrow Y$, называется непрерывным в точке $x_o\in X$ если
$ \forall \epsilon > 0 \quad \exists \delta = \delta (\epsilon) > 0:
\rho_y(f(x), f(x_0)) < \epsilon \quad \text{при} \quad \rho_x(x,x_0)<\delta $

Эквивалентное определение непрерывности на языке последовательностей:
$f: X \rightarrow Y$ называется непрерывным в точке $x_0 \in X$, если
$\forall \{x_n\}\subset X: x_n\xrightarrow[X]{} x_0$
$f(x_n) \xrightarrow[Y]{} f(x_0)$

Доказательство этих определений непрерывности дословно воспроизводят
соответствующие доказательства для функций соответствующих переменных

Пусть f взаимно однозначное отображение пространства X на Y
$f: X \rightarrow Y$, тогда существует обратное отображение
$f^-1: Y\rightarrow X$, отображение f называется гомеоморфизмом или гомеоморфным
отображением, если оно взаимно однозначно и взаимно непрерывно.

Отображение $f: X \rightarrow Y$ называется изометрией или изометрическим
отображением, если оно взаимно однозначное и оно сохраняет расстояние между
точками.
\[ \rho_Y(f(x_1,x_2))=f_X(x_1,x_2)\forall x_1,x_2\in X\]

Изометрия является частным случаем гомеоморфизма, два пространства между
которыми можно установить гомеоморфизм (изометрию) называются гомеоморфными
(изометрическими). Если два пространства изометрические, то метрические
соотношения между их элементами идентичны, различаться может только природа
этих элементов, что с точки зрения теории метрических пространств несущественно.

\section{Полные метрические пространства}

В анализе существенную роль играет свойство полноты числовой прямой, то есть
что любая фундаментальная последовательность действительных чисел является
сходящейся.

Последовательность называется фундаментальной, либо удовлетворяет условию Коши,
если расстояние между двумя элементами последовательности становится сколь
угодно малым с некоторого номера.

Рассмотрим последовательность элементов $\{x_n\} \in X$, последовательность
называется фундаментальной если она удовлетворяет условию Коши, то есть
$ \forall \epsilon > 0 \exists N = N(\epsilon): \rho(x_n, x_m) < \epsilon $
$ \forall n, m \geq N $

Любая сходящаяся последовательность является фундаментальной.

\btev[]

Пусть $x_n \rightarrow x$, тогда
$\forall \epsilon >0 \exists N: \rho(x_n, x) < \epsilon, \forall n \geq N$
тогда согласно аксиоме треугольника
$\rho(x_n, x_m) \leq \rho(x_n, x)+\rho(x, x_m) < 2\epsilon, \forall n, m \geq N$,
что и доказывает фундаментальность последовательности. \etev

Обратное утверждение в произвольном метрическом пространстве неверно.
Например \[ X = \{x \in R^1: a < x < b\}\quad x_n = a + \cfrac1n\]

\tdef Метрическое пространство называется полным если любая
фундаментальная последовательность его элементов является сходящейся.

Полными являются следующие метрические пространства:

\begin{enumerate}
  \item $R^1$
  \item $R_p^n, R_\infty^n$, так как сходимость в этих пространствах означает
  покоординатную сходимость, а пространство $R^1$ является полным
  \item $l_p, l_\infty$ - пространство последовательностей
  \item $L_p[a,b]$
  \item $L_\infty$
  \item $C[a,b]$
\end{enumerate}

\tth[] Пространство непрерывных функций $C[a,b]$ является полным, а
сходимость в этом пространстве означает равномерную сходимость.

\btev[]

Рассмотрим произвольную фундаментальную последовательность непрерывных функций:
${x_n(t) \subset C[a, b]}$, тогда:
\[  \forall \epsilon > 0 \quad \exists N: \rho(x_n, x_m) =
\max_{a \leq t \leq b} |x_n(t) - x_m(t)| < \epsilon, \forall n, m \geq N \]
легко можно узнать известный критерий Коши равномерной сходимости
функциональной последовательности, в соответствии с этим критерием
\[ x_n(t) \xrightrightarrows{[a,b]} x(t)\]
причем функция x(t) непрерывная $x(t)\in C[a,b]$

\textbf{Лекция 4.}

Докажем теперь что $x_n(t)$ сходится к $x(t)$, $x_n(t) \rightarrow x(t)$
так как последовательность сходиться равномерно, то
\[  \forall \epsilon > 0 \quad
\exists N = N(\epsilon): |x_n(t) - x(t)| < \epsilon \sep
\forall t \in [a,b] \sep \forall n \geq N\]
отсюда сразу вытекает:
\[ \rho(x_n, x) = \max_{a \leq t \leq b} |x_n(t) - x(t)| < \epsilon \sep
\forall n \geq N \lim_{n \rightarrow \infty} \rho(x_n,x) = 0\]

Пополнение метрического пространства: оказывается что любое неполное
метрическое пространство можно включить в полное.

Пространтсво $\bar X$ называется пополнением X, если $X \subset \bar{X}$ и X
всюду плотно в $\bar X$: $[X] = \bar X$


Например: действительная ось это пополнение пространства действительных чисел.
Пополнением пространства $X = \{x \in R^1: a < x < b\}$ является
$\bar{X} = \{x \in R^1: a \leq x \leq b\}$

\tth[ о пополнении] Любое неполное метрическое пространство можно
включить в полное в качестве всюду плотного подмножества, причем такое
пополнение единственное с точностью до изометрии, которая оставляет неподвижной
точки неполного пространства.

На действительной оси имеет место лемма о вложенных отрезках. Пусть дана
последовательность вложенных отрезков $[a_n, b_n]$ $a_1 \leq a_2 \leq \dots \leq
a_n \leq \dots \leq b_n \leq b_{n-1} \leq \dots \leq b_1 $, тогда пересечение
этих отрезков является непустым множеством и пересечение состоит из одной
точки, при условии что длина отрезка стремится к нулю.
\[ \lim_{n \rightarrow \infty}(b_n - a_n) = 0\]

Аналогом этого свойства для произвольного полного метрического пространства
является теорема о вложенных шарах.

\tth[] Для того чтобы метрическое пространство $X$ было полным
необходимо и достаточно чтобы пересечение любой последовательности вложенных
замкнутых шаров состояло из единственной точки.

\textbf{Доказательство необходимости.}

Пусть метрическое пространство является полным, то есть любая фундаментальная
последовательность его элементов сходится. Рассмотрим последовательность
вложенных замкнутых шаров
\[ B_{1} \supset B_{2} \supset \dots \supset B_{n} \supset \dots\]
c центрами $x_{n}$ и радиусами $r_{n}$, последовательность $\{x_n\}$ является
фундаментальным потому что $\rho(x_n, x_m) < r_n\sep m > n\sep r_n\rightarrow_0$

Так как пространство полное, то $x_{n}$ сходится к некоторой точке x,
причем точка x принадлежит пересечению каждого из шаров $x \in \bigcap _{n = 1}^{\infty} B_{n}$ (так как точка x
предельная точка для каждого шара, а все шары замкнуты).

Для доказательства единственности предположим что существуют две точки $x, y$
принадлежащие пересечению шаров
$x, y \in \bigcap _{n = 1}^{\infty} B_{n}, \\ \rho(x, y) = \epsilon > 0$, так как
$r_n \xrightarrow[n \rightarrow \infty]{} 0$ то для указанного
$\epsilon \sep \exists N: r_N < \cfrac\epsilon2$
то есть точки x, у не могут одновременно принадлежать шару $B_N$, таким образом
получили противоречие и доказали единственность.

\textbf{Доказательство достаточности.} Предположим что пресечение любой
последовательности вложенных замкнутых шаров состоит из единственной точки,
докажем что в этом случае пространство является полным. Для этого рассмотрим
произвольную фундаментальную последовательность $\{x_{n}\} \subset X$, так как
последовательность фундаментальная, то
$\forall \epsilon > 0 \exists N: \rho(x_n, x_N) < \epsilon\sep \forall n > N$,
отсюда следует что можно указать последовательность номеров
$n_1 < n_2 <\dots < n_k < \dots $, обладающую следующим свойствами:

\[\rho(x_n, x_{n_1}) < \cfrac{1}{2}\sep \forall n > n_{1} \]
\[\rho(x_n, x_{n_2}) < \cfrac{1}{2^{2}}\sep \forall n > n_{2} \]
\[\dots \]
\[\rho(x_n, x_{n_k}) < \cfrac{1}{2^{k}}\sep \forall n > n_{k} \]


Рассмотрим последовательности замкнутых шаров $B_k$ с центром в точке $x_{n_k}$ и
радиусом $r_n = \cfrac{1}{2^{k}}$, по условию теоремы эта последовательность
шаров имеет не пустое пересечение, состоящее из одной точки x, таким образом
последовательность $x_n$ содержит сходящуюся подпоследовательность
$x_{n_k} \rightarrow x$, а если фундаментальная последовательность содержит
сходящуюся подпоследовательность, то и сама последовательность сходится, что и
доказывает полноту пространства.

\section{Принцип сжимающих отображений}

Рассмотрим отображение A метрического пространства X в себя.
\[ A: X \rightarrow X\sep \forall x \rightarrow Ax\]

Отображение A называется отображением сжатия или сжимающим отображением, если
\[ \exists \alpha \in [0, 1]: \rho(Ax, Ay) \leq \alpha \rho(x,y)\sep{}
\forall x, y \in X\]

Натуральные степени отображения $A$ определяются следующим образом:
\[ A^nx = A^{n-1}Ax\sep n=2, 3, \dots\]
легко видеть:
\[ \rho(A^nx, A^ny) \leq \alpha^n \rho(x, y)\sep \forall x, y \in X\]

Точка x пространства называется неподвижной точкой отображения, если $Ax = x$.

\textbf{Принцип сжимающих отображений (теорема о неподвижной точке).} Любое
отображение сжатия в полном метрическом пространстве имеет единственную
неподвижную точку. Иными словами, если $Ax$ отображение сжатия, то уравнение
$Ax = x$ имеет единственное решение в метрическом пространстве $X$.

\btev[]

Доказательство конструктивное, то есть дает алгоритм построения этой
неподвижной точки.

Возьмем произвольную точку $x_0 \in X$, построим последовательность:
\[ x_1 = Ax_0 \]
\[ x_2 = Ax_1 = A^{2}x_{0} \]
\[ \dots \]
\[ x_n = Ax_{n-1} = A^{n}x_{0} \]

Докажем что эта последовательность является фундаментальной, причем будем
считать что $m > n$


\[\rho(x_n, x_m)=\rho(A^nx_o, A^n\underbrace{A^{m-n}x_0}_{x_{m-n}})
  \leq \alpha^n \rho(x_0, x_{m-n}) \]
\[\rho(x_n, x_m) \leq \alpha^n (\rho(x_0, x_1) + \rho(x_1, x_2) + \dots +
  \rho(x_{m-n-1}, x_{m-n})) =
\alpha^n (\rho(x_0, x_1) + \rho(Ax_0, Ax_1) + \dots + \rho(A^{m-n-1}x_0, A^{m-n-1}x_1)) \leq
\alpha^n (1 + \alpha + \dots + \alpha^{m-n-1})\rho(x_0,x_1) \leq
\alpha^n (1 + \alpha + \dots) \rho(x_0, x_1) =
  \cfrac{\alpha^n}{1 - \alpha} \rho(x_0, x_1) \rightarrow 0\]


Что и доказывает фундаментальность последовательности $x_{n}$ \etev

Так как пространство является полным то последовательность
$x_{n} \rightarrow x \in X$, то есть $\forall \epsilon > 0 \quad
\exists N: \rho(x_{n}, x) < \epsilon\sep \forall n \geq N$

\textbf{Лекция 5}


\[\rho(Ax_{n},Ax)\leq\alpha\rho(x_{n},x_{n})<\alpha\epsilon\sep
\forall n \geq N \]
\[\lim_{n\rightarrow \infty}\underbrace{Ax_{n}}_{x_{n+1}}=Ax=x \]
\[\lim_{n\rightarrow \infty}{x_{n+1}}=x\]

Ax - неподвижная точка

Единственность неподвижной точки докажем от противного, пусть у нас две точки $x, y:\quad Ax=x\sep Ay=y$, тогда:
\[ \underbrace{\rho (A_{x}, A_{y})}_{\rho(x, y)}\leq\alpha\rho(x, y)\]

Поэтому $(1-\alpha)\rho(x,y) \leq 0\sep \rho(x, y) \leq 0\sep x=y$

\tth[ (обобщение принципа сжимающих отображений)] Пусть $A$ непрерывное отображение пространства $X$ в себя $A: X\rightarrow X$, такое что некоторая натуральная степень этого отображения является отображением сжатия, тогда отображение $A$ имеет единственную неподвижную точку.

\btev[] Так как $B$ отображение сжатия, то согласно принципу отображения сжатия:
\[ \lim_{n\rightarrow \infty}B^{n}x_{0}=x=Bx\sep\forall x_{0}\in X\]

Тогда используя непрерывность отображения $A$ и тот факт что $B=A^{m}$ получаем:
\[ x = \lim_{n\rightarrow \infty}B^{n}Ax =
\lim_{n\rightarrow \infty}AB^{n}x =
A\lim_{n\rightarrow \infty}B^{n}x = Ax\]

Неподвижная точка отображения $A$ является неподвижной и для отображения $B$, а для неподвижгой точки отображения $B$ единственность доказана.

\section{Простейшие применения принципа сжимающих отображений}

\subsubsection{Решения уравнения $x=f(x)$}

Рассмотрим функцию $f(x)$, удовлетворяющую на отрезке $[a, b]$ условию Липшица с константой K < 1:
\[ |f(x_1)-f(x_2)|\leq K|x_1-x_2|\sep\forall x_1, x_2 \in [a, b], 0<K<1\]

Эта функция задает отображение метрического пространства $[a, b]$ в себя: $f:[a,b] \rightarrow[a, b]$. Выясним при каких условиях это отображение является отображением сжатия. Согласно принципу сжимающих отображений уравнение $x=f(x)$ имеет единственное решение и это решение может быть получено методом последовательных приближений $x_{n+1}=f(x_n), x_{n+1} \in [a, b]$, а в качестве начального приближения возьмем $x_0 \in [a, b]$.

\resizebox{\pltwdth}{!}{\begin{tikzpicture}[>=stealth]
%\begin{axis}[
  %minor tick num = 0,
  %xmin=0,   xmax=5,
  %ymin=0,   ymax=5,
%]
%\addplot[blue] {x};
%\addplot[red] {(x-2)^2+1};
%\end{axis}
\draw[->] (-0.1,0) -- (4,0) node[below] {$x$}; % Ох
\draw[->] (0,-0.1) -- (0,4) node[right] {$y$}; % Оу
\draw[thick, blue, domain=0:4] plot (\x, {\x});
\draw[thick, blue, domain=0.5:3.5] plot (\x, {(\x-1)^2/2+0.5});
\draw[dotted] (0.75,0.75) -- (0.75,0) node[below] {$a$};
\draw[dotted] (3,3) -- (3,0) node[below] {$b$};
\draw[dotted] (2.75, 2) -- (2.75,0) node[below] {$x_0$};
\draw[dotted] (2.03125, 2.03125) -- (2.03125, 0) node[below] {$x_1$};
\draw[dotted] (2.03125, 2.03125) -- (2.75, 2.03125);
\draw[dotted] (1.03173828125, 1.03173828125) -- (1.03173828125, 0) node[below] {$x_2$};
\draw[dotted] (1.03173828125, 1.03173828125) -- (2.03125, 1.03173828125);
\end{tikzpicture}}

Решение уравнения $F(x)=0$ на отрезке $[a,b]$, причем будем считать $F(a)<0,\; F(b)>0$. Функция $F(x)$ имеет непрерывную производную на отрезке $[a,b]$, это легко сходится к предыдущему случаю, рассмотрим функцию:
\[ f(x)=x-\lambda F'(x)\sep \lambda \neq 0 \sep x=f(x)\]
\[ 1 - \lambda K_2 \leq f'(x) = 1 - \lambda F'(x) \leq 1 - \lambda K_1\]

При достаточно малых $\lambda$, f является отображением сжатия, так как непрерывно дифференцируемая функция заведомо удовлетворяет условию Липшеца, поэтому при достаточно малых $\lambda$ при решении уравнения $F(x)=0$ можно использовать метод последовательных приближений.

\subsubsection{Линейное отображение $R^n \rightarrow R^N$}

Точка $y=Ax\sep y=(y_1,\dots,y_n) \in R^n\sep x=(x_1,\dots,x_n) \in R^n$
\[ y_i=\sum_{j=1}^n a_{i,j}x_j+b_i\]

Выясним при каких условиях отображение $A$ является сжимающим. Эти условия зависит от выбора метрики в n-мерном арифметическом пространстве. Установить это условие для пространств $R_1^n,R_2^n,R_\infty^n$. Нарисовать $\epsilon$ окрестность точки для пространств $R_1^2,R_2^2,R_\infty^2$

\section{Применение принципа сжимающих отображений к решению интегральных уравнений.}

\tdef Интегральное уравнение называется интегральным уравнением первого рода если неизвестная функция входит в него только под знаком интеграла.

\tdef Интегральное уравнение называется интегральным уравнением второго рода если неизвестная функция входит в него как под знаком интеграла так и вне его.

Интегральное уравнение Фредгольма второго рода:
\[ f(x)-\lambda\int_a^bK(x,y)f(y)dy=\phi(x)\]

Здесь f(x) неизвестная функция, K(x,y) - ядро интегрального уравнения, Это является известной функцией, которая предполагает ограничения:
\[ |K(x,y)|\geq M\sep\forall x,y\in[a,b]\]
$\phi(x)$ - заданная функция, непрерывная на отрезке $[a,b]$, $\lambda$ - параметр.

Если $\phi(x) = 0$, то уравнение называется однородным, в противном случае неоднородным.

Рассмотрим отображение $Af(x)$, которое действует на функцию $f$:
\[ Af(x)=\phi(x)+\lambda\int_a^bK(x,y)f(y)dy\]

\[ A: C[a, b]\rightarrow C[a,b]\sep f,\phi \in C[a,b]\]

Выясним при каком условии отображение $A$ будет отображением сжатия. $f_1, f_2 \in C[a,b]$

\[ \rho(Af_1,Af_2)=\max_{a\geq x \geq b}|Af_1(x)-Af_2(x)|=\]
\[ =\max_{a\geq x \geq b} |\lambda\int_a^bK(x, y)(f_1(y)-f_2(y))dy|\leq\]
\[ |\lambda|M(b-a)\max_{a\geq x \geq b}|f_1(x)-f_2(x)|=\]
\[ \underbrace{|\lambda|M(b-a)}_{<1}\rho(f_1,f_2) \]

Интегральное уравнение Фредгольма имеет единственное решение, которое может быть получено методом последовательных приближений.

\[ f_{n+1}(x)=\phi(x)+\lambda\int_a^bK(x,y)f_n(y)dy\]
\[ f_0^{(x)} \in C[a,b]\]

\textbf{Лекция 6}

\subsubsection{Нелинейное интегральное уравнение второго рода.}

\[ f(x)-\lambda\int_a^bK(x,y,f(x))dy=\phi(x)\]

Ядро интегрального уравнения $K(x,y,z)$ удовлетворяет условию Липшеца по своему 'функциональному' аргументу

\[|K(x,y,z_1)-K(x,y,z_2)|\leq M|z_1-z_2|\sep\forall z_1,z_2\sep M>0\]


$f(x), \phi(x)$ непрерывные функции на отрезке $[a,b]$. Рассмотрим оператор $A:C[a,b]\rightarrow C[a,b]$

\[ Af(x) = \phi(x)+\lambda\int_a^bK(x,y,f(y))dy \]

Выясним в каком случае отображение A будет сжимающим.

\[ \rho(Af_1, Af_2) = \max_{a\leq x\leq b}|Af_1(x)-Af_2(x)|=\\=
\max_{a\leq x\leq b}\bb[|]\lambda\int_a^b(K(x,y,f_1(y))-K(x,y,f_2(y)))dy\bb[|]\leq\\\leq
\max_{a\leq x\leq b}\bb[(]\bb[|]\lambda\bb[|](b-a)M|f_1(x)-f_2(x)|\bb[)] = \\
= \underbrace{|\lambda|(b-a)M}_{<1}\rho(f_1,f_2)
\]
При достаточно малых $\lambda$ это отображение сжимающие.

\subsubsection{Интегральное уравнение Вольтрры второго рода}

Интегральное уравнение Вольтрры второго рода отличается от интегрального уравнения Фредгольма переменным верхним пределом интегрирования.

\[ f(x)-\lambda \int_a^xK(x,y)f(y)dy=\phi(x)\]

Неизвестная функция $f(x)$ предполагается непрерывной на $[a,b]$, заданная правая часть непрерывна на $[a,b]$, ядро считается ограниченной функцией в квадрате: $|K(x,y)|\leq M\sep\forall x,y\in[a,b]\sep A:C[a,b]\rightarrow C[a,b]$.

Рассмотрим отображение:

\[ Af(x)=\phi(x)+\lambda\int_a^xK(x,y)f(y)dy \]

Покажем что некоторая степерь отображения $A$ ($A^n$), является отображением сжатия:
\[\rho(Af_1,Af_2)=\max_{a\leq x\leq b}\bb[|]\lambda\int_a^xK(a,b)(f_1(y)-f_2(y))dy\bb[|]\leq\\
\leq |\lambda|M(x-a)\rho(f_1,f_2)
\]

Предположим, что $\rho(A^mf_1,A^mf_2)\leq |\lambda|^m\cfrac{M^m}{m!}(x-a)\rho(f_1,f_2)$

Используя это предположение, оценим
\[
\rho(A^{m+1}f_1,A^{m+1}f_2)=\\
=\max_{a\leq x\leq b}\bb[|]\lambda\int_a^xK(x,y)\cdot(A^mf_1(y)-A^mf_2(y))dy\bb[|]\leq\\
\leq|\lambda|^{m+1}\cfrac{M^{m+1}}{(m+1)!}(x-a)^{m+1}\rho(f_1,f_2)
\]

В силу метода математической индукции справедлива оценка для любого натурального $n$, таким образом для произвольного значения $\lambda$ можно указать такое $m$, что $|\lambda|^m\cfrac{M^m}{m!}(b-a)^m<1$. Таким образом для отображения $A$ существует при любом $\lambda$ такое натуральное значения $m$, при котором $A^m$ является отображением сжатия, тогда можно применить обобщение принципа сжимающего отображения, которое  указывает на существование единственной неподвижной точки оператора $A$, эта неподвижная точка и будет решением интегрального уравнения Вольтрра второго рода.

Для нахождения решения можно применить инерционный процесс.

\textbf{Замечание!} Интегральное уравнение Вольтрры второго рода можно рассматривать как частный случай интегрального уравнения Фрильдгольма, если доопределить ядро следующим образом $K(x,y)=0,\quad y>x$. Но отображение $A$ является сжимающим только при достаточно малых $\lambda$, а для уравнения Вольтрры мы показали, что $A^m$ является сжимающим отображением при произвольных $\lambda$.

\section{Компактность в метрических пространствах}

Из анализа известно теорема Больцано-Веерштраса: из любой ограниченной числовой последовательности можно извлечь сходящуюся подпоследовательность. Обобщение этого факта привело к понятию компактности в метрических пространствах.

Рассмотрим некоторое множество $M\subset X$

Множество называется предкомпактным, если из любой последовательности его элементов ${x_n}\subset M$ можно извлечь сходящуюся подпоследовательность $x_{n_k}\rightarrow x\in X$

Точка $x$ может принадлежать множеству M, а может и не принадлежать ему.

Множество $М$ называется компактным если пределы всех указных подпоследовательностей принадлежат этому множеству.

Множество является компактным тогда и только тогда когда оно предкомпактно и замкнуто. Метрическое пространство $X$ называется компактным если из любой последовательности его элементов можно извлечь сходящуюся подпоследовательность.

Компактное метрическое пространство всегда является полным.

Числовая ось ($R^1$) не является компактным пространством.

Метрическое пространство ($[a,b]$) является компактным пространством.

Оказывается что в $n$-мерном арифметическом ($R_p^n$) пространстве предкомпактность эквивалентно его ограниченности, так как сходимость в этом пространстве является покоординатной сходимостью.

\tth[] Любое ограниченное множество $M\subset R_p^n$ предкомпактно, а ограниченное замкнутое множество компактно

\btev[] Пусть $M$ ограниченное множество. Рассмотрим последовательность его элементов: $x^k=(x_1^k,\dots,x_n^k)\in M$.
Числовая последовательность $x_1^k$ ограниченная, поэтому из нее можно извлечь сходящуюся подпоследовательность $x_1^k$, которая сходится к некоторой точке $x_1^0$.
Рассмотрим $x_2^k$ из нее можно извлечь сходящуюся подпоследовательность $x_2^k$, которая сходится к некоторой точке $x_2^0$, а из последовательности $x_1^{k_2}$ можно извлечь $x_1^{k_2}\rightarrow x_1^0$. Продолжая этот процесс получим последовательности координат $x_i^{k_n}\rightarrow x_i^0$.

Рассмотренный метод доказательства называется диагональным процессом.
\[ x_1^{k_1} \rightarrow x_1^0 \]
\[
\begin{matrix}
x_1^{k_2} \rightarrow x_1^0 & \dots & x_1^{k_n} \rightarrow x_1^0 \\
x_2^{k_2} \rightarrow x_2^0 & \dots & x_2^{k_n} \rightarrow x_2^0 \\
\dots\\
& & x_n^{k_n} \rightarrow x_n^0 \\
\end{matrix}
\]

$x^{k_n}=(x_1^{k_n},\dots,x_n^{k_n})\sep x^{k_n}\rightarrow x^0 \in R_p^n$

Так как сходимость в $R_n^p$ означает покоординатную сходимость.

\subsubsection{Эквивалентная терминология для описания компактности множества}

Вариант I: предкомпактное, компактное.

Вариант II: компактное в пространстве, компактное в себе.

Вариант III: компактное, бикомпактное.

\textbf{Лекция 7}

В пространстве непрерывной на $C[a,b]$ существуют замкнутые ограниченные множества. Таким множеством является, например, единичный шар. Для доказательства некомпактности единичного шара можно рассмотреть последовательность $x_n(t)=sin2^{n-1}\pi t$. Для простоты рассмотрим пространство на отрезке $C[0,1]$.

Докажем что расстояние между любыми двумя элементами этой последовательности больше либо равно 1, пусть $m>n$:
\[
\rho(x_n,x_m) = \max_{0\leq t \leq 1} |x_n(t)-x_m(t)| = \max_{0\leq t \leq 1} |sin 2^{n-1}\pi t-sin 2^{m-1}\pi t|\geq 1\\
x_n|_{t=2^{-n}}=1, x_m|_{t=2^{-n}}=0
\]

Эта последовательность принадлежит единичному шару, но из нее нельзя вывести сходящуюся подпоследовательности.

\section{Критерий компактности множеств в метрических пространствах}

Рассмотрим некоторое множество $M\subset X$, множество $N_{\epsilon} \subset X$ называется $\epsilon$ сетью для множества $M$, если $\forall x \in M \exists x_{\epsilon} \in N_{\epsilon}: \rho(x,x_{\epsilon})<\epsilon$. $\epsilon$-сетью называется конечной если она состоит из конечного числа элементов.

\tth[ Хауздорфа (критерий предкомпактности множества)] Для того чтобы множество $M\subset X$ было предкомпактным необходимо, а в случае полного пространства и достаточно, чтобы для этого множества при произвольном положительном $\epsilon$ существовала конечная $\epsilon$-сеть.

\btev[ необходимости]
  Пусть множество $M$ является предкомпактным. Возьмем произвольный элемент $x_1 \in M\sep \epsilon > 0$.
  Построим $O_{\epsilon}(x_1)$. Если $M \subset O_{\epsilon}(x_1)$, то множество состоящие из одного элемента $\{x_1\}$ и есть конечная $\epsilon$-сеть, в противном случае найдется элемент $x_2 \in M\sep x_2 \not\in O_\epsilon(x_1)$, если $M\subset O_\epsilon(x_1)\cup O_\epsilon(x_2)$, то $\{x_1,x_2\}$ и есть конечная $\epsilon$-сеть. Продолжая этот процесс получим такой набор точек $x_1, x_2,\dots,x_n$ при котором $\rho(x_i,x_j)\geq \epsilon\sep \forall i,j=1,2,\dots,n\sep i\neq j$.

  Этот процесс является конечным, в противном случае (в случае бесконечного процесса) была бы получена последовательность ${x_n}\subset M\sep n=1,2,\dots$, расстояние между любыми двумя точками было бы больше $\epsilon$. Из такой последовательности невозможно выделить сходящуюся подпоследовательность, а это противоречит предположению о предкомпактности множества.
\etev

\btev[ достаточности]
  Пусть для любого положительного $\epsilon$ существует конечная $\epsilon$-сеть множества $M$. Рассмотрим последовательность $\epsilon_n \rightarrow 0$. Для каждого $\epsilon_n$ существует конечная $\epsilon_n$ сеть.

  Рассмотрим произвольную последовательность $A=\{a_{1},\dots,a_n,\dots\}\subset M$. Возьмем $\epsilon_1$ и построим шары радиусом $\epsilon_1$ с центром в точке $\epsilon$-сети, число таких шаров конечное. Один из этих шаров $B_1$ будет содержать бесконечное число членов последовательности выделим из последовательности $A$ подпоследовательность $A_1\subset B_1$.

  Берем $\epsilon_2$ и строим шары радиуса $\epsilon_2$ с центрами в точках $\epsilon_2$-сети. Один из этих шаров будет содержать бесконечное число членов последовательностей $A_1$. Обозначим $A_2$ подпоследовательность последовательности $A_1$, содержащуюся в шаре $B_2$: $A_2\subset A_1\sep A_2\subset B_2$.

  Продолжая этот процесс получим вложенные последовательности $A\supset A_1\supset A_2\supset \dots\supset A_n\supset \dots$. Причем $A_n$ внутри шара $B_n$ радиуса $\epsilon_n$. Построим последовательность различных $\{x_n\}$ следующим образом: $x_1 \in A_1, x_2 \in A_1, x_2 \not\in A_1, x_2 \neq x_1$ и рассмотрим точки $x_n, x_{x+k} \in B_n$ тогда $\rho(x_n,x_{x+k})\leq 2 \epsilon_n \rightarrow 0$. Последовательность $\{x_n\}$ является фундаментальной, тогда в силу полноты оно сходится в этом пространстве. Таким образом $\{x_n\}$ это сходящаяся подпоследовательность последовательности $A$. Следовательно множество $M$ компактно и теорема доказана.
\etev

\textbf{Следвие} Для того чтобы множество $M$ полного метрического пространства $X$ было предкомпактным достаточно чтобы для этого множества существовала предкомпактная $\epsilon$-сеть.

\btev[] Пусть для произвольного $\epsilon$ существует предкомпактная $\epsilon$-сеть множества $M$. Тогда по теореме Хауздорфа для множества $N_\epsilon \subset M$ существует конечная $\epsilon$-сеть $N_0 \subset X$, тогда для $\forall x \in M \exists x_\epsilon \in N_\epsilon: \rho(x,x_\epsilon) < \epsilon$, а для указанного $x_\epsilon \exists x_0\in N_0: \rho(x_\epsilon,x_0)<\epsilon$. Согласно неравенству треугольника $\rho(x,x_0)\leq \rho(x,x_\epsilon) + \rho(x_\epsilon,x_0) \leq 2\epsilon$. Значит $N_0$ конечная $\epsilon$-сеть для множества $M$. тогда по теореме Хауздорфа пространство предкомпактно. \etev

Следствиями теоремы являются также два важных свойства компактных множеств:

\be
  \item Компактное метрическое пространство сепарабельно.
  \item Предкомпактное множество любого метрического пространства ограничено.
\ee

\btev[ свойства 1] Пусть метрическое пространство $X$ является компактным, тогда для него существует при произвольном $\epsilon$ конечная $\epsilon$-сеть. Рассмотрим $\epsilon_n \rightarrow 0$. Для каждого $\epsilon$ существует конечная $\epsilon$-сеть $N_n$ метрического пространства $M$. $N=\bigcup _{n_1}^{\infty} N_n$ образует счетное всюду плотное множество, что и доказывает его сепарабельность. \etev

\btev[ свойства 2] Рассмотрим предкомпактное множество $M$ метрического пространства $X$, согласно теореме Хауздорфа для этого множества существует конечная 1-сеть $N_1$ тогда для любого $x \in M\quad \exists x_n \in N_1: \rho(x, x_n) \leq 1$. Обозначим $d=\max_{n=1,2,\dots,m}\rho(x_i,x)$, где $m$ число элементов сети. Тогда $\rho(x, x_i) \leq \rho(x, x_n)+\rho(x_n, x_i) \leq 1+d$. Что и доказывает неограниченность множества. \etev

\subsubsection{Конечное покрытие компактного множества}

Замкнутый отрезок числовой оси является компактным множеством. Известно из любого покрытия замкнутого отрезка интервалов можно выбрать конечное подпокрытие (лемма Гейне-Бореля). Этот результат обобщается на компактном множестве в произвольных метрических пространств.

Система отрытых множеств $\{G_\alpha\}$ образует покрытие множества $M$, если любой элемент множества $M$ принадлежит хотя бы одному из $G_\alpha$.

\tth[] Для того чтобы замкнутое множество $M$ метрического пространства $X$ было компактным необходимо и достаточно чтобы из любого его покрытия можно было выделить конечное подпокрытие.

\section{Критерий компактности множеств в пространстве непрерывных функций}

Непосредственно использование теоремы Хауздорфа в произвольных метрических пространствах затруднительно, поэтому большую роль играют компактности множеств в конкретных метрических пространствах. Теорема Арцела представляет собой критерий предкомпактности множества в пространстве непрерывных функций $C[a,b]$.

\textbf{Лекция 8}

Рассмотрим множество $M\subset C[a,b]$. Множество $M$ называется равномерно ограниченным, если существует постоянная $K>0$, такая что для $\forall x(t)\in M \quad |x(t)|\leq K \forall t \in [a,b]$.

Множество $M$ называется равномерно непрерывным, если для $\forall \epsilon > 0 \exists \delta = \delta(\epsilon)>0:\forall x(t)\in M\quad \forall t_1,t_2\in [a,b]$ удовлетворяет неравенству $|t_1-t_2|<\delta \quad |x(t_1)-x(t_2)|<\epsilon$

\tth[ Арцела] Для того ч тобы множество $M\in C[a,b]$ было предкомпактным необходимо и достаточно, чтобы оно было равномерно ограниченная и равностепенно непрерывно.

Пример. Рассмотрим множество функций удовлетворяющую на отрезке $[a,b]$ условию Гельдера.

\[
M=\{x(t)\in C[a,b]: |x(t_1)-x(t_2)|\leq K|t_1-t_2|^\alpha \quad 0 < \alpha \leq 1 \quad \forall t_1,t_2 \in [a;b]\}
\]

Рассмотрим $M_0=\{x(t)\in M:x(a)=0\}$. Легко видеть, что множество $M_0$ равномерно ограничена и равностепенно. Согласно теореме Арцела множество $M_0$ предкомпактно.

\section{Свойства непрерывных функционалов на компактных множествах}

Доказательство Теоремы Веерштраса о свойствах непрерывных функций на отрезке числовой прямой существенно опирается на компактность этого отрезка. Эти теоремы обобщаются на функциональном определении на компактном множестве абстрактного метрического пространства.

\tth[] Пусть $M\subset X$, а функционал f(x) определен на некотором множестве $x \in M$, тогда на этом множестве:

\be
  \item Функционал является ограниченным
  \item Функционал достигает на этом множестве своих точной верхней и точно нижней граней.
\ee

\chapter{Мера и интеграл Лебега}

\section{Мера Лебега множеств в арифметическом пространстве R}

Мера Лебега является обобщением понятий длины отрезка, площади плоской фигуры, объема трех или n-мерного тела.

Мера Лебега множеств в $R^n$ определяется в 3 этапа:

На \textbf{первом этапе}: Меры Лебега определяются для параллелепипедов в $R^n$, то есть для следующих множеств

\[
  \Pi = \{x=(x_1,\dots,x_n)\in R^n: x_i\in [a_i,b_i], i=1,2,\dots,n\sep a_i<b_i\}
\]

Рассмотрим параллелепипеды с произвольным включением граней ($x_i<x_i<b_i$).

Мера Лебена параллелепипеда считается равным его объему $m\Pi = \Pi{i=1}^n(b_i-a_i)$ Введенная таким образом мера обладает свойством неотрицательности.

На \textbf{втором этапе} меня Лебега определяется для элементарных множеств. Элементарным множеством, называется объединение конечного числа конечного числа параллелепипедов, то есть
\[
  A=\bigcup _{i=1}^K {\Pi}_i\sep{\Pi}_i \cap {\Pi}_j = \O\sep i\neq j
\]
причем эти параллелепипеды попарно не пересекаются.

Мера Лебега элемента множества $m'A=\sum_{i=1}^Km\Pi_i$.

Мера элемента множества обладает свойствами счетной аддитивности ($\sigma$-аддитивности)

Мера объединения счетного числа попарно непересекаемых элементарных множеств равна сумме мер этих множеств.

Совокупность элементарных множеств в $R^n$ замкнуто относительно операций пересечения, объединения и разности.

Мера m представляет собой распространение меры $m$ на элементарные множества.

\textbf{Третий этап} определение меры

Рассмотрим всевозможные множества $A$, которые содержатся внутри произвольного параллелепипеда $A\subset F \subset R^n$

\tdef Внешние меры Лебега множества $A$, называют величиной $\mu^*A$, это точная нижняя грань: $\mu^*A=inf\sum_{i=1}^\infty m\Pi_i: A\subset\bigcup _{i=1}^\infty\Pi_i\sep \Pi_i\cup\Pi_j=\emptyset\sep i\neq j$.

Внутренние меры Лебега $\mu_*A=\mu E-\mu^*(E\backslash A)$. можно доказать что $\mu_*A=sup\sum_{i=1}^\infty m\Pi_i: \bigcup _{i=1}^\infty\Pi_i \subset A\sep \Pi_i\cup\Pi_j=\emptyset\sep i\neq j$

\tdef Множество A называют измеримым по Лебегу, если совпадают его внешние и внутренние меры Лебега, то есть $\mu^*A=\mu_*A=\mu A$

Число $\mu A$ - это мера Лебега множества $A$. Меры Лебега обладают свойствами пополнения и счетной аддитивности и являются распространением меры $m'$ элементарных множеств на произвольные множества из $R^n$.

Совокупность измеримых множеств описываемыми следующими утверждениями:
\be
  \item Объединение и пересечение счетного числа измеримых множеств измеримы . разность двух измеримых множеств измеримо.
  \item Все замкнутые и открытые множества в $R^n$ измеримы.
  \item Множество состоящие из конечного или счетного числа элементов измеримо и его мера равно нулю.
  \item Полнота меры: любое подмножество множества нулевой меры измеримо и его мера равно нулю.
\ee

Свойства меры Лебега:
\be
  \item Счетная аддитивность

  Если $A=\bigcup _{i=0}^\infty A_i\sep A_i\cap A_j=\emptyset\sep i\neq j$, то $\mu A=\sum_{i=1}^\infty \mu A_i$

  \item Непрерывность меры

  Если рассматривается вложенная последовательность множеств $A_1 \supset A_0 \supset \dots \supset A_i \supset \dots$, то $\mu\bigcap _{i=1}^\infty A_i=\lim_{i \rightarrow \infty} \mu A_i $

  Для возрастающей последовательности $A \supset A _ \epsilon \supset \dots \supset A_i \supset \dots \sep \mu \bigcup _ {i = 1}  ^ \infty A_i = \lim_{i \rightarrow \infty} \mu A_i$

  \item Инвариантность относительно изометрического отображения

  Если два множества $A$ и $B$ конгруэнтны и одно из них измеримо, то измеримо и второе и их меры совпадают.
\ee

\tdef Два множества называют конгруэнтными, если одно из них является образом другого при некотором изометрическом отображении $R^n \rightarrow R^n$

\textbf{Лекция 9}

\textbf{О множествах бесконечной меры}

Разобьем все пространство $R^n$ на параллелепипеды $E_{k_1,k_2,\dots,k_n}=\{(x_1,\dots,x_n) \in R^n:k_i<x_i \leq k_i+1 \} \sep i=\overline{1,n}, k_i=0, \pm 1, \pm 2, \dots$

Множество $A$ из $R^n$ называется измеримым по Лебегу, если измеримы все множества $A \cap E_{k_1,k_2,\dots,k_n}$ и мерой этого множества называют суммой ряда.

\[\mu A: \sum_{k_1}...\sum_{R_n} \mu (A \cap E_{k_1,k_2,\dots,k_n})\]

Если ряд сходится, то мера множества $A$ - конечное число, если же ряд расходится, то мерой есть $/mu A = + \infty$ Все свойства меры переносятся и а этот случай. отличается только одно, мера объединения счетного числа попарно непересекаемых множеств может быть как конечным числом, так и бесконечностью.

Пример поясняющий своеобразне меры Лебега.

Рассмотрим множества $A$ b $B$ рациональных и иррациональных числ некоторого интервала $(a,b)$. Оба эти множества различаются по мере Лебега. $\mu A = 0$, так как это множество счетное, а $\mu B = |b-a|$. А мера Жордана этих множеств не существует, так как не совпадает нижняя и верхняя меры Жордана, это связано с тем, что при построении меры Жордана рассматриваются конечные системы параллелепипедов, а при построении меры Лебега счетные. Мера Лебга - обобщение меры Жордана.

Мера Лебега на абстрактных множествах.

Мерой Лебега на некотором абстрактном множестве $E$ называют неотрицательную функцию подмножеств этого множества обладающих свойством счетной аддитивности. Оказывается, что совокупность подмножеств абстрактного множества для которых определена мера Лебега является $\sigma$-алгебра.

$\sigma$-алгеброй называют совокупность подмножеств некоторого множества замкнутое относительно разности пересеченияя и объединения счетного числа подмножеств, а также содержит множество $E$, обозначим ее $ \sigmalgebra \sep A \subset E \sep A_i \in \sigmalgebra \sep A_i \backslash A_j \in \sigmalgebra \sep \bigcup _ {i=1}^\infty \in \sigmalgebra \sep \bigcap _ {i=1}^\infty A_i \in \sigmalgebra \sep E \in \sigmalgebra$.

$\sigmalgebra$ - это сигма-кольцо с единицей

Термин "почти всюду".

Будем говорить, что некоторое свойство выполняется на множестве $A$ почти всюду если мера множества точек $x \in A$ на котором это свойство не выполняется $=0$. Иными словами свойство выполняется почти всюду если оно выполняется на множестве во всех точках за исключением множества точек нулевой меры.

Если две функции $f(x)$ b $g(x)$ на неполном множестве $A$ совпадают почти всюду ($f(x) \eeq[\text{п. в.}] g(x) \sep x \in A$), то эти функции называют эквивалентными. для непрерыи функции называют эквивалентными. для непрерывной функции эквивалентность равносильно тождественности.
вной функции эквивалентность равносильно тождественности.

Для примера рассмотрим функцию Дирихле заданную на интервале [0,1],
\[ f(x) = \begin{cases} 1, & x \text{ - рациональный} \\ 0, & x \text{ - иррациональный} \end{cases}\]
на $[0,1]$ она эквивалентна тождественному 0.

\[f(x) \eeq[\text{п. в.}] g(x) \sep x \in A \quad \mu \{ x \in A \cdot f(x) \neq g(x) \} = 0 \quad f(x) \sim g(x)\]

\section{Измеримые функции}

Будем рассматривать функции определенные на некотором множестве $E \subset R^n$.

Функция $f(x)$ называется измеримой на $E$, если для любого действительного числа $c \quad \forall \{x: f(x) < c \}$ измеримо.

Множествами Лебега функции $f(x)$ называют множества четырех типов:

\be
  \item $\{x: f(x) < c\}$
  \item $\{x: f(x) \leq c\}$
  \item $\{x: f(x) > c\}$
  \item $\{x: f(x) \geq c\}$
\ee

Можно доказать основываясь на свойстве меры Лебега, что из измеримости множеств одного типа вытекает измеримость всех остальных типов.

Например, если первое множество измеримо, то по свойствам меры Лебега измеримо и второе множество.
\[\{x: f(x) \leq c\}= \bigcup_{k=1}^\infty\{x:f(x) < c - \cfrac{1}{k}\}\]

Если функция $f(x)$ измерима, то измеримы: само множество $E$ и множество $\{x:f(x)\leq x\}$

Свойства измеримых функций:

\be
  \item $f(x)=const \sep x \in E$ - измеримо.
  \item Всякая функция определенныя на множестве меры 0 измерима.
  \item Функция определенная на измеримом подмножестdt $E'$ множества $E$ измерима.
  \item Если функция $f(x)$ измерима на каждом $E_R$, то измерима на их объединении $\bigcup _ {k=1}^\infty E_k$
  \item Замкнутость множества измеримых функций относительно арифметических операций если $f(x)$ и $g(x)$ измеримы, то измеримы $af(x)+ bg(x)$, где $a,b$ действительные числа и $f(x)g(x)$, $\cfrac{f(x)}{g(x)}$ если $g(x) \neq 0$.
  \item Замкнутость множества измеримой функции относительно предельного перехода если функции $f_k(x)$ измеримы и $f_k(x)$ сходится к $f(x)$ п.в. $f_k(x) \xrightarrow{п.в.} f(x)$ на $E$, то $f(x)$ измеримо.
  \item Всякая функция непрерывная на множестве $E$ измерима на нем.
  \item Если две функции эквивалентны на множестве $E$ и одна из них измерима, то измерима и вторая.
\ee

\textbf{Простые функции}

Функция $f(x)$ называется простой, если она измерма и принимает не более чем счетное число значnu ений.

\tth[ о структуре простых функций] Функция $f(x)$ принимающая счетное число значений $y_k\sep k=1,2,\dots$ является простой и тогда и только тогда, когда измеримы все множества $\{x: f(x) = y_k \}$

\tth[ (критерий измеримости функций)] Для того чтобы функция $f(x)$ заданная на $E$ была измерима необходимо и достаточно чтобы существовала последовательность простых функций $f_k (x) \xrightarrow{\text{п.в.}} f(x), x \in E$.

Теорема Егорова Д.Ф. устанавливает связь между сходимостью почти всюду и равномерной сходимости. А теорема Лузина Н.Н. устанавливает связь между измеримыми функциями и непрерывными функциями.

\tth[ Егорова($\sim$ 1910)] Если последовательность $f_k(x) \xrightarrow{п.в.} f(x) \sep x \subset E$, то $\forall  > 0 \ \exists$ подмножество $ E_? \subset E : \mu E_? > \mu E - ? $ и на множестве $E_? \quad f_k(x) \rightrightarrows f(x)$ равномерно, $x \in E_?$.

\tth[ Лузина ($\sim$ 1919)] Для того чтобы функция $f(x)$ была измеримой на $[a,b]$ необходимо и достаточно чтобы существовала функция $\phi(x)$ - непрерывная на этом отрезке отличающаяся от $f(x)$ на множестве меры 0, то есть совпадающая с $f(x)$ почти всюду на этом отрезке.
\[\mu\{x:f(x) \neq \phi (x)\} = 0\]

Эту теорему можно обобщить на функции определяющие на множествах $R^n$.

%\end{multicols}
\end{document}
